\section{Introduction}
\label{sec:intro}

Local optima are a common problem plaguing the learning of parameters in
latent variable models. Recently, there has been an interest in spectral
algorithms that provide statistically consistent estimators, i.e. given
sufficient data they are guaranteed to converge to the global optimum,
given that the model is identifiable. Furthermore, these algorithms have
polynomial sample and computational complexities for several such
models, e.g.  hidden Markov models \cite{AnandkumarHsuKakade2012},
probabilistic context free grammars \cite{HsuKakadeLiang2012}, etc. In
this work, we provide a spectral algorithm for a discriminative model,
the mixture of linear regressions.

\todo{Where are mixtures of experts used? Why are they interesting?}

Our approach uses regression to recover the moments of the data from the
scalar observable output variable. Once we recover the second and third
moments of the regression coefficients, we use a slight modification of
the method outlined in \citet{AnandkumarHsuKakade2012}.

We evaluate our algorithm on simulated linear and non-linear data to
understand how well the algorithm scales with the number of components,
dimensions and separation. To test how the algorithms perform on a more
real dataset, we study their performance on a motion tracking dataset.

In \sectionref{sec:model}, we formally define the mixture of linear
regressions model and discuss when it is identifiable. We describe our
algorithm and provide some theoretical results in \sectionref{sec:algo}. We
present the results of our evaluations in \sectionref{sec:evaluation}.
Finally, we discuss the merits and demerits of spectral algorithms in
discriminative models and conclude in \sectionref{sec:discussion}. The
details of some proofs can be found in \sectionref{sec:proofs}.


