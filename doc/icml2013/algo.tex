\section{Spectral Experts Algorithm}
\label{sec:algo}

In this section, we outline our Spectral Experts algorithm
for recover the model parameters $\theta = (\pi, B, \sigma^2)$.
Our algorithm consists of two parts:
(i) low-rank regression to estimate certain symmetric tensors;
and (ii) tensor factorzation to recover the parameters.
% Motivate how we can recover $B$ from regression by lower order moments
%ideally from lower order moments.
%Conventionally, when using the method of moments, the
%moments of observed variables are expressed algebraically in terms of
%the model parameters. Using these equations, we attempt to solve for the
%model parameters using empirical estimates of the moments. 

%%% first moment
To warm-up, let us consider performing linear regression
on the response $y$ given $x$:
\begin{align}
  y &= \innerp{\bar \beta}{x} + (\innerp{\bar\beta - \beta_h}{x} + \epsilon),
\end{align}
where $\bar\beta \eqdef \sum_{h=1}^K \pi_h \beta_h$ is the average regression coefficient.
Here, there are two noise terms:
(i) the \emph{mixing noise} $\innerp{\bar\beta - \beta_h}{x}$
due to the choice of the mixture component $h$
and (ii) the \emph{observation noise} $\epsilon \sim \normal{0}{\sigma^2}$.
Performing regression on the data $\{\xni,
\yni\}_{i=1}^{N}$ would provide a consistent estimate of $\bar\beta$,
but this is clearly insufficient for identifying all the parameters $\theta$.

%%% second moment
Regressing on $x$ provides only first-order
information.  The key insight is that to get more information about the
parameters, we need to consider regressing on higher order powers,
analogous to consider higher order moments in method of moments estimators.
Let us then consider regressing $y^2$ on $x^\otimes$:
\begin{align}
  y^2 &= \innerp{\beta_h\tp{2}}{x\tp{2}} + \epsilon \innerp{\beta_h\tp{2}}{x\tp{2}} + \epsilon^2 \label{eq:y2}.
\end{align}

Let us consider some more low powers of $y$, noting
that taking the expectation gives us the corresponding low-order
moments,
\begin{align}
  y^2 &= \innerp{\beta_h\tp{2}}{x\tp{2}} + \epsilon \innerp{\beta_h\tp{2}}{x\tp{2}} + \epsilon^2 \label{eq:y2} \\ 
  y^3 &= \innerp{\beta_h\tp{3}}{x\tp{3}} + 3 \epsilon \innerp{\beta_h\tp{2}}{x\tp{2}} \notag \\
  &+ 3 \epsilon^2 \innerp{\beta_h}{x} + \epsilon^3. \label{eq:y3} 
\end{align}
Regression with the second order moments would give us $M_2 = \E[\beta_h
\tp{2}] = \sum_h \pi_h \beta_h\tp{2}$ which can identify $\pi$ and the
subspace spanned the $\beta_h$, but not the $\beta_h$ themselves. To see
this, note that the eigen-decomposition of $M_2$ is not unique.
\todo{explain better}.

This brings us to regression with the third order moments, giving us the
tensor $M_3 = \E[\beta_h \tp{3}] = \sum_h \pi_h \beta_h\tp{3}$.
\citet{AnandkumarGeHsu} show that these three moments suffice to
identify $\pi$ and $B$ and provide an efficient algorithm to recover the
parameters.

We note that the eigendecomposition of $M_2$ is not sufficient to
recover the $\beta_k$ because they are not necessarily orthogonal to
each other. \citet{AnandkumarGeHsu2012} describes an approach that uses
rotates $M_3$ to an orthogonal basis by using the whitening transform of
$M_2$. The eigenvectors and eigenvalues recovered from the
eigendecomposition of $M_3(W, W, W)$ can be de-whitened to recover the
$\beta_k$ and $\pi_k$.

% Describe the rest of the algorithm.
This description completes a sketch of the algorithm, described in
\algorithmref{algo:spectral-experts}. Going ahead, we have yet to show
that the regression is well-behaved which we will do in
\sectionref{sec:regression}. This is of concern because the regression
problem we have has variance introduced from component selection,
independent of any observation noise. We will show that we can indeed
efficiently recover $M_2$ and $M_3$ using ideas from low-rank
regression. Finally, we will outline the tensor power method to recover
$B$ and $\pi$ given these two quantities, $M_2$ and $M_3$ in
\sectionref{sec:tensor-power}. 

\cite{candesPhaseLift} regresses on the square of the response,
because that's the observation.

\begin{theorem}
  Recovery analysis of mixture of linear regressions.
\end{theorem}
\begin{proof}
\end{proof}

\begin{algorithm}[t]
  \caption{Spectral Experts}
  \label{algo:spectral-experts}
  \begin{algorithmic}[1]
    \REQUIRE $\mathcal{D} = \{ (\xn{1}, \yn{1}), \cdots, (\xn{N}, \yn{N}) \}$
    \REQUIRE $K$, the number of clusters, $\sigma^2$ 
    \STATE Recover $M_1$ using least-squares regression on $\mathcal{D}$.
    \STATE Recover $M_2, M_3$ using low-rank regression on $\mathcal{D}$.
    \STATE Return the eigenvectors $\beta_k$ and eigenvalues $\pi_k$ of $M_3$ using $M_2$.
  \end{algorithmic}
\end{algorithm}

We will conclude this section by
discussing some identifiability concerns for our algorithm.

%%%%%%%%%%%%%%%%%%%%%%%%%%%%%%%%%%%%%%%%%%%%%%%%%%%%%%%%%%%%

\subsection{Identifiability from moments}

The mixture of linear regressions model is identifiable subject to
permutations of the components\citationneeded. However, our approach
employs regression on $x\tp{2}$ and $x\tp{3}$, and thus requires
$\mathcal{X} \otimes \mathcal{X}$ and $\mathcal{X} \otimes \mathcal{X}
\otimes \mathcal{X}$ to be linearly independent modulo symmetry. We call
these condition {\em quadratic} and {\em cubic} independence.

This condition trivially holds when the space $\mathcal{X}$ consists of
independent vectors $\{ x_1, x_2, \cdots, x_d \}$. Unfortunately, it
places some constraints on non-linear featurizations. For example, the
common polynomial basis expansion, $1, x_1, x_1^2, \dots, x_1^p$ is
neither quadratically nor cubically independent because the product
space is strictly a subset of $\Re^{\frac{d (d-1)}{2}}$ since it
contains $x_1 \times x_1 = x_1^2 \times 1$, and $x_1 \times x_1 \times
x_1 = x_1^2 \times x_1 \times 1 = x_1^3 \times 1 \times 1$. In fact, the
number of unique terms in $x\tp{p}$ is only of the order of $p \times
d$, and hence we will not have enough equations to solve for $M_2$ or
$M_3$ unless we consider $p = d^2$ moments.

%%%%%%%%%%%%%%%%%%%%%%%%%%%%%%%%%%%%%%%%%%%%%%%%%%%%%%%%%%%%

% \subsection{Recovering $M_2$ and $M_3$}

% In the case of the mixture of linear regressions, we have two observed
% quantities, $x$ and $y$. Unfortunately, because $y$ is a scalar
% quantity, we have only one equation for every moment of $y$, and with
% this naive approach, would have to consider $O( (K+1) D)$ moments to
% recover $B \in \Re{K \times D}$ and $\pi \in \Re^K$. 

% To keep the discussion simple, we will consider the case where there is
% no noise, i.e.  $\sigma^2 = 0$. In this case, the moments of $y$ are,
% \begin{eqnarray*}
%   \E[ y ] &=& (\sum_{k=1}^{K} \pi_{k} \beta_k)^T x \\
%           &=& M_1^T \E[x] \\
%   \E[ y^2 ] &=& \sum_{k=1}^{K} \pi_{k} \E[(\beta_k^T x)]^2 \\
%   &=& \sum_{k=1}^{K} \pi_{k} (\beta_k\tp{2} \odot \E[x\tp{2}]) \\
%   &=& M_2 \odot \E[x\tp{2}] \\
%   \E[ y^p ] &=& M_p \odot \E[x\tp{p}].
% \end{eqnarray*}
% 
% We note that joint moments with $x$ will not help either; consider
% \begin{eqnarray}
%   \E[ y^p x^q ] &=& M_p \odot E[ x\tp{p+q} ].
% \end{eqnarray}
% The elements of $\E[ y^p x^q ]$ are linear combinations of $\E[y^p]$,
% and hence we do not get any new equations; \todo{This isn't exactly
% correct; should we even be presenting this line of reasoning?}. This is
% to be expected, since any information about the $\beta$s is really
% encoded in $y$, and not in the $X$.

