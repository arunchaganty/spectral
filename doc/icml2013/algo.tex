\section{Spectral Experts algorithm}
\label{sec:algo}

In this section, we describe our Spectral Experts algorithm
for estimating model parameters $\theta = (\pi, B)$.
The algorithm consists of two steps:
(i) low-rank regression to estimate certain symmetric tensors;
and (ii) tensor factorization to recover the parameters.
The two steps can be performed efficiently using
convex optimization and tensor power method, respectively.

%%% first moment
To warm up, let us consider linear regression
on the response $y$ given $x$.
From the model definition, we have $y = \beta_h^\top x + \epsilon$.
The challenge is that the regression coefficients $\beta_h$ depend on the random $h$.
%$\epsilon \sim \normal{0}{\E[\epsilon^2]}$, and $h$ is a random quantity
%independent of $x$.
The first key step is to average over this randomness by defining
average regression coefficients
$M_1 \eqdef \sum_{h=1}^k \pi_h \beta_h$.
Now we can express $y$ as a linear function of $x$ with non-random coefficients $M_1$
plus some a new noise term $\eta_1(x)$:
\begin{align}
  y &= \innerp{M_1}{x} +
  \underbrace{(\innerp{\beta_h - M_1}{x} + \epsilon)}_{\eqdef \eta_1(x)}. \label{eqn:y1}
\end{align}
The noise $\eta_1(x)$ is the sum of two terms:
(i) the \emph{mixing noise} $\innerp{M_1 - \beta_h}{x}$
due to the random choice of the mixture component $h$,
and (ii) the \emph{observation noise} $\epsilon \sim \sE$.
Although the noise depends on $x$,
it still has zero mean conditioned on $x$.
We will later show that we can
perform linear regression on the data $\{\xni,
\yni\}_{i=1}^{n}$ to produce a consistent estimate of $M_1$.
%under identifiability conditions.
But clearly, knowing $M_1$ is insufficient
for identifying all the parameters $\theta$,
as
$M_1$ only contains $d$ degrees of freedom whereas $\theta$ contains $O(kd)$.

%%% second moments
Intuitively, performing regression on $y$ given $x$ provides only first-order
information.  The second key insight is that we can perform regression
on higher-order powers to obtain more information about the parameters.
Specifically, for an integer $p \ge 1$, let us define the average
$p$-th order tensor power of the parameters as follows:
\begin{align}
M_p &\eqdef \sum_{h=1}^k \pi_h \beta_h\tp{p}. \label{eqn:Mp} % \in (\Re^{d})\tp{p}.
\end{align}
Now consider performing regression on $y^2$ given $x\tp{2}$.
Expanding $y^2 = (\innerp{\beta_h}{x} + \epsilon)^2$,
using the fact that $\innerp{\beta_h}{x}^p = \innerp{\beta_h\tp{p}}{x\tp{p}}$,
we have:
\begin{align}
  y^2 &= \innerp{M_2}{x\tp{2}} + \E[\epsilon^2] + \eta_2(x), \label{eqn:y2} \\
\eta_2(x) &= \innerp{\beta_h\tp{2} - M_2}{x\tp{2}} + 2 \epsilon \innerp{\beta_h}{x} + (\epsilon^2 - \E[\epsilon^2]). \nonumber
\end{align}
Again, we have expressed $y^2$ has a linear function of $x\tp{2}$
with regression coefficients $M_2$, plus a known bias $\E[\epsilon^2]$ and noise.\footnote{If $\E[\epsilon^2]$ were not known,
we could treat it as another coefficient
to be estimated.  The coefficients $M_2$ and $\E[\epsilon^2]$ can be estimated jointly
provided that $x$ does not already contain a bias ($x_j$ must be non-constant for every $j \in [d]$).}
Importantly, the noise has mean zero; 
in fact each of the three terms has zero mean
by definition of $M_2$ and independence of $\epsilon$ and $h$.

Performing regression yields a consistent estimate of $M_2$,
but still does not identify all the parameters $\theta$.
In particular, $B$ is only identified up to rotation:
if $B = [\beta_1 \mid \cdots \mid \beta_k]$ satisfies
$B \diag(\pi) B^\top = M_2$ and $\pi$ is uniform, then $(B Q) \diag(\pi) (Q^\top B^\top) = M_2$
for any orthogonal matrix $Q$.

%%% third moment
Let us now look to the third moment for additional information.
We can write $y^3$ as a linear function of $x\tp{3}$ with coefficients $M_3$,
a bias $3 \E[\epsilon^2] \innerp{M_1}{x} + \E[\epsilon^3]$ that can be estimated from \refeqn{y1}, and zero mean noise $\eta_3(x)$:
\begin{align}
  y^3 &= \innerp{M_3}{x\tp{3}} + 3\E[\epsilon^2] \innerp{M_1}{x} + \E[\epsilon^3] + \eta_3(x), \nonumber \\
\eta_3(x) &= \innerp{\beta_h\tp{3} - M_3}{x\tp{3}}
+ 3 \epsilon \innerp{\beta_h\tp{2}}{x\tp{2}} \label{eqn:y3} \\
&\quad + 3(\epsilon^2 \innerp{\beta_h}{x} - \E[\epsilon^2] \innerp{M_1}{x})
+ (\epsilon^3 - \E[\epsilon^3]). \nonumber
\end{align}
Performing this regression yields an estimate of $M_3$.
We will see shortly that knowledge of $M_2$ and $M_3$ are sufficient to recover
all the parameters.

\begin{algorithm}[t]
  \caption{Spectral Experts}
  \label{algo:spectral-experts}
  \begin{algorithmic}[1]
    \INPUT Datasets $\mathcal{D}_p = \{ (\xn{1}, \yn{1}), \cdots, (\xn{n}, \yn{n}) \}$ for $p = 1, 2, 3$;
    %regularization strengths $\lambda_n^{(2)} = \frac{c_2}{\sqrt{n}}$, $\lambda_n^{(3)} = \frac{c_3}{\sqrt{n}}$;
    regularization strengths $\lambda_n^{(2)}$, $\lambda_n^{(3)}$;
    observation noise moments $\E[\epsilon^2], \E[\epsilon^3]$.
    \OUTPUT Parameters $\hat\theta = (\hat \pi, [\hat \beta_1 \mid \cdots \mid \hat \beta_k])$.
    \STATE Estimate compound parameters $M_2, M_3$ using \textbf{low-rank regression}:
    \begin{align}
      &\hat M_1 = \arg\min_{M_1} \label{eqn:estimateM1} \\
      &\quad\frac{1}{2n}\sum_{(x,y) \in \sD_1} (\innerp{M_1}{x} - y)^2, \nonumber \\
      &\hat M_2 = \arg\min_{M_2} \quad \lambda_n^{(2)} \|M_2\|_* + \label{eqn:estimateM2} \\
      &\quad\frac{1}{2n}\sum_{(x,y) \in \sD_2} (\innerp{M_2}{x\tp{2}} + \E[\epsilon^2] - y^2)^2, \nonumber \\
      &\hat M_3 = \arg\min_{M_3} \quad \lambda_n^{(3)} \|M_3\|_* + \label{eqn:estimateM3} \\
      &\quad\frac{1}{2n}\sum_{(x,y) \in \sD_3} (\innerp{M_3}{x\tp{3}} + 3 \E[\epsilon^2]\innerp{\hat M_1}{x} + \E[\epsilon^3] - y^3)^2. \nonumber
    \end{align}
    \STATE Estimate parameters $\theta = (\pi, B)$ using \textbf{tensor factorization}:
    \begin{enumerate}
      \item [(a)] Compute whitening matrix $\hat W \in \Re^{d \times k}$ (such that $\hat W^\top
      \hat M_2 \hat W = I$) using SVD.
      \item [(b)] Compute eigenvalues $\{\hat a_h\}_{h=1}^k$
      and eigenvectors $\{\hat v_h\}_{h=1}^k$
      of the whitened tensor $\hat M_3(\hat W, \hat W, \hat W) \in \Re^{k \times k \times k}$
      by using the robust tensor power method.
    \item [(c)] Return parameter estimates $\hat\pi_h = \hat a_h^{-2}$
    and $\hat\beta_h = (\hat W^{\top})^\dagger (\hat a_h \hat v_h)$.
    \end{enumerate}
  \end{algorithmic}
\end{algorithm}


% Full algorithm
Now we are ready to state our full algorithm, which we call Spectral Experts
(\algorithmref{algo:spectral-experts}).
First, we perform three regressions to recover the \emph{compound parameters}
$M_1$ \refeqn{y1},
$M_2$ \refeqn{y2}, and
$M_3$ \refeqn{y3}.
Since $M_2$ and $M_3$ both only have rank $k$,
we can use nuclear norm regularization
\cite{Tomioka2011,NegahbanWainwright2009}
to exploit this low-rank structure and improve our compound parameter estimates.
In the algorithm, the regularization strengths $\lambda_n^{(2)}$ and $\lambda_n^{(3)}$
are set to $\frac{c}{\sqrt{n}}$ for some constant $c$.
%The resulting semidefinite program is a standard one which has received
%much attention in recent years.
% We use a standard off-the-shelf SDP solver, CVX, to solve .
% We use a rather simple proximal gradient-based approach,
% in which the nuclear norm is handled in closed form by taking an SVD
% and soft-thresholding the singular values \cite{donoho95soft,cai10soft}.

% Tensor factorization
Having estimated the compound parameters $M_1$, $M_2$ and $M_3$, it
remains to recover the original parameters $\theta$.
\citet{AnandkumarGeHsu2012} showed that for $M_2$ and $M_3$ of
the forms in \refeqn{Mp}, it is possible to efficiently accomplish this.
Specifically, we first compute a whitening matrix $W$ based on the SVD of $M_2$
and use that to construct a tensor $T = M_3(W, W, W)$ whose factors are orthogonal.
We can use the robust tensor power method to compute all the
eigenvalues and eigenvectors of $T$, from which it is easy to recover
the parameters $\pi$ and $\{\beta_h\}$.

\paragraph{Related work}

% Spectral
In recent years, there has a been a surge of interest in ``spectral'' methods
for learning latent-variable models.  One line of work has
focused on observable operator models \cite{hsu09spectral,song10kernel,parikh12spectral,cohen12pcfg,balle11transducer,balle12automata}
in which a reparametrization of the true parameters are recovered,
which suffices for prediction and density estimation.
Another line of work is based on the method of moments and uses eigendecomposition of a certain tensor
to recover the parameters \cite{anandkumar12moments,anandkumar12lda,hsu12identifiability,hsu13spherical}.
Our work extends this second line of work to models that
require regression to obtain the desired tensor.

% Unmixing
In spirit, Spectral Experts bears some resemblance to the unmixing
algorithm for estimation of restricted PCFGs
\cite{hsu12identifiability}.
In that work, the observations (moments) provided a linear combination over
the compound parameters.  ``Unmixing'' involves solving for the compound
parameters by inverting a mixing matrix.
In this work,
each data point (appropriately transformed) provides a different noisy projection of
the compound parameters.

% 
%Previous work has focused on using spectral methods to learn finite state automata and transducers.
Other work has focused on learning discriminative models,
notably \citet{balle11transducer} for finite state transducers (functions from strings to strings),
and \citet{balle12automata} for weighted finite state automata (functions from strings to real numbers).
Similar to Spectral Experts,
\citet{balle12automata} used a two-step approach,
where convex optimization is first used to estimate moments (the Hankel matrix in their case),
after which these moments are subjected to spectral decomposition.  
However, these methods are developed in the observable operator framework, whereas we consider parameter estimation.

% Signal
The idea of performing low-rank regression on $y^2$ has been explored
in the context of signal recovery from magnitude measurements
\cite{candes11phaselift,ohlsson12phase}.
There, the actual observed response was $y^2$,
whereas in our case, we deliberately construct powers $y,y^2,y^3$
to identify the underlying parameters.

%\citet{AnandkumarGeHsu2012} describes an approach that uses
%rotates $M_3$ to an orthogonal basis by using the whitening transform of
%$M_2$. The eigenvectors and eigenvalues recovered from the
%eigendecomposition of $M_3(W, W, W)$ can be de-whitened to recover the
%$\beta_k$ and $\pi_k$.

% Describe the rest of the algorithm.
%This description completes a sketch of the algorithm, described in
%\algorithmref{algo:spectral-experts}. Going ahead, we have yet to show
%that the regression is well-behaved which we will do in
%\sectionref{sec:regression}. This is of concern because the regression
%problem we have has variance introduced from component selection,
%independent of any observation noise. We will show that we can indeed
%efficiently recover $M_2$ and $M_3$ using ideas from low-rank
%regression. Finally, we will outline the tensor power method to recover
%$B$ and $\pi$ given these two quantities, $M_2$ and $M_3$ in
%\sectionref{sec:tensor-power}. 

