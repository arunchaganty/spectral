\documentclass[tablecaption=bottom]{jmlr}

\usepackage[cm]{fullpage}
\usepackage{booktabs}
\usepackage{ctable}

\title{Spectral Experts}

\author{\Name{Arun Tejasvi Chaganty} \Email{chaganty@stanford.edu}}

% Text
\newcommand{\todo}[1]{\hl{\textbf{TODO:} #1}}
\newcommand{\citationneeded} {\ensuremath{^{[\textrm{citation needed}]}}}


%Math Operators
%\DeclareMathOperator {\argmax} {argmax}
%\DeclareMathOperator {\argmin} {argmin}
\DeclareMathOperator {\sgn} {sgn}
\DeclareMathOperator {\trace} {tr}
\DeclareMathOperator{\E} {\mathbb{E}}
\DeclareMathOperator{\Var} {Var}
\DeclareMathOperator{\diag} {diag}
\DeclareMathOperator{\triu} {triu}
\DeclareMathOperator{\mult} {Multinomial}
\DeclareMathOperator{\normalt} {Normal}
\DeclareMathOperator{\cvec} {cvec}

\newcommand{\ud}{\, \mathrm{d}}
\newcommand{\diff}[1] {\frac{\partial}{\, \partial #1}}
\newcommand{\difff}[2] {\frac{\partial^2}{\, \partial #1\, \partial #2}}
\newcommand{\diffn}[2] {\frac{\partial^{#2}}{\, \partial {#1}^{#2}}}
\newcommand{\tuple}[1] {\langle #1 \rangle}
\newcommand{\innerprod}[2] {\langle #1, #2 \rangle}

% Constants/etc.
\renewcommand{\Re} {\mathbb{R}}
\newcommand{\Cm} {\mathbb{C}}
\newcommand{\Qm} {\mathbb{Q}}
\newcommand{\half} {\frac{1}{2}}

\newcommand{\inv}[1] {{#1}^{-1}}

\newcommand{\normal}[2] {\mathcal{N}(#1, #2)}
\newcommand{\mL} {\mathcal{L}}

\newcommand\eqdef{\ensuremath{\stackrel{\rm def}{=}}} % Equal by definition
\newcommand\refeqn[1]{(\ref{eqn:#1})}
\newcommand\sD{\ensuremath{\mathcal{D}}}
\newcommand\sM{\ensuremath{\mathcal{M}}}
\newcommand\refapp[1]{Appendix~\ref{sec:#1}}
\newcommand\refthm[1]{Theorem~\ref{thm:#1}}
\newcommand\sigmamin{\sigma_\text{\rm min}}
\newcommand\sigmamax{\sigma_\text{\rm max}}
\newcommand\op{{\text{\rm op}}}
\newcommand\BP{\ensuremath{\mathbb{P}}}
\newcommand\reflem[1]{Lemma~\ref{lem:#1}}


\begin{document}

\maketitle

\newcommand{\tp}

{[}1{]}\{\^{}\{\otimes \#1\}\}
\newcommand{\opX}{\mathfrak{X}}

\newcommand{\cvec}{\textrm{cvec}}

\section{Introduction}

\begin{itemize}
\item
  Latent variable models (HMM, mixture models, CRFs, etc.) are extremely
  powerful models of data we've developed.
\item
  However, learning parameters in these models is typically hard because
  their likelihood functions are not convex in the parameters.
\item
  Most common approach is to use local methods like EM, variational
  techniques, etc.
\item
  Local optima are a serious concern when using these methods.
\item
  Importantly, this is a problem that does not necessarily go away with
  more and more data!
\item
  Recently, there have been a number of \emph{consistent estimators}
  proposed based on the method of moments \todo{Include citations}.
  \begin{itemize}
  \item
    These are called ``spectral methods'' for how they utilize spectral
    decompositions to recover the parameters.
  \end{itemize}
\item
  Our work extends this approach of learning parameters to the
  discriminative setting wherein the moments of the parameters are not
  directly observed.
  \begin{itemize}
  \item
    The crux of our approach will be to use regression to first learn
    these moments, followed by application of tensor decomposition to
    learn the parameters.
  \end{itemize}
\end{itemize}
\section{Background}

\subsection{Method of Moments}

\begin{itemize}
\item
  Let us study how method of moment estimators work in general.
\item
  Consider a moment map $\mathcal{M}$ that maps the parameters $\theta$
  to the moments $m$. For a Gaussian, we have that
  $\mathcal{M} = (\mu, \sigma^2)$.
\item
  In general, we will compute the inverse of the moment map to learn the
  parameters from the sample estimates.
\item
  By the central limit theorem, our sample estimates of the moments
  converge at a $1/\sqrt{n}$ rate, so we expect that our parameters will
  also converge at this rate.
\end{itemize}
\subsection{Mixture of Regressions}

\begin{itemize}
\item
  The mixture of linear regressions model defines a conditional
  distribution over a response $y \in \Re$ given covariates
  $x \in \Re^d$.
\item
  The generative procedure is as follows,
  \begin{enumerate}[i)]
  \item
    Draw a mixture component $h \in [k] \sim Mult(\pi)$, where
    $\pi = [\pi_1 | ... | \pi_k]$ defines the mixture proportions.
  \item
    Draw the noise $\epsilon \sim \mathcal{E}$, where $\mathcal{E}$ is
    the noise distribution.
  \item
    Set $y = \beta_h^T x + \epsilon$, where $\{\beta_h\}_{h=1}^{k}$ are
    the conditional means of the regression coefficients.
  \end{enumerate}
\item
  The parameters that we would like to learn from this model are $\pi$
  and $B = [ \beta_1 | ... | \beta_k ]$.
\item
  The challenge in our scenario is that the moments of the data give us
  very filtered information of the parameters.
\end{itemize}
\section{Algorithm}

\subsection{Recovering the moments}

\begin{itemize}
\item
  As noted earlier, the first problem we run into is that we can't
  observe the moments of the parameters $B$ and $\pi$ directly!
\item
  However, observe that
  \begin{align}
    y &= \innerp{\beta_h}{x} + \epsilon \\
      &= \innerp{M_1}{x} + \underbrace{ {\beta_h - M_1}{x} + \epsilon }_{\textrm{noise}},
  \end{align}

  where $M_1 = \sum_{h=1}^k \pi_h \beta_h$, the mean regression
  coefficent. We note that while the noise term is dependent on $x$, it
  has a zero-mean. Thus, we could potentially recover $M_1$ through
  regression.
\item
  However, the first moments are insufficient to learn this model.
\item
  Let's look at the second and third moments of the data,
  \begin{align*}
    y^2 &= (\innerp{\beta_h}{x} + \epsilon)^2 \\
      &= \innerp{M_2}{x\tp{2}} + \E[\epsilon^2] +
       \underbrace{ \innerp{\beta_h\tp{2} - M_2}{x\tp{2}} + \innerp{\beta_h\tp{1} - M_1}{x\tp{1}} + (\epsilon^2 - \E[\epsilon^2]) }_{\textrm{noise}}, \\
    y^3 &= (\innerp{\beta_h}{x} + \epsilon)^3 \\
      &= \innerp{M_3}{x\tp{3}} + 3\E[\epsilon^2] \innerp{M_1}{x}  + \E[\epsilon^3] + {\textrm{noise}},
  \end{align*}
\item
  On it's own, $M_2$ is insufficient to identify the model, because it
  is invariant to rotations of $B$.
\item
  However, it turns out that $M_2$ and $M_3$ are sufficient to identify
  the model, via tensor decomposition, if $k < d$.
\item
  An additional fact that we can exploit is that both $M_2$ and $M_3$
  are low rank, so we can use low-rank regression to recover estimates
  $\hat M_2$ and $\hat M_3$ efficiently from data.
\end{itemize}
\subsubsection{Caveat: Requirements for Identifiability}

\begin{itemize}
\item
  In ordinary linear regression, the regression coefficients
  $\beta \in \Re^d$ are identifiable if and only if the data has full
  rank: $\E[x\tp{2}] \succ 0$.
\item
  However, because we need regression on higher moments to recover $M_2$
  and $M_3$, we also need that $\E[\cvec(x\tp{p})\tp{2}] \succ 0$ for
  $p \in \{1,2,3\}$.
\item
  This has some subtle implications when the features are completely
  independent of each other.
\item
  For example, if $x = (1, t, t^2)$, then $\E[\cvec(x\tp{2})\tp{2}]$ is
  singular for any data distribution.
\end{itemize}
\subsection{Recovering the parameters}

\begin{itemize}
\item
  Now that we've the moments, let us review how the tensor decomposition
  technique can be used to learn the parameters.
\item
  The method exploits the fact that $M_2$ and $M_3$ share a basis,
  \begin{align*}
      M_2 &= \sum_{h=1}^k \pi_h \beta_h\tp{2} \\
      M_3 &= \sum_{h=1}^k \pi_h \beta_h\tp{3}.
    \end{align*}
\item
  Decompositions are not unique however, but we can use the whitening
  transformation for $M_2$ to give \emph{whiten} $M_3$ such that it has
  a orthogonal decomposition.
  \begin{align*}
      I &= W^T M_2 W  \\
        &= \sum_{h=1}^k (\underbrace{\sqrt{\pi_h} W^T \beta_h}_{v_h})\tp{2} \\
      M_3(W,W,W) &= \sum_{h=1}^k \pi_h (W^T \beta_h)\tp{3} \\
                 &= \sum_{h=1}^k \frac{1}{\sqrt{\pi_h}} v_h\tp{3}.
    \end{align*}
\item
  The robust tensor power method by \cite{AnandkumarHsuGe2012} find
  stable eigenvectors using the following iterative algorithm,
  $v_h \to \frac{T(I, v_h, v_h)}{||T(I, v_h, v_h)||_2}.$
\end{itemize}
\section{Theorem: Rates of Recovery}

\begin{itemize}
\item
  The rate of convergence for the spectral experts algorithm to the true
  parameters breaks into two parts; the rates for learning the moments,
  which feeds into the rates for learning the parameters.
\item
  \todo{Diagram showing how the error breaks down.}
\item
  For low rank regression, we have the following bound on recovery by
  (Tomioka2011);
  \[ || \hat M_p - M_p ||_F \le \frac{32 \lambda^{(p)}_n \sqrt{k}}{\kappa(\opX_p)}, \]
  where $\kappa(\opX_p)$ is the (restricted) strong convexity constant,
  and $\lambda^{(p)} > ||\opX^*_p(\eta)||$.
\item
  Because we assume our noise is bounded, it is easy to show that the
  error concentrates.
\item
  In the tensor recovery case, we will need to whiten $M_3$ before
  applying the tensor decomposition and unwhiten it afterwards; this
  modifies the error bounds slightly.
\end{itemize}
\section{Spectral Experts in Practice}

\begin{itemize}
\item
  We simulated the performance of spectral experts and compared it to
  EM.
  \begin{itemize}
  \item
    We generated data from the mixtures of linear regressions model,
    with $x$ drawn uniformly in $[-1,1]^d$.
  \end{itemize}
\item
  In a non-trivial number of cases, EM did not converge to the right
  parameters and got stuck in local optima.
\item
  As a specific instance, we studied this example,
  \todo{Include diagram}.
  \begin{itemize}
  \item
    Only 13 of a 100 attempts with EM successfully identified the true
    parameters.
  \item
    On the other hand, even with $O(10^5)$ samples, the parameters by
    the spectral method weren't great.
  \item
    However, EM when initialized with these parameters did extremely
    well.
  \end{itemize}
\item
  This finding that spectral methods should be a good initialization for
  EM is not surprising.
  \begin{itemize}
  \item
    The biggest sell for spectral methods is that they give a global
    guarantee on where the parameters are.
  \item
    The parameters might not be at the global optima, but will hopefully
    lie in the potential well around it.
  \item
    EM will then converge to the global optima.
  \item
    \todo{Include diagram.}
  \end{itemize}
\item
  \todo{Summarize Other experiments.}
  \begin{itemize}
  \item
    Be frank, this isn't going to replace EM on it's own, but perhaps it
    highlights a principled approach to initializing EM.
  \end{itemize}
\end{itemize}
\section{Conclusions}

\begin{itemize}
\item
  We can learn the parameters of conditional models where the observed
  moments on their own have such sparse information.
\item
  The key intuition is that we could construct a regression problem to
  learn the moments of the parameters from their projections onto the
  data.
\item
  We found that while the parameters learned this way are usually not
  better than those learned via local methods like EM, etc., they are a
  good way to initialize these local methods.
\end{itemize}

\end{document}
