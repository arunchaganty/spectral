\section{Proofs: Regression}
\label{sec:proofs:regression}

\iftoggle{withappendix}{}{
\todo{Fix the setup so that the bias is captured.}

Let us review the regression problem set up in \cite[Section 3]{ChagantyLiang2013}. 
We assume we are given data $(x_i,y_i) \in \sD_p$
generated by the following process,
\begin{align*}
  y_i &= \innerp{M_p}{x_i\tp{p}} + \eta_p(x_i),
\end{align*}
where $M_p = \sum_{h=1}^k \pi_h \beta_h\tp{p}$, the p-th order moments
of $\beta_h$ and $\eta_p(x)$ is zero mean noise. In particular, for $p \in \{1,2,3\}$,
we showed that $\eta_p(x)$ were defined to be,
\begin{align}
  \eta_1(x) &= \innerp{\beta_h - M_1}{x} + \epsilon \label{eqn:eta1} \\
  \eta_2(x) &= \innerp{\beta_h\tp{2} - M_2}{x\tp{2}} + 2 \epsilon \innerp{\beta_h}{x} + (\epsilon^2 - \E[\epsilon^2]) \label{eqn:eta2}\\
  \eta_3(x) &= \innerp{\beta_h\tp{3} - M_3}{x\tp{3}}
        + 3 \epsilon \innerp{\beta_h\tp{2}}{x\tp{2}} 
        + 3(\epsilon^2 \innerp{\beta_h}{x} - \E[\epsilon^2] \innerp{M_1}{x})
        + (\epsilon^3 - \E[\epsilon^3]). \label{eqn:eta3}
\end{align}
We assume that $\|x_i\| \le R$, $\| \beta_h \| \le L$ and $|\epsilon| \le S$.

We then defined the observation operator $\opX_p(M_p) : \Re^{d\tp{p}} \to \Re^{n}$,
\begin{align*}
\opX_p(M_p; \sD_p)_i &\eqdef \innerp{M_p}{x\tp{p}_i}, & (x_i, y_i) \in \sD_p,
\end{align*}
which let us succinctly represent the low-rank regression problem as follows,
\begin{align*}
  \min_{M_p \in \Re^{d\tp{p}}} \frac{1}{2n} \| y - \opX_p(M_p; \sD_p) \|^2_2 + \lambda_p \|M_p\|_*.
\end{align*}

Let us also recall the adjoint of the observation operator, $\opX_p^* : \Re^{n} \to \Re^{d^p}$,
\begin{align*}
  \opX_p^*(\eta_p; \sD_p) &= \sum_{x \in \sD_p} \eta_p(x) x\tp{p},
\end{align*}
where we have used $\eta_p$ to represent the vector $\left[\eta_p(x)\right]_{x \in \sD_p}$. 

\citet{Tomioka2011} showed that error in the estimated $\hat M_p$ can be
bounded as follows;

\begin{lemma}[\citet{Tomioka2011}, Theorem 1]
\label{lem:lowRank}
Suppose there exists a restricted strong convexity constant $\kappa(\opX_p)$ such that
$$\frac{1}{2n} \| \opX_p( \Delta )\|_2^2 \ge \kappa(\opX_p) \|\Delta\|^2_F \quad \text{and} \quad
\lambda_n \ge \frac{\|\opX_p^*(\eta_p)\|_\op}{n}.$$
Then the error of $\hat M_p$ is bounded as follows:
$\| \hat M_p - M_p^* \|_F \le \frac{\lambda_n \sqrt{k}}{\kappa(\opX_p)}$.
\end{lemma}
} % endif

%%%%%%%%%%%%%%%


In this section, we will derive an upper bound on $\kappa(\opX_p)$ and
a lower bound on $\frac{1}{n} \| \opX_p^*(\eta_p) \|_\op$.

\begin{lemma}[Lower bound on restricted strong convexity]
\label{lem:lowRankLower}
Let $\Sigma_p \eqdef \E[\cvec(x\tp{p})\tp{2}]$.
If $$n \ge \frac{16 (p!)^2 R^{4p}}{\sigmamin(\Sigma_p)^2} \left(1 + \sqrt{\frac{\log(1/\delta)}{2}}\right)^2,$$
then, with probability at least $1-\delta$,
$$\kappa(\opX_p) \ge \frac{\sigmamin(\Sigma_p)}{2}.$$
\end{lemma}

\begin{proof}
  Recall that $\kappa(\opX_p)$ is defined to be a constant such that the following inequality holds, 
  $$\frac{1}{n} \|\opX_p(\Delta)\|_2^2 \ge \kappa(\opX_p) \|\Delta\|^2_F,$$
where
$$\|\opX_p(\Delta)\|_2^2 = \sum_{(x,y) \in \sD_p} \innerp{\Delta}{x\tp{p}}^2.$$
To proceed, we will unfold the tensors $\Delta$ and $x\tp{p}$ to get a lower bound in terms of $\|\Delta\|^2_F$. This will allow us to choose an appropriate value for $\kappa(\opX_p)$ that will hold with high probability.

First, note that $x\tp{p}$ is symmetric, and thus
$\innerp{\Delta}{x\tp{p}} = \innerp{\cvec{\Delta}}{\cvec{x\tp{p}}}$.
This allows us to simplify $\|\opX_p(\Delta)\|_2^2$ as follows,
\begin{align*}
  \frac{1}{n} \|\opX_p(\Delta)\|_2^2 
    &= \frac{1}{n} \sum_{(x,y) \in \sD_p} \innerp{\Delta}{x\tp{p}}^2 \\
    &= \frac{1}{n} \sum_{(x,y) \in \sD_p} \innerp{\cvec(\Delta)}{\cvec(x\tp{p})}^2 \\
    &= \frac{1}{n} \sum_{(x,y) \in \sD_p} \trace( \cvec(\Delta)\tp{2} \cvec(x\tp{p})\tp{2} ) \\
    &= \trace\left( \cvec(\Delta)\tp{2} \frac{1}{n} \sum_{(x,y) \in \sD_p} \cvec(x\tp{p})\tp{2} \right).
\end{align*}
Let $\hat\Sigma_p \eqdef \frac{1}{n}
\sum_{(x,y) \in \sD_p} \cvec(x\tp{p})\tp{2}$, so that $\frac{1}{n}
\|\opX_p(\Delta)\|_2^2 = \trace(\cvec(\Delta)\tp{2} \hat\Sigma_p)$. 
For symmetric $\Delta$, $\|\cvec(\Delta)\|_2 = \|\Delta\|_F$. \todo{How can we make this assumption?}
Then, we have 
\begin{align*}
\frac{1}{n} \|\opX_p(\Delta)\|_2^2 
  &= \trace(\cvec(\Delta)\tp{2} \hat\Sigma_p) \\
  &\ge \sigmamin(\hat\Sigma_p) \|\Delta\|_F^2.
\end{align*}
By Weyl's theorem, $$\sigmamin(\hat\Sigma_p) \ge
\sigmamin(\Sigma_p) - \|\hat\Sigma_p - \Sigma_p\|_\Lop.$$
Since $\|\hat\Sigma_p - \Sigma_p\|_\Lop \le \|\hat\Sigma_p - \Sigma_p\|_{F}$,
it suffices to show that the empirical covariance concentrates in Frobenius norm.
Applying \reflem{conc-norms}, with
probability at least $1 - \delta$, $$\| \hat\Sigma_p - \Sigma_p \|_F
\le \frac{2 \|\Sigma_p\|_F}{\sqrt n} \left( 1 + \sqrt{\frac{\log(1/\delta)}{2}} \right).$$
Now we seek to control $\|\Sigma_p\|_F$.
Since $\|x\|_2 \le R$, we can use the
bound $$\| \Sigma_p \|_F \le p! \| \vvec(x\tp{p})\tp{2} \|_F \le p! R^{2p}.$$

Finally, $\|\hat\Sigma_p - \Sigma_p\|_\op \le \sigmamin(\Sigma_p)/2$ with probability at least $1 - \delta$ if,
$$n \ge \frac{16 (p!)^2 R^{4p}}{\sigmamin(\Sigma_p)^2} \left(1 + \sqrt{\frac{\log(1/\delta)}{2}}\right)^2.$$

% To do this, we first apply Hoeffding's inequality elementwise.
% Since $\|x\|_2 \le R$, we have that for each element $(i,j)$,
% $|(\hat\Sigma_p)_{ij} - (\Sigma_p)_{ij}| = O(R^{2p}\sqrt{\frac{\log (1/\delta)}{n}})$.
% Applying the union bound over the $d^{2p}$ elements of $\hat\Sigma_p - \Sigma_p$,
% we have that the max norm is bounded:
% $\|\hat\Sigma_p - \Sigma_p\|_\text{\rm max} = O(R^{2p} \sqrt{\frac{p \log(d) \log (1/\delta)}{n}})$.
% The max norm times $d^p$ upper bounds the Frobenius norm, which upper bounds the operator norm, so we have that
% $\|\hat\Sigma_p - \Sigma_p\|_\text{\rm op} = O(d^p R^{2p} \sqrt{\frac{p \log(d) \log (1/\delta)}{n}})$.
% Using the fact that $\sigmamin(\hat\Sigma_p) \ge \sigmamin(\Sigma_p) - \|\hat\Sigma_p - \Sigma_p\|_\text{\rm op}$
% yields the result.
\end{proof}

\begin{lemma}[Upper bound on adjoint operator]
\label{lem:lowRankUpper}
With probability at least $1-\delta$, the following holds,
\begin{align*}
  \frac{1}{n} \|\opX_1^*(\eta_1)\|_\op
      &\le 2 \frac{R (2LR + S)}{\sqrt{n}} \left( 1 + \sqrt{\frac{\log(3/\delta)}{2}} \right) \\
  \frac{1}{n}  \|\opX_2^*(\eta_2)\|_\op 
      &\le 2 \frac{(4L^2 R^2 + 2 S L R + 4S^2)R^2}{\sqrt{n}} \left( 1 + \sqrt{\frac{\log(3/\delta)}{2}} \right) \\
  \frac{1}{n} \|\opX_3^*(\eta_3)\|_\op 
      &\le 2 \frac{(8L^3 R^3 + 3 L^2 R^2 S + 6 L R S^2 + 2S^3) R^3}{\sqrt{n}} \left( 1 + \sqrt{\frac{\log(6/\delta)}{2}} \right) \\
  &\quad + 3 R^4 S^2 \left(\frac{128R(2LR+S)}{\sigmamin(\Sigma_1) \sqrt{n}} \left(1 + \sqrt{\frac{\log(6/\delta)}{2}} \right) \right).
\end{align*}
\end{lemma}

It follows that, with probability at least $1-\delta$,
\begin{align*}
  \frac{1}{n} \|\opX_p^*(\eta_p)\|_\op
  &= O\left( L^p S^p R^{2p} \sigmamin(\Sigma_1)^{-1} \sqrt{\frac{\log(1/\delta)}{n}} \right),
\end{align*}
for each $p \in \{1,2,3\}$.

\begin{proof}
Let $\hat\E_p[f(x,\epsilon,h)]$ denote the empirical expectation over
the examples in dataset $\sD_p$ (recall the $\sD_p$'s are independent to
simplify the analysis).  By definition,
$$\frac1n \|\opX_p^*(\eta_p)\|_\op = \left\| \hat\E_p[\eta_p(x) x\tp{p}] \right\|_\op $$
for $p \in \{1,2,3\}$. To proceed, we will bound each $\eta_p(x)$
\iftoggle{withappendix}{}{
, defined in \refeqn{eta1}, \refeqn{eta2} and \refeqn{eta3},} and use \reflem{conc-norms} to bound $\|
\hat\E_p[\eta_p(x) x\tp{p}] \|_F$. The Frobenius norm to bounds the
operator norm, completing the proof.

%composed of several zero-mean random variables, and
%since $\|A + B\|_\op \le \|A\|_\op + \|B\|_\op$, it suffices to consider
%each term in turn. 

\paragraph{Bounding $\eta_p(x)$.}
Using the assumptions that $\|\beta_h\|_2 \le L$, $\|x\|_2 \le R$ and
$|\epsilon| \le S$, it is easy to bound each $\eta_p(x)$,
\begin{align*}
  \eta_1(x) &= \innerp{\beta_h - M_1}{x} + \epsilon \\
            &\le \|\beta_h - M_1\|_2 \|x\|_2 + |\epsilon| \\
            &\le 2LR + S \\
  \eta_2(x) 
    &= \innerp{\beta_h\tp{2} - M_2}{x\tp{2}} + 2 \epsilon \innerp{\beta_h}{x} + (\epsilon^2 - \E[\epsilon^2]) \\
    &\le \|\beta_h\tp{2} - M_2\|_F \|x\tp{2}\|_F + 2 |\epsilon| \|\beta_h\|_2\|x\|_2 + |\epsilon^2 - \E[\epsilon^2]| \\
    &\le (2L)^2 R^2 + 2 S L R + (2S)^2 \\
  \eta_3(x) &= \innerp{\beta_h\tp{3} - M_3}{x\tp{3}}
        + 3 \epsilon \innerp{\beta_h\tp{2}}{x\tp{2}} \\
        &\quad + 3\left(\epsilon^2 \innerp{\beta_h}{x} - \E[\epsilon^2] \innerp{\hat M_1}{x}\right)
        + (\epsilon^3 - \E[\epsilon^3]) \\
  &\le \|\beta_h\tp{3} - M_3\|_F\|x\tp{3}\|_F
        + 3 |\epsilon| \|\beta_h\tp{2}\|_F \|x\tp{2}\|_F  \\
        &\quad + 3 \left( |\epsilon^2|~\|\beta_h\|_F\|x\|_F + \left|\E[\epsilon^2]\right|~\|\hat M_1\|_2\|x\|_2 \right)
        + |\epsilon^3| + \left| \E[\epsilon^3] \right| \\
  &\le (2L)^3 R^3 + 3 S L^2 R^2 + 3 ( S^2 L R + S^2 L R ) + 2S^3.
\end{align*}
We have used inequality $\|M_1 - \beta_h\|_2 \le 2L$ above. 

\paragraph{Bounding $\left\| \hat\E[\eta_p(x)x\tp{p}] \right\|_F$.}
We may now apply the above bounds on $\eta_p(x)$ to bound $\|\eta_p(x) x\tp{p}\|_F$, using the fact that $\|c X\|_F \le c\|X\|_F$.
By \reflem{conc-norms}, each of the following holds with probability at least $1-\delta_1$,
\begin{align*}
    \left\|\hat\E_1[\eta_1(x) x] \right\|_2
    &\le 2 \frac{R (2LR + S)}{\sqrt{n}} \left( 1 + \sqrt{\frac{\log(1/\delta_1)}{2}} \right) \\
  \left\|\hat\E_2[\eta_2(x) x\tp{2}] \right\|_F
      &\le 2 \frac{(4L^2 R^2 + 2 S L R + 4S^2)R^2}{\sqrt{n}} \left( 1 + \sqrt{\frac{\log(1/\delta_2)}{2}} \right) \\
  \left\|\hat\E_3[\eta_3(x) x\tp{3}] - \E[\eta_3(x) x\tp{3} \mid x] \right\|_F
      &\le 2 \frac{(8L^3 R^3 + 3 L^2 R^2 S + 6 L R S^2 + 2S^3) R^3}{\sqrt{n}} \left( 1 + \sqrt{\frac{\log(1/\delta_3)}{2}} \right).
\end{align*}

Recall that $\eta_3(x)$ does not have zero mean, so we must bound the bias:
\begin{align*}
  \| \E[\eta_3(x) x\tp{3} \mid x] \|_F &= \|3 \E[\epsilon^2] \innerp{M_1 - \hat M_1}{x} x\tp{3} \|_F \\
    &\le 3 \E[\epsilon^2] \|M_1 - \hat M_1\|_2 \|x\|_2 \|x\tp{3}\|_F.
\end{align*}
Note that in all of this, both $\hat M_1$ and $M_1$ are treated as
constants. Further, by applying standard results for ordinary least-squares linear regression (e.g. \todo{Find citation}), we have a bound on $\|M_1 - \hat
M_1\|_2$; with probability at least $1-\delta_3$,
\begin{align*}
  \| M_1 - \hat M_1 |_2
  &\le \frac{32 \lambda^{(1)}_n}{\kappa(\opX_1)} \\
  &\le 32 \frac{2R(2LR+S)}{\sqrt{n}}\left(1 + \sqrt{\frac{\log(1/\delta_3)}{2}}\right) \frac{2}{\sigmamin(\Sigma_1)}.
\end{align*}

So, with probability at least $1 - \delta_3$,
\begin{align*}
  \| \E[\eta_3(x) x\tp{3} \mid x] \|_F
  &\le 3 R^4 S^2 \left(\frac{128 R(2LR+S)}{\sigmamin(\Sigma_1) \sqrt{n}} \left(1 + \sqrt{\frac{\log(1/\delta_3)}{2}} \right) \right).
\end{align*}

% {%
% \begin{align*}
% %   \| \hat\E_1[ \eta_1(x) x ] \|_\Lop &\le 
% %             \underbrace{ \|\hat\E_1[\innerp{M_1 - \beta_h}{x} x]\|_\op  }_{ O( L R^2     \sqrt{\frac{\log(1/\delta_1)}{n}} ) } +
% %             \underbrace{ \|\hat\E_1[\epsilon x]\|_\op                   }_{ O( R S  \sqrt{\frac{\log(1/\delta_1)}{n}} ) } \\
%   %  &= O( S L R^2 \sqrt{\frac{\log(1/\delta_1)}{n}} ) \\
%     \| \hat\E_2[ \eta_2(x) x\tp{2} ] \|_\Lop &\le 
%             \underbrace{ \|\hat\E_2[\innerp{\beta_h\tp{2} - M_2}{x\tp{2}} x\tp{2}]\|_\op }_{ O( L^2 R^4      \sqrt{\frac{\log(1/\delta_1)}{n}} ) } \\
%    &\quad + \underbrace{ \|\hat\E_2[2 \epsilon \innerp{\beta_h}{x} x\tp{2}]\|_\op        }_{ O( S L R^3 \sqrt{\frac{\log(1/\delta_1)}{n}} ) } 
%           + \underbrace{ \|\hat\E_2[(\epsilon^2 - S^2) x\tp{2}]\|_\op               }_{ O( S^2 R^2 \sqrt{\frac{\log(1/\delta_1)}{n}} ) } \\
%   %&= O( S^2 L^2 R^4 \sqrt{\frac{\log^2(1/\delta_1)}{n}} ) \\
%   \| \hat\E_3[ \eta_3(x) x\tp{3} ] \|_\Lop &\le 
%             \underbrace{ \|\hat\E_3[\innerp{\beta_h\tp{3} - M_3}{x\tp{3}}                      x\tp{3}]\|_\op }_{O( L^3 R^6        \sqrt{\frac{\log(1/\delta_1)}{n}} ) } \\
%    &\quad + \underbrace{ \|\hat\E_3[3 \epsilon \innerp{\beta_h\tp{2}}{x\tp{2}}                 x\tp{3}]\|_\op }_{O( S L^2 R^5 \sqrt{\frac{\log(1/\delta_1)}{n}} ) } 
%           + \underbrace{ \|\hat\E_3[\epsilon^3                                                 x\tp{3}]\|_\op }_{O( S^3 R^3   \sqrt{\frac{\log (1/\delta_1)}{n}} ) } \\
%    &\quad + \underbrace{ \|\hat\E_3[3(\epsilon^2 \innerp{\beta_h}{x} -S^2 \innerp{\hat M_1}{x}) x\tp{3}]\|_\op }_{O( (S^2 L R^4 + S^2 R^4) \sqrt{\frac{\log(1/\delta_1)}{n}} ) } \\
%   %&= O( S^3 L^3 R^6 \sqrt{\frac{\log^3(1/\delta_1)}{n}} ).
% \end{align*}
% }
% There are two additional considerations when bounding the terms in $\eta_3$.
% First, to bound $\hat\E_3[\epsilon^3]$, we use Lemma 19 of \cite{hsu13spherical}.
% Second, there is an additional bias since the estimator uses $\hat M_1$ instead of $M_1$; this bias is small by the previous bound (and is constructed on an independent dataset, $\sD_1$).
% Therefore, we incur another $O(S^2 \cdot R^4 \sqrt{\frac{\log(1/\delta_1)}{n}})$.
%We can handle the other terms using conventional means, the details of which can be found in \appendixref{sec:proofs}.
%Note that we are bounding quantities fairly crudely for the sake of simplicity.

Finally, taking $\delta_1 = \delta/3, \delta_2 = \delta/3, \delta_3
= \delta/6$, and taking the union bound over the bounds for $p \in
\{1,2,3\}$, we get our result.
\end{proof}

%%%%%%%%%%%%

\section{Proofs: Tensor Decomposition}
\label{sec:proofs:tensors}

Once we have estimated the moments from the data through regression, we apply the robust tensor eigen-decomposition algorithm to recover the parameters, $\beta_h$ and $\pi$. However, the algorithm is guaranteed to work only for symmetric matrices with (nearly) orthogonal eigenvectors, so, as a first step, we will need to whiten the third-order moment tensor using the second moments. We then apply the tensor decomposition algorithm to get the eigenvalues and eigenvectors. Finally, we will have to undo the transformation by applying an un-whitening step. In this section, we present error bounds for each step, and combine them to prove the following lemma,
\begin{lemma}[Tensor Decomposition with Whitening]
  \label{lem:tensorPower} 
  Let $M_2 = \sum_{h=1}^{k} \pi_h \beta_h\tp{2}$,
  $M_3 = \sum_{h=1}^{k} \pi_h \beta_h\tp{3}$.
  Let $\aerr{M_2} \eqdef \|\hat M_2 - M_2\|_\op$ and
  $\aerr{M_3} \eqdef \|\hat M_3 - M_3\|_\op$ both be such that,

\begin{align*}
  \max\{\aerr{M_2}, \aerr{M_3}\} &\le \min\Bigg\{
    \frac{\sigma_k(M_2)}{2},
      \left(\frac{15 k \pi_{\max}^{5/2}
      \left(24 \frac{\| {M_3} \|_\op}{\sigma_k(M_2)} + 2\sqrt{2} \right)}{2\sigma_k(M_2)^{3/2}} \right)^{-1} \epsilon,\\
     &\quad 
      \left(
      4\sqrt{3/2} \|M_2\|_\op^{1/2} \sigma_k(M_2)^{-1} + 8 k \pi_{\max} 
      \|M_2\|_\op^{1/2} \sigma_k(M_2)^{-3/2}
      \left(24 \frac{\| {M_3} \|_\op}{\sigma_k(M_2)} + 2\sqrt{2} \right) \right)^{-1} \epsilon
  \Bigg\}
\end{align*}
  
%  less than $\frac{1}{2 \sigma_k(M_2)}$
%  \begin{align*}
%    \frac{2\sigma_k(M_2)^{3/2}}{15 k \pi_{\max}^{5/2}
%    \left(24 \frac{\| {M_3} \|_\op}{\sigma_k(M_2)} + 2\sqrt{2} \right)}~ \epsilon
%  \end{align*}
%  and
%\begin{align*}
%    \left(
%    4\sqrt{3/2} \|M_2\|_\op^{1/2} \sigma_k(M_2)^{-1} + 8 k \pi_{\max} 
%    \|M_2\|_\op^{1/2} \sigma_k(M_2)^{-3/2}
%    \left(24 \frac{\| {M_3} \|_\op}{\sigma_k(M_2)} + 2\sqrt{2} \right) \right)^{-1}~ \epsilon,
%\end{align*}
for some $\epsilon < \frac{1}{2\sqrt{\pi_{\max}}}$.
%\begin{align*}
%  \epsilon &\le 
%    \min\Bigg\{
%    \left(
%    4\sqrt{3/2} \|M_2\|_\op^{1/2} \sigma_k(M_2)^{-1} + 8 k \pi_{\max} 
%    \|M_2\|_\op^{1/2} \sigma_k(M_2)^{-3/2}
%    \left(24 \frac{\| {M_3} \|_\op}{\sigma_k(M_2)} + 2\sqrt{2} \right) \right) \frac{\sigma_k(M_2)}{2}, \\
%  &\quad 
%    \frac{10 k \pi_{\max}^{5/2}
%    \left(24 \frac{\| {M_3} \|_\op}{\sigma_k(M_2)} + 2\sqrt{2} \right)}
%    {3\sigma_k(M_2)^{3/2}}~ \frac{\sigma_k(M_2)}{2}, 
%    \frac{1}{2\sqrt{\pi_{\max}}} \Bigg\}.
%\end{align*}

  Then, there exists a permutation of indices such that  the parameter
  estimates found in step 2 of 
  \iftoggle{withappendix}{%
  \algorithmref{algo:spectral-experts}
  }{%
  \citet[Algorithm 1]{ChagantyLiang2013}
  }
  satisfy the following with probability at least $1 - \delta$,
  \begin{align*}
  \|\hat \pi - \pi \|_{\infty} &\le \epsilon \\
  \|\hat \beta_h - \beta_h\|_2 &\le \epsilon.
  \end{align*}
  for all $h \in [k]$.
\end{lemma}

\begin{proof}
We will use the general notation, $\aerr{X} \eqdef \|\hat X - X\|_\Lop$
to represent the error of the estimate, $\hat X$, of $X$ in the operator
norm. 

Through the course of the proof, we will make some assumptions on errors
that allow us simplify portions of the expressions. At the end of
the proof, we will collect these conditions together to state the
assumptions on $\epsilon$ above.

\paragraph{Step 1: Whitening}
Much of this matter has been presented in \citet[Lemma 11, 12]{hsu13spherical}. We present our own version for completeness.

Let $W$ and $\hat W$ be the whitening matrices for $M_2$ and $\hat M_2$
respectively. Also define $\Winv$ and $\Whinv$ to be their
pseudo-inverses.

We will first show that the whitened tensor $T = M_3(W,W,W)$ is symmetric with orthogonal
eigenvectors. Recall that $M_2 = \sum_h \pi_h \beta_h\tp{2}$, so,
\begin{align*}
  I 
    &= W^T M_2 W\\
    &= \sum_h \pi_h W^T \beta_h\tp{2} W\\
    &= \sum_h (\underbrace{\sqrt{\pi_h} W^T \beta_h}_{v_h})\tp{2}.
\end{align*}
Thus $W \beta_h = \frac{v_h}{\sqrt{\pi_h}}$,
where $v_h$ form an orthonormal
basis. Applying the same whitening transform to $M_3$, we get, 
\begin{align*}
  M_3 &= \sum_h \pi_h \beta_h\tp{3} \\
  M_3(W,W,W) &= \sum_h \pi_h (W^T \beta_h)\tp{3} \\
  &= \sum_h \frac{1}{\sqrt{\pi_h}} v_h\tp{3}.
\end{align*}
Consequently, $T$ has an orthogonal decomposition with eigenvectors $v_h$ and eigenvalues $1/\sqrt{\pi_h}$.

Let us now study how far $\hat T = \hat M_3(\hat W, \hat W, \hat W)$ differs from $T$, in terms of the
errors of $M_2$ and $M_3$, following \citet{AnandkumarGeHsu2012}. We note that while $\hat T$ is also symmetric, it may not have an orthogonal decomposition. 
To do so, we use the triangle inequality to
break the difference into a number of simple terms, that differ in exactly one element. \todo{We will then apply $\|M_3(W,W,W-\hat W)\|_\op \le \|M_3\|_\op \|W\|_\op \|W\|_\op \|\hat W\|_\op$}.

\begin{align*}
  \aerr{T} &= \|M_3(W,W,W) - \hat M_3(\hat W,\hat W, \hat W)\|_\op \\
           &\le 
           \| {M_3}(W,W,W) - {M_3}(W,W,\hat W) \|_\op
           + \| {M_3}(W,W,\hat W) - {M_3}(W, \hat W, \hat W)\|_\op \\
           &\quad 
           + \|{M_3}(W,\hat W,\hat W) - {M_3}(\hat W, \hat W, \hat W)\|_\op 
           + \|{M_3}(\hat W,\hat W,\hat W) - \hat {M_3}(\hat W,\hat W, \hat W)\|_\op \\
           &\le 
           \| {M_3}(W,W,W - \hat W) \|_\op
           + \| {M_3}(W,W - \hat W,\hat W) |_\op 
           + \|{M_3}(W - \hat W,\hat W,\hat W) |_\op \\
           &\le
           \| {M_3} \|_\op \|W\|^2_\op \aerr{W} +
            \| {M_3} \|_\op \|\hat W\|_\op \|W\|_\op \aerr{W} +
            \| {M_3} \|_\op \|\hat W\|^2_\op \aerr{W} +
            \aerr{M_3} \|\hat W\|^3_\op  \\
           &\le
           \| {M_3} \|_\op (\|W\|^2_\op + \|\hat W\|_\op \|W\|_\op + \|\hat W\|^2_\op) \aerr{W} +
            \aerr{M_3} \|\hat W\|^3_\op 
\end{align*}
We can relate $\|\hat W\|$ and $\aerr{W}$ to $\aerr{M_2}$ using 
\reflem{white}. 
\begin{align*}
  \|\hat W\|_\op 
  &\le \frac{\sigma_k(M_2)^{-1/2}}{\sqrt{1 - \frac{\aerr{M_2}}{\sigma_k(M_2)} }} \\
  \aerr{W} 
  &\le 2 \sigma_k(M_2)^{-1/2} \frac{\frac{\aerr{M_2}}{\sigma_k(M_2)}}{1 - \frac{\aerr{M_2}}{\sigma_k(M_2)}}.
\end{align*}

To apply \reflem{white} though, we need the following condition,
\begin{condition}
  Let $\aerr{M_2} < \sigma_k(M_2)/3$.
\end{condition}

This allows us to simplify the expression,
\begin{align*}
  \|\hat W\|_\op &\le \sqrt{2} \sigma_k(M_2)^{-1/2} \\
  \aerr{W} &\le 4 \sigma_k(M_2)^{-3/2} \aerr{M_2}.
\end{align*}
Thus,
\begin{align*}
  \aerr{T} &\le 
  6 \| {M_3} \|_\op \|W\|^2_\op (4 \sigma_k(M_2)^{-3/2}) \aerr{M_2} +
  \aerr{M_3} 2\sqrt{2} \|W\|^3_\op \\
  &\le 
  24 \| {M_3} \|_\op \sigma_k(M_2)^{-5/2} \aerr{M_2} +
  2\sqrt{2} \sigma_k(M_2)^{-3/2} \aerr{M_3} \\
  &\le 
    \sigma_k(M_2)^{-3/2}
      \left(24 \frac{\| {M_3} \|_\op}{\sigma_k(M_2)} + 2\sqrt{2} \right)
      \max\{ \aerr{M_2}, \aerr{M_3} \}.
\end{align*}

\paragraph{Step 2: Decomposition}

We have constructed $T$ to be a symmetric tensor with orthogonal
eigenvectors. We can now apply the results of \citet[Theorem
5.1]{AnandkumarGeHsu2012} to bound the error in the eigenvalues,
$\lambda_W$, and eigenvectors, $\omega$, returned by the robust tensor
power method;
\newcommand{\lW}{\lambda_W}
\newcommand{\lhW}{{\hat\lambda}_W}
\newcommand{\mW}{\omega}
\newcommand{\mhW}{{\hat\omega}}
\begin{align*}
  \|\lW - \lhW \|_{\infty} 
  &\le \frac{5 k \aerr{T}}{(\lW)_{\min}} \\
\|\mW_h -\mhW_h \|_2 
&\le \frac{8 k \aerr{T}}{(\lW)_{\min}^2},
\end{align*}
for all $h \in [k]$, where $(\lW)_{\min}$ is the smallest
eigenvalue of $T$. 

\paragraph{Step 3: Unwhitening}

Finally, we need to invert the whitening transformation to recover $\pi$ and
$\beta_h$ from $\lW$ and $\mW_h$. Let us complete the proof by
studying how this inversion relates the error in $\pi$ and $\beta$ to
the error in $\lW$ and $\mW$.

First, we will bound the error in the $\beta$s,
\begin{align*}
  \|\hat \beta_h - \beta_h\|_2
  &= \| \Whinv \mhW - \Winv \mW \|_2 \\
  &\le \aerr{\Winv} \|\mhW_h\|_2 + \|\Winv\|_2 \|\mhW_h - \mW_h \|_2. \comment{Triangle inequality}
\end{align*}

Once more, we can apply the results of \reflem{white}, 
\begin{align*}
  \|\Whinv\|_\op 
    &\le \sqrt{\sigma_1(M_2)} \sqrt{1 + \frac{\aerr{M_2}}{\sigma_k(M_2)} } \\
    \aerr{\Winv} 
    &\le 2 \sqrt{\sigma_1(M_2)} \left(\sqrt{1 + \frac{\aerr{M_2}}{\sigma_1(M_2)}}\right) \frac{\frac{\aerr{M_2}}{\sigma_k(M_2)}}{1 - \frac{\aerr{M_2}}{\sigma_k(M_2)}}.
\end{align*}
Using Condition 1, this simplifies to, 
\begin{align*}
  \|\Whinv\|_\op &\le \sqrt{3/2} \|M_2\|_\op^{1/2} \\
  \aerr{\Winv} &\le 4\sqrt{3/2} \|M_2\|_\op^{1/2} \sigma_k(M_2)^{-1} \aerr{M_2}.
\end{align*}

Thus,
\begin{align*}
  \|\hat \beta_h - \beta_h\|_2
  &\le 4\sqrt{3/2} \|M_2\|_\op^{1/2} \sigma_k(M_2)^{-1} \aerr{M_2} 
    + 8 \|M_2\|_\op^{1/2} \frac{k \aerr{T}}{(\lW)_{min}^2} \\
    &\le 4\sqrt{3/2} \|M_2\|_\op^{1/2} \sigma_k(M_2)^{-1} \aerr{M_2} \\
  &\quad + 8 \|M_2\|_\op^{1/2} k \pi_{\max} 
    \sigma_k(M_2)^{-3/2}
      \left(24 \frac{\| {M_3} \|_\op}{\sigma_k(M_2)} + 2\sqrt{2} \right)
      \max\{\aerr{M_2}, \aerr{M_3}\}.
\end{align*}

Next, let us bound the error in $\pi$,
\begin{align*}
  |\hat \pi_h - \pi_h |
  &= \left| \frac{1}{(\lW)_h^2} - \frac{1}{(\lhW)_h^2} \right| \\
  &= \left| \frac{\left( (\lW)_h + (\lhW)_h \right) \left( (\lW)_h - (\lhW)_h \right)}
  {(\lW)_h^2(\lhW)_h^2} \right| \\
  &\le \frac{( 2(\lW)_h - \|\lW - \lhW\|_{\infty} )}{(\lW)_h^2 \left( (\lW)_h + \|\lW - \lhW\|_{\infty} \right)^2} \|\lW - \lhW\|_{\infty}.
\end{align*}
To simplify the above expression, we would like that $\|\lW - \lhW
\|_{\infty} \le (\lW)_{\min}/2$ or $\|\lW - \lhW \|_{\infty} \le
\frac{1}{2\sqrt{\pi_h}}$, recalling that $(\lW)_h = \pi_h^{-1/2}$. Thus,
we would like to require the following condition to hold on $\epsilon$;  
\begin{condition}
  $\epsilon \le \frac{1}{2\sqrt{\pi_{\min}}}$.
\end{condition}

Now,
\begin{align*}
  |\hat \pi_h - \pi_h |
  &\le \frac{(3/2)(\lW)_h}{(\lW)_h^4}
  \|\lW - \lhW\|_{\infty} \\
  &\le \frac{3}{2(\lW)_h^3} \frac{5 k \aerr{T}}{(\lW)_{\min}^2} \\
  &\le \frac{3 \pi_{\max}^{3/2}}{2} 5 k \pi_{\max} 
    \sigma_k(M_2)^{-3/2}
    \left(24 \frac{\| {M_3} \|_\op}{\sigma_k(M_2)} + 2\sqrt{2} \right) \max\{\aerr{M_2}, \aerr{M_3}\} \\
    &\le \frac{15}{2} \pi_{\max}^{5/2} k 
    \sigma_k(M_2)^{-3/2}
    \left(24 \frac{\| {M_3} \|_\op}{\sigma_k(M_2)} + 2\sqrt{2} \right) \max\{\aerr{M_2}, \aerr{M_3}\}.
\end{align*}

Finally, we complete the proof by requiring that the bounds $\aerr{M_2}$ and
$\aerr{M_3}$ imply that $\|\hat \pi - \pi \|_{\infty} \le \epsilon$ and
$\|\hat \beta_h - \beta_h\|_2 \le \epsilon$, i.e.
\begin{align*}
%  \frac{15}{2} \pi_{\max}^{5/2} k 
%  \sigma_k(M_2)^{-3/2}
%  \left(24 \frac{\| {M_3} \|_\op}{\sigma_k(M_2)} + 2\sqrt{2} \right) \max\{\aerr{M_2}, \aerr{M_3}\} \le \epsilon \\
  \max\{\aerr{M_2}, \aerr{M_3}\} &\le
  \left( \frac{15}{2} \pi_{\max}^{5/2} k 
  \sigma_k(M_2)^{-3/2}
  \left(24 \frac{\| {M_3} \|_\op}{\sigma_k(M_2)} + 2\sqrt{2} \right) \right)^{-1} \epsilon \\
%  \frac{%
%    2 \sigma_k(M_2)^{3/2}
%   }{%
%    15 k \pi_{\max}^{5/2} \left(24 \frac{\| {M_3} \|_\op}{\sigma_k(M_2)} + 2\sqrt{2} \right)
%   }~ \epsilon \\
% \end{align*}
% and
% \begin{align*}
%   4\sqrt{3/2} \|M_2\|_\op^{1/2} \sigma_k(M_2)^{-1} \aerr{M_2}
%     + 8 \|M_2\|_\op^{1/2} k \pi_{\max} 
%       \sigma_k(M_2)^{-3/2}
%         \left(24 \frac{\| {M_3} \|_\op}{\sigma_k(M_2)} + 2\sqrt{2} \right)
%       \max\{\aerr{M_2}, \aerr{M_3}\} \le \epsilon \\
  \max\{\aerr{M_2}, \aerr{M_3}\} &\le
  \left( 4\sqrt{3/2} \|M_2\|_\op^{1/2} \sigma_k(M_2)^{-1} \aerr{M_2}
    + 8 \|M_2\|_\op^{1/2} k \pi_{\max} 
      \sigma_k(M_2)^{-3/2}
        \left(24 \frac{\| {M_3} \|_\op}{\sigma_k(M_2)} + 2\sqrt{2} \right)
        \right)^{-1} \epsilon.
\end{align*}

\end{proof}

\section{Basic Lemmas}

In this section, we have included some standard results that we employ for completeness.

\begin{lemma}[Concentration of vector norms]
  \label{lem:conc-norms}
  Let $X, X_1, \cdots, X_n \in \Re^d$ be i.i.d.\ samples
  from some distribution with bounded support
  ($\|X\|_2 \le M$ with probability 1).
  Then with probability at least $1 - \delta$,
  \begin{align*}
    \left\| \frac{1}{n} \sum_{i=1}^{n} X_i - \E[X] \right\|_2 &
    \le \frac{2M}{\sqrt{n}} \left(1 + \sqrt{\frac{\log(1/\delta)}{2}}\right).
  \end{align*}
\end{lemma}
\begin{proof}
  Define $Z_i = X_i - \E[X]$.

%  First, $\|Z_i\|_2 \le \|X_i\|_2 + \|\E[X_i]\|_2 \le 2M$,
%  where the first inequality is due to the triangle inequality,
%  and the second follows by Jensen's inequality on $\|\cdot\|_2$
%  and the boundedness assumption on $X_i$.

The quantity we want to bound can be expressed as follows:
  \begin{align*}
  f(Z_1, Z_2, \cdots, Z_n) = \left\| \frac1n \sum_{i=1}^n Z_i \right\|_2.
  \end{align*}

Let us check that $f$ satisfies the bounded differences inequality:
  \begin{align*}
|f(Z_1, \cdots, Z_i, \cdots, Z_n) - f(Z_1, \cdots, Z_i', \cdots, Z_n)|
& \le \frac1n \|Z_i - Z_i'\|_2 \\
& = \frac1n \|X_i - X_i'\|_2 \\
&\le \frac{2M}{n},
  \end{align*}
  by the bounded assumption of $X_i$ and the triangle inequality.

By McDiarmid's inequality,
with probability at least $1 - \delta$,
we have:
\begin{align*}
\Pr[f - \E[f] \ge \epsilon] \le
\exp\left(\frac{-2 \epsilon^2}{\sum_{i=1}^n (2M/n)^2}\right).
\end{align*}
Re-arranging:
\begin{align*}
  \left\|\frac{1}{n}\sum_{i=1}^n Z_i\right\|_2
  &\le \E\left[ \left\| \frac{1}{n} \sum_{i=1}^n Z_i \right\|_2 \right]
  + M\sqrt{\frac{2\log(1/\delta)}{n}}.
\end{align*}

Now it remains to bound $\E[f]$.
By Jensen's inequality, $\E[f] \le \sqrt{\E[f^2]}$,
so it suffices to bound $\E[f^2]$:
\begin{align*}
  \E\left[ \frac1{n^2} \left\| \sum_{i=1}^n Z_i \right\|^2 \right]
  &= \E\left[ \frac1{n^2} \sum_{i=1}^n \|Z_i\|_2^2 \right] +
\E\left[ \frac1{n^2} \sum_{i\neq j} \innerp{Z_i}{Z_j} \right] \\ 
& \le \frac{4M^2}{n} + 0,
\end{align*}
where the cross terms are zero by independence of the $Z_i$'s.

Putting everything together, we obtain the desired bound:
\begin{align*}
\left\|\frac{1}{n}\sum_{i=1}^n Z_i \right\|
&\le \frac{2M}{\sqrt{n}} + M \sqrt{\frac{2\log(1/\delta)}{n}}.
\end{align*}
\end{proof}

%%%%%%%%%%%%%%%%%%%%%%%%%%%%%%%%%%%%%%%%%%%%%%%%%%%%%%%%%%%%

\textbf{Remark}: The above result can be directly
applied to the Frobenius norm of a matrix
$M$ because $\|M\|_F = \|\vvec(M)\|_2$.

\begin{lemma}[Perturbation Bounds on Whitening Matrices]
  \label{lem:white}
  Let $A$ be a rank-k $d\times d$ matrix, $\Wp$ be a $d \times k$ matrix that
  whitens $\hat A$, i.e. $\Wp^T \Ap \Wp = I$.  Suppose $\Wp^T A \Wp
  = U D U^T$, then define $W = \hat{W} U D^{-\half} U^T$. Note that $W$
  is also a $d \times k$ matrix that whitens $A$. If $\serr{A}
  = \frac{\aerr{A}}{\sigma_k(A)} < \frac{1}{3}$ 
  then, 
  \begin{align*}
    \|\hat W\|_\op 
      &\le \frac{\|W\|_\op}{\sqrt{1 - \serr{A}} } \\
  \|\Whinv\|_\op 
    &\le \|\Winv\|_\op \sqrt{1 + \serr{A}} \\
    \aerr{W} 
      &\le \|W\|_\op \frac{\serr{A}}{1 - \serr{A}} \\
    \aerr{\Winv} 
      &\le \|\Winv\|_\op \sqrt{1 + \serr{A}} \frac{\serr{A}}{1 - \serr{A}}.
  \end{align*}
\end{lemma}
\begin{proof}
  This lemma has also been proved in \citet[Lemma 10]{hsu13spherical},
  but we present it differently here for completeness.
  First, note that for a matrix $W$ that whitens $A = V \Sigma V^T$,
  $W = V \Sigma^{-\half} V^T$ and $\Winv = V \Sigma^{\half} V^T$. Thus, by rotational invariance,
  \begin{align*}
    \|W\|_\op &= \frac{1}{\sqrt{\sigma_k(A)}} \\
    \|\Winv\|_\op &= \sqrt{\sigma_1(A)} \\
  \end{align*}
  This allows us to bound the operator norms of $\hat W$ and $\Whinv$ in
  terms of $W$ and $\Winv$,
  \begin{align*}
    \|\hat W\|_\op &= \frac{1}{\sqrt{\sigma_k(\hat A)}} \\
    &\le \frac{1}{\sqrt{\sigma_k({A}) - \aerr{A}} } \comment{By Weyl's Theorem} \\
    &\le \frac{1}{1 - \serr{A}} \frac{1}{\sqrt{\sigma_k(A)}} \\
    &= \frac{\|W\|_\op}{\sqrt{1 - \serr{A}} } \\
    \|\Whinv\|_\op &= \sqrt{\sigma_1(\hat A)} \\
    &\le \sqrt{\sigma_1({A}) + \aerr{A}}  \comment{By Weyl's Theorem} \\
    &\le \sqrt{1 + \serr{A}} \sqrt{\sigma_1(A)} \\
    &= \sqrt{1 + \serr{A}} \|\Winv\|_\op.
  \end{align*}

  To find $\aerr{W}$, we will exploit the rotational invariance of the operator norm. 
  \begin{align*}
    \aerr{W} &= \| \Wp - W \|_\op \\
    &= \| W U D^{\half} U^T - W \|_\op  \comment{$\hat W = W U D^{-\half} U^T$}\\
    &\le \|W\|_\op \| I - U D^{\half} U^T \|_\op \comment{Sub-multiplicativity}.
  \end{align*}

  We will now bound $\| I - U D^{\half} U^T \|_\op$. Note that by
  rotational invariance, $\| I - U D^{\half} U^T \|_\op = \|
  I - D^{\half} \|_\op$. By Weyl's inequality, $|1 - \sqrt{D_{ii}}| \le
  \| I - D^{\half} \|_\op$; put differently, $\| I - D^{\half} \|_\op$
  bounds the amount $\sqrt{D_{ii}}$ can diverge from $1$. Using the
  property that $|(1+x)^{-1/2} - 1| \le |x|$ for all $|x| \le 1/2$, we
  will take an alternate approach and bound $D_{ii}$ separately, and use
  it to bound $\| I - D^{\half} \|_\op$.
  \begin{align*}
    \| I - D \|_\op 
    &= \| I - U D U^T \|_\op \comment{Rotational invariance} \\
    &= \| \hat W^T \hat A \hat W - \hat W^T A \hat W \|_\op \comment{By definition} \\
    &= \| \hat W^T (\hat A - A) \hat W \|_\op \\
    &\le \|\hat W\|_\op^2 \aerr{A} \\
    &\le \frac{1}{\sigma_k(A)} \frac{\aerr{A}}{1 - \serr{A}} \\
    &\le \frac{\serr{A}}{1 - \serr{A}}.
  \end{align*}

  Therefore, if $\frac{\serr{A}}{1 - \aerr{A}} < 1/2$, or $\aerr{A} < \frac{\sigma_k(A)}{3}$,
  \begin{align*}
    \| I - U D^{1/2} U^T \|_\op &\le \frac{\serr{A}}{1 - \serr{A}}.
  \end{align*}

  We can now complete the proof of the bound on $\aerr{W}$, 
  \begin{align*}
    \aerr{W} 
    &= \|W\|_\op \| I - U D^{1/2} U^T \|_\op \comment{Rotational invariance}\\
    &\le \|W\|_\op \frac{\serr{A}}{1 - \serr{A}}.
  \end{align*}

  Similarly, we can bound the error on the un-whitening transform, $\Winv$,
  \begin{align*}
    \aerr{\pinv{W}} &= \| \pinv{\Wp} - \pinv{W} \|_\op \\
    &= \| \pinv{\Wp} - U D^{\half} U^T \pinv{\Wp} \|_\op \comment{$\pinv{W} = U D^{1/2} U^T \pinv{\Wp}$} \\
    &\le \|\pinv{\Wp}\|_\op \| I - U D^{\half} U^T \|_\op \\
    &\le \|\pinv{W}\|_\op \sqrt{1+\serr{A}} \frac{\serr{A}}{1 - \serr{A}}.
  \end{align*}
\end{proof}

 
