\section{Introduction} \label{sec:introduction}

% Goal: parameter estimation in latent-variable models
Latent-variable log-linear models provide two complementary strengths: (i) they allow
latent representations to be automatically learned from data, and (ii)
they allow domain knowledge to be encoded in the form of features to influence learning when appropriate.
Despite their success across many fields \cite{quattoni04crf,haghighi06prototype,liang06discrimative,kirkpatrick10painless,deselaers12latent},
learning these models remains a difficult problem due to the non-convexity of the likelihood.
Local optimization is the standard approach but is susceptible to local optima.

% Method of moments
Recently, unsupervised learning techniques based on the method of moments and
spectral decomposition have offered a refreshing and promising perspective on
this learning problem \citep{hsu09spectral,anandkumar11tree,anandkumar12moments,anandkumar12lda,hsu12identifiability,balle11transducer,balle12automata}.
These methods exploit the linear algebraic properties of the model and
factorize the moments into parameters, providing strong theoretical guarantees.
However, these methods are not as universally applicable as EM, and have yet to
be developed for log-linear models.

\Fig{figures/schema}{0.3}{schema}{Schema of our approach.}

% This paper
In this work, we develop a technique for parameter estimation in log-linear
models, shown in \reffig{schema}.  The key idea in
\citet{anandkumar12moments,anandkumar13tensor} is to study the conditional
independence factorization structure implied by the model and identify
bottlenecks.  We construct third-order tensors around these bottlenecks, which
are factorized in our case into moments involving the latent variables.

This paper contributes three ideas:
First, in certain cases, these latent moments are exactly the sufficient statistics of
the log-linear model, in which case we use convex optimization directly to obtain the parameters,
resulting in a consistent parameter estimation procedure (\refsec{threeViewMixtureModel}).
Second, in other cases, the latent moments only provide partial information about the parameters,
but we show that this information can be naturally incorporated
using the measurements framework of \citet{liang09measurements} (\refsec{generalModels}).
This provides a natural way to leverage the method of moments when constructing
a consistent estimator is difficult.
Finally, for factorial models, we introduce a new \emph{unshuffling
factorization} procedure to recover parameter estimates (\refsec{factorialModels}).
