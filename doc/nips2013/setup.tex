\section{Setup} \label{sec:setup}

Let $z$ be a collection of variables indexed by $V$ (that is, $z = \{z_j\}_{j \in V}$).
Suppose that some of the variables are observed ($x = \{ z_j \}_{j \in O}$, where $O \subset V$) 
and the rest are latent ($h = \{ z_j \}_{j \in H}$, where $H = V \backslash O$).
Let $C \subset 2^V$ be a collection of subset of variables,
and let $\phi_c(z_c) \in \R^d$ be the feature map for each $c \in C$.
In this paper, we will consider both directed and undirected models over $z$.
In the undirected case, we define the standard globally-normalized exponential family model:
\begin{align}
  p_\theta(z) = \exp\{ \phi(z)^\top\theta - A(\theta) \}, \aside{undirected}
\end{align}
where $\phi(z) \eqdef \sum_{c \in C} \phi_c(z_c) \in \R^d$ is the global feature vector, $\theta \in \R^d$ is the parameter vector,
and $A(\theta) = \log \int \exp\{\phi(z)^\top\theta\} dz$ is the global log-partition function.
To some extent, our methods only depend on the conditional independence
structure of the models.
We focus on undirected models for notational simplicity;
directed models are discussed in \refapp{directedModels}.

\paragraph{Problem statement}

% Statement
This paper focuses on the problem of parameter estimation:
We are given $n$ i.i.d.~examples of the observed variables $D = (x^{(1)}, \dots, x^{(n)})$
where each $x^{(i)} \sim p_{\theta^*}$ for some true parameters $\theta^*$.
Our goal is to produce a parameter estimate $\hat\theta$ that approximates $\theta^*$.

% Maximum likelihood
The standard estimation procedure is maximum (marginal) likelihood,
$\hatthetaml \eqdef \max_{\theta \in \R^d} \sum_{x \in D} \log p_\theta(x)$,
which is statistically efficient but computationally intractable.
In practice, one uses EM or gradient-based optimization procedure on the marginal likelihood,
which can get stuck in local optima.
