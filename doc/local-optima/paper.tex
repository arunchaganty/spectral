\documentclass[tablecaption=bottom]{jmlr}

\jmlrproceedings{}{}
\jmlrpages{}{}
\usepackage[cm]{fullpage}
\usepackage{booktabs}
\usepackage{ctable}

\title{Counting local-optima in a log-linear model}

\author{Arun Tejasvi Chaganty \Email{chaganty@cs.stanford.edu}}

% Text
\newcommand{\todo}[1]{\hl{\textbf{TODO:} #1}}
\newcommand{\citationneeded} {\ensuremath{^{[\textrm{citation needed}]}}}


%Math Operators
%\DeclareMathOperator {\argmax} {argmax}
%\DeclareMathOperator {\argmin} {argmin}
\DeclareMathOperator {\sgn} {sgn}
\DeclareMathOperator {\trace} {tr}
\DeclareMathOperator{\E} {\mathbb{E}}
\DeclareMathOperator{\Var} {Var}
\DeclareMathOperator{\diag} {diag}
\DeclareMathOperator{\triu} {triu}
\DeclareMathOperator{\mult} {Multinomial}
\DeclareMathOperator{\normalt} {Normal}
\DeclareMathOperator{\cvec} {cvec}

\newcommand{\ud}{\, \mathrm{d}}
\newcommand{\diff}[1] {\frac{\partial}{\, \partial #1}}
\newcommand{\difff}[2] {\frac{\partial^2}{\, \partial #1\, \partial #2}}
\newcommand{\diffn}[2] {\frac{\partial^{#2}}{\, \partial {#1}^{#2}}}
\newcommand{\tuple}[1] {\langle #1 \rangle}
\newcommand{\innerprod}[2] {\langle #1, #2 \rangle}

% Constants/etc.
\renewcommand{\Re} {\mathbb{R}}
\newcommand{\Cm} {\mathbb{C}}
\newcommand{\Qm} {\mathbb{Q}}
\newcommand{\half} {\frac{1}{2}}

\newcommand{\inv}[1] {{#1}^{-1}}

\newcommand{\normal}[2] {\mathcal{N}(#1, #2)}
\newcommand{\mL} {\mathcal{L}}

\newcommand\eqdef{\ensuremath{\stackrel{\rm def}{=}}} % Equal by definition
\newcommand\refeqn[1]{(\ref{eqn:#1})}
\newcommand\sD{\ensuremath{\mathcal{D}}}
\newcommand\sM{\ensuremath{\mathcal{M}}}
\newcommand\refapp[1]{Appendix~\ref{sec:#1}}
\newcommand\refthm[1]{Theorem~\ref{thm:#1}}
\newcommand\sigmamin{\sigma_\text{\rm min}}
\newcommand\sigmamax{\sigma_\text{\rm max}}
\newcommand\op{{\text{\rm op}}}
\newcommand\BP{\ensuremath{\mathbb{P}}}
\newcommand\reflem[1]{Lemma~\ref{lem:#1}}


\begin{document}

\maketitle

\section{Introduction}

At the high-level, we're looking to count the number of the local optima
of a log-linear model,
\begin{align}
p_{\theta}(x) &= \sum_z \exp(\theta^T \phi(x,z) - A(\theta)).
\end{align}

\todo{Describe mixtures of Gaussians}

\section{Preliminaries}

\section{Mixture of Gaussians}

\section{Log-linear models}

In EM, we maximize a lower bound on the log-likelihood of model which is
derived as follows,
\begin{align}
  \sL(\theta ; \sD) 
    &\defeq \sum_{x_n \in \sD} \log p_{\theta}(x_n) \\
    &= \sum_{x_n \in \sD} \log \sum_z \exp(\theta^T \phi(x,z) - A(\theta)) \\
    &= \sum_{x_n \in \sD} \log \sum_z \frac{q(z|x_n)}{q(z|x_n)} \exp(\theta^T \phi(x,z) - A(\theta)) \\
    &= \sum_{x_n \in \sD} \sum_z q(z|x_n) \log \frac{1}{q(z|x_n)} \exp(\theta^T \phi(x,z) - A(\theta)) \\
    &\le \sum_{x_n \in \sD} \sum_z q(z|x_n) [\theta^T \phi(x,z) - A(\theta) - \log q(z|x_n)].
\end{align}

We use the notation $\sL(\theta, q) \defeq \sum_{x_n \in \sD} \sum_z
q(z|x_n) [\theta^T \phi(x,z) - A(\theta) - \log q(z|x_n)]$. The solution
to EM is $\max_\theta \max_q \sL(\theta,q)$ which we do via alternating
maximization between $\theta$ and $q$.

Furthermore, it can be shown that local optima of $\sL(\theta, q)$ are
also local optima of the original function, $\sL(\theta ; \sD)$.

Finally, we derive closed form solutions to the E and M steps,
\begin{align}
  q^*(\theta) 
    &\defeq \max_{q \in \Delta_{n-1}} \sL(\theta,q)\\
  \diff{\sL(\theta,q)}{q(z|x_n)} 
    &= \theta^T \phi(x_n,z) - A(\theta) - \log q(z|x_n) - 1 + \lambda \\
  q^*(z|x_n) &\propto \exp(\theta^T \phi(x_n,z) - A(\theta)) \\
  q^*(z|x_n) &= \frac{exp(\theta^T \phi(x_n,z) - A(\theta))}{\sum_{z'} exp(\theta^T \phi(x_n,z') - A(\theta))}.
\end{align}

\begin{align}
  \theta^*(q)
    &\defeq \max_{\theta \in \Re^{n}} \sL(\theta,q)\\
  \diff{\sL(\theta,q)}{\theta} 
  &= \sum_{x_n \sD}\sum_{z} q(z|x_n) \phi(x_n,z) - \diff{A(\theta)}{\theta} \\
  &= \sum_{x_n \sD}\sum_{z} q(z|x_n) \phi(x_n,z) - \E_{\theta}(\phi(x,z)) \\
  &= \hat\E_{q}[\phi(x_n,z)] - \E_{\theta}(\phi(x,z)) \\
  \E_{\theta^*}(\phi(x,z)) &= \hat\E_{q}[\phi(x_n,z)].
\end{align}

\begin{definition}(Trapping Region)
  A closed subset of $\Re^n$ is said to be trapping for $\theta \in
  \Re^n$ if for every $\theta' \in \del R$, $\sL(\theta',
  q(\theta')) \le \sL( \theta, q(\theta) )$.
\end{definition}

\begin{lemma}(Trapping property)
  For any $\theta$ and a trapping region $R$ containing $\theta$, expectation
  maximization initialized at $\theta$ converges to a fixed point in $R$.
\end{lemma}
\begin{proof}
  Let $\theta', q(\theta')$ be the new set of parameters after one
  iteration of EM. The proof proceeds by constructing a path $\sP$ from
  $\theta$ to $\theta'$ such that every point along the path
  $\theta^\alpha \in \sP$ has a higher likelihood than $\theta$ and thus
  could not cross the boundary $\del R$. 

  Let us define a trajectory $\sP : \Re \to \Re^n$ from $\theta$ to
  $\theta'$ as follows,
  \begin{align}
    \sP(\alpha) &\defeq \alpha \theta + (1-\alpha) \theta'.
  \end{align}
  For notational convenience, we use $\theta_{\alpha} \defeq \sP(\alpha)$.

  Now, note that with $q$ fixed, $\sL(\theta,q)$ is {\em convex} in $\theta$. Thus, 
  $\sL(\theta^\alpha, q) = (1-\alpha)\sL(\theta, q) + \alpha
  \sL(\theta', q) > \sL(\theta, q)$. Furthermore, $\sL(\theta^\alpha,
  q^*(\theta^\alpha)) \ge \sL(\theta^\alpha, q) > \sL(\theta, q)$.

  \todo{Refine with a path for $q$ too}
\end{proof}

Next, we will define a local region around some parameter $\theta$ that is trapping.


\end{document}
