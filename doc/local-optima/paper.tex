\documentclass[tablecaption=bottom]{jmlr}

\jmlrproceedings{}{}
\jmlrpages{}{}
\usepackage[cm]{fullpage}
\usepackage{booktabs}
\usepackage{ctable}

\title{Counting local-optima in a log-linear model}

\author{Arun Tejasvi Chaganty \Email{chaganty@cs.stanford.edu}}

% Text
\newcommand{\todo}[1]{\hl{\textbf{TODO:} #1}}
\newcommand{\citationneeded} {\ensuremath{^{[\textrm{citation needed}]}}}


%Math Operators
%\DeclareMathOperator {\argmax} {argmax}
%\DeclareMathOperator {\argmin} {argmin}
\DeclareMathOperator {\sgn} {sgn}
\DeclareMathOperator {\trace} {tr}
\DeclareMathOperator{\E} {\mathbb{E}}
\DeclareMathOperator{\Var} {Var}
\DeclareMathOperator{\diag} {diag}
\DeclareMathOperator{\triu} {triu}
\DeclareMathOperator{\mult} {Multinomial}
\DeclareMathOperator{\normalt} {Normal}
\DeclareMathOperator{\cvec} {cvec}

\newcommand{\ud}{\, \mathrm{d}}
\newcommand{\diff}[1] {\frac{\partial}{\, \partial #1}}
\newcommand{\difff}[2] {\frac{\partial^2}{\, \partial #1\, \partial #2}}
\newcommand{\diffn}[2] {\frac{\partial^{#2}}{\, \partial {#1}^{#2}}}
\newcommand{\tuple}[1] {\langle #1 \rangle}
\newcommand{\innerprod}[2] {\langle #1, #2 \rangle}

% Constants/etc.
\renewcommand{\Re} {\mathbb{R}}
\newcommand{\Cm} {\mathbb{C}}
\newcommand{\Qm} {\mathbb{Q}}
\newcommand{\half} {\frac{1}{2}}

\newcommand{\inv}[1] {{#1}^{-1}}

\newcommand{\normal}[2] {\mathcal{N}(#1, #2)}
\newcommand{\mL} {\mathcal{L}}

\newcommand\eqdef{\ensuremath{\stackrel{\rm def}{=}}} % Equal by definition
\newcommand\refeqn[1]{(\ref{eqn:#1})}
\newcommand\sD{\ensuremath{\mathcal{D}}}
\newcommand\sM{\ensuremath{\mathcal{M}}}
\newcommand\refapp[1]{Appendix~\ref{sec:#1}}
\newcommand\refthm[1]{Theorem~\ref{thm:#1}}
\newcommand\sigmamin{\sigma_\text{\rm min}}
\newcommand\sigmamax{\sigma_\text{\rm max}}
\newcommand\op{{\text{\rm op}}}
\newcommand\BP{\ensuremath{\mathbb{P}}}
\newcommand\reflem[1]{Lemma~\ref{lem:#1}}


\begin{document}

\maketitle

\section{Introduction}

At the high-level, we're looking to count the number of the local optima
of a log-linear model,
\begin{align}
p_{\theta}(x) &= \sum_z \exp(\theta^T \phi(x,z) - A(\theta)).
\end{align}

This procedure does not exactly cover the mixture of Gaussians though.

\section{Preliminaries}

\section{Mixture of Gaussians}

\section{Log-linear models}

\todo{Write down log-likelihood}

\begin{definition}(Trapping Region)
  $\sT(\theta)$ is a closed subset of $\Re^n$ such that for every
  $\theta' \in \partial \sT(\theta)$, $\sL(\theta', q(\theta')) \le \sL(
  \theta, q(\theta) )$.
\end{definition}

\begin{lemma}(Trapping property of $\sT(\theta)$)
  For any $\theta$, EM converges to a fixed point in $\sT(\theta)$.
\end{lemma}
\begin{proof}
  Let $\theta', q(\theta')$ be the new set of parameters after one
  iteration of EM. We will prove by contradiction that $\theta'$ must
  lie within $\sT(\theta)$. 
  
  Consider a trajectory $\sP$ from $\theta$ to $\theta'$, defined as follows,
  \begin{align}
    \theta^\alpha = \alpha \theta + (1-\alpha) \theta'.
  \end{align}


\end{proof}

\end{document}
