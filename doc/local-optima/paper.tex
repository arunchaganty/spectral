\documentclass[tablecaption=bottom]{jmlr}

\jmlrproceedings{}{}
\jmlrpages{}{}
\usepackage[cm]{fullpage}
\usepackage{booktabs}
\usepackage{ctable}

\title{Counting local-optima in a log-linear model}

\author{Arun Tejasvi Chaganty \Email{chaganty@cs.stanford.edu}}

\providecommand{\Pa}{\textrm{Pa}}
\providecommand{\TensorFactorize}{\textsc{TensorFactorize}~}
\providecommand{\LearnFactors}{\textsc{LearnFactors}~}


\begin{document}

\maketitle

\section{Introduction}

At the high-level, we're looking to count the number of the local optima
of a log-linear model,
\begin{align}
p_{\theta}(x) &= \sum_z \exp(\theta^T \phi(x,z) - A(\theta)).
\end{align}

This procedure does not exactly cover the mixture of Gaussians though.

\section{Preliminaries}

\section{Mixture of Gaussians}

\section{Log-linear models}

\todo{Write down log-likelihood}

\begin{definition}(Trapping Region)
  $\sT(\theta)$ is a closed subset of $\Re^n$ such that for every
  $\theta' \in \partial \sT(\theta)$, $\sL(\theta', q(\theta')) \le \sL(
  \theta, q(\theta) )$.
\end{definition}

\begin{lemma}(Trapping property of $\sT(\theta)$)
  For any $\theta$, EM converges to a fixed point in $\sT(\theta)$.
\end{lemma}
\begin{proof}
  Let $\theta', q(\theta')$ be the new set of parameters after one
  iteration of EM. We will prove by contradiction that $\theta'$ must
  lie within $\sT(\theta)$. 
  
  Consider a trajectory $\sP$ from $\theta$ to $\theta'$, defined as follows,
  \begin{align}
    \theta^\alpha = \alpha \theta + (1-\alpha) \theta'.
  \end{align}


\end{proof}

\end{document}
