\documentclass[xcolor={svgnames}]{beamer}

\setbeameroption{hide notes} 

%\usetheme{NLP}
\usepackage{beamerthemesplit}
\usetheme{default}
\useoutertheme{infolines}

\usepackage{graphicx}
\usepackage{lmodern}

\usepackage{soul}

\usepackage{amsmath,amsthm,amssymb}   

\usepackage{listings}

\usepackage{algorithm,algorithmic}

\usepackage{tikz}
\usetikzlibrary{positioning,shapes,shadows,arrows}

% Text
\newcommand{\todo}[1]{\hl{\textbf{TODO:} #1}}
\newcommand{\citationneeded} {\ensuremath{^{[\textrm{citation needed}]}}}


%Math Operators
%\DeclareMathOperator {\argmax} {argmax}
%\DeclareMathOperator {\argmin} {argmin}
\DeclareMathOperator {\sgn} {sgn}
\DeclareMathOperator {\trace} {tr}
\DeclareMathOperator{\E} {\mathbb{E}}
\DeclareMathOperator{\Var} {Var}
\DeclareMathOperator{\diag} {diag}
\DeclareMathOperator{\triu} {triu}
\DeclareMathOperator{\mult} {Multinomial}
\DeclareMathOperator{\normalt} {Normal}
\DeclareMathOperator{\cvec} {cvec}

\newcommand{\ud}{\, \mathrm{d}}
\newcommand{\diff}[1] {\frac{\partial}{\, \partial #1}}
\newcommand{\difff}[2] {\frac{\partial^2}{\, \partial #1\, \partial #2}}
\newcommand{\diffn}[2] {\frac{\partial^{#2}}{\, \partial {#1}^{#2}}}
\newcommand{\tuple}[1] {\langle #1 \rangle}
\newcommand{\innerprod}[2] {\langle #1, #2 \rangle}

% Constants/etc.
\renewcommand{\Re} {\mathbb{R}}
\newcommand{\Cm} {\mathbb{C}}
\newcommand{\Qm} {\mathbb{Q}}
\newcommand{\half} {\frac{1}{2}}

\newcommand{\inv}[1] {{#1}^{-1}}

\newcommand{\normal}[2] {\mathcal{N}(#1, #2)}
\newcommand{\mL} {\mathcal{L}}

\newcommand\eqdef{\ensuremath{\stackrel{\rm def}{=}}} % Equal by definition
\newcommand\refeqn[1]{(\ref{eqn:#1})}
\newcommand\sD{\ensuremath{\mathcal{D}}}
\newcommand\sM{\ensuremath{\mathcal{M}}}
\newcommand\refapp[1]{Appendix~\ref{sec:#1}}
\newcommand\refthm[1]{Theorem~\ref{thm:#1}}
\newcommand\sigmamin{\sigma_\text{\rm min}}
\newcommand\sigmamax{\sigma_\text{\rm max}}
\newcommand\op{{\text{\rm op}}}
\newcommand\BP{\ensuremath{\mathbb{P}}}
\newcommand\reflem[1]{Lemma~\ref{lem:#1}}

\newcommand{\sidenote}[1]{\begin{itemize} \item #1 \end{itemize}}
\newcommand{\hlmath}[1]{\textrm{\color{blue}\ensuremath{#1}}}

\pgfdeclarelayer{bg}
\pgfsetlayers{bg,main}
\newcommand{\obj}[1]{%
    {%
    \begin{minipage}{6cm}
      #1
    \end{minipage}
    }
}
\newcommand{\objw}[2]{%
    {%
    \begin{minipage}{#1}
      #2
    \end{minipage}
    }
}
\tikzstyle{box}=[scale=0.8,rectangle,fill=white,draw=black]
\tikzstyle{loop}=[smooth,dashed,in=-90,out=-90,looseness=0.75]

\makeatletter
\newcommand*{\centerfloat}{%
  \parindent \z@
  \leftskip \z@ \@plus 1fil \@minus \textwidth
  \rightskip\leftskip
  \parfillskip \z@skip}
\makeatother

% Tensor powers
\newcommand{\tp}[1] {^{\otimes #1}}

% Matrix Perturbation
\newcommand{\pinv}[1] {#1^{\dagger}}
\newcommand{\Ap} {\hat{A}}
\newcommand{\Bp} {\hat{B}}
\newcommand{\Up} {\hat{U}}
\newcommand{\Vp} {\hat{V}}
\newcommand{\Xp} {\hat{X}}
\newcommand{\Wp} {\hat{W}}
\newcommand{\cM} {\mathcal{M}}
\newcommand{\cMp} {\hat{\mathcal{M}}}
\newcommand{\Mp} {\hat{M}}
\newcommand{\Zp} {\hat{Z}}
\newcommand{\vp} {\hat{v}}
\newcommand{\lambdap} {\hat{\lambda}}
\newcommand{\sigmap} {\hat{\sigma}}
\newcommand{\mup} {\hat{\mu}}
\newcommand{\cnd}[1] {\kappa(#1)}
\newcommand{\aerr}[1] {\varepsilon_{#1}}
\newcommand{\rerr}[1] {\delta_{#1}}
\newcommand{\serr}[1] {\alpha_{#1}}
\newcommand{\berr}[1] {\beta_{#1}}
\newcommand{\gap}[1] {\Delta_{#1}}

% Keywords
\newcommand{\Pairs}{\mathrm{Pairs}}
\newcommand{\Triples}{\mathrm{Triples}}



% these will be used later in the title page
\title[Spectral Experts]{Spectral Experts for Estimating Mixtures of Linear Regressions}
\author[Chaganty, Liang]{%
    Arun Tejasvi Chaganty\\
    Percy Liang
}
\institute{Stanford University}

\begin{document}

% "Beamer, do the following at the start of every section"
\AtBeginSection[] 
{%
\begin{frame}<beamer> 
\frametitle{Outline} % make a frame titled "Outline"
\tableofcontents[currentsection]  % show TOC and highlight current section
\end{frame}
}

\begin{frame}
  \titlepage
\end{frame}

\begin{frame}
  \frametitle{Discriminative Latent Variable Models}

  \begin{itemize}
    \item {\bf Discriminative Latent Variable Models are useful, but hard to learn.}
    \item Discriminative LVMs combine the predictive accuracy with compact expressiveness.
    \item They have been used in several applications, e.g.\ object recognition, syntactic parsing, machine translation, etc.
    \item Learning parameters for these models is hard because they have non-convex likelihood functions.
  \end{itemize}
\end{frame}

\begin{frame}
  \frametitle{Goal: Consistent Parameter Estimation.}

  \begin{itemize}
    \item {\bf We would like to develop efficient consistent estimators for discriminative LVMS.}
    \item Past approaches rely on local optimization, which are susceptible to local optima.
    \item And they don't just go away with more data.
    \item Recently, there has been work on consistent parameter
      estimation for generative LVMS\@. We'd like to extend their work to
      the discriminative case.
  \end{itemize}
\end{frame}

\begin{frame}
  \frametitle{Mixture of Linear Regressions}

  \begin{itemize}
    \item {\bf First step, consider a simple discriminative model, mixture of linear regressions.}
    \item The mixture of linear regressions model defines a conditional distribution over a response $y \in \Re$ given covariates $x \in \Re^d$.
     \item The generative procedure is as follows,
       \begin{enumerate}
        \item Draw a mixture component $h \in [k] \sim Mult(\pi)$, where $\pi = [\pi_1 | \dots | \pi_k]$ defines the mixture proportions.
        \item Draw the noise $\epsilon \sim \mathcal{E}$, where $\mathcal{E}$ is the noise distribution.
        \item Set $y = \beta_h^T x + \epsilon$, where $\{\beta_h\}_{h=1}^{k}$ are the conditional means of the regression coefficients.
       \end{enumerate}
    \item The parameters that we would like to learn from this model are $\pi$ and $B = [ \beta_1 | \cdots | \beta_k ]$.
  \end{itemize}
\end{frame}

\begin{frame}
  \frametitle{Method of Moments.}

  \begin{itemize}
    \item {\bf We approach the problem by using the method of moments on three conditional views. }
    \item Consider a moment map $\mathcal{M}$ that maps the parameters $\theta$ to the moments $m$. For a Gaussian, we have that $\mathcal{M} = (\mu, \sigma^2)$.
    \item In general, we will compute the inverse of the moment map to learn the parameters from the sample estimates.
    \item By the central limit theorem, our sample estimates of the moments converge at a $1/\sqrt{n}$ rate, so we expect that our parameters will also converge at this rate.
  \end{itemize}
\end{frame}

\begin{frame}
  \frametitle{Method of Moments for Generative LVMs.}

  \begin{itemize}
    \item {\bf We can recover the means of a generative LVM by exploiting their symmetric structure. }
    \item 
    \begin{align*}
    M_2 &= \sum_{h=1}^k \pi_h \beta_h\tp{2} & M_3 &= \sum_{h=1}^k \pi_h \beta_h\tp{3}.
  \end{align*}
    \item In general, $M_2$ is insufficient to identify the model, because it is invariant to rotations of $B$. 
    \item If the $\beta_h$ were orthogonal, then the eigenvectors and eigenvalues of $M_3$ would give us the answer. 
    \item In general, we need both $M_2$ and $M_3$. We can use the whitening transformation for $M_2$ to whiten $M_3$ such that it has a orthogonal decomposition.
\item The robust tensor power method by \cite{AnandkumarHsuGe2012} find stable eigenvectors.
  \item In the discriminative case, we do not observe $M_2$ directly from the data, but rather, as we'll see, linear projections of them.
  \end{itemize}
\end{frame}

\begin{frame}
  \frametitle{Recovering the Moments.}
  \begin{itemize}
    \item {\bf The powers of the response variables $y^p$ are linear measurements of the moments of the parameters $M_p$!}
    \item As noted earlier, the first problem we run into is that we can't observe the moments of the parameters $B$ and $\pi$ directly!
\item However, observe that 
\begin{align}
  y &= \innerp{\beta_h}{x} + \epsilon \\
    &= \innerp{M_1}{x} + \underbrace{{\beta_h - M_1}{x} + \epsilon }_{\textrm{noise}},
\end{align}
  where $M_1 = \sum_{h=1}^k \pi_h \beta_h$, the mean regression coefficient. 
  \item We note that while the noise term is dependent on $x$, it has a zero-mean. Thus, we could potentially recover $M_1$ through regression.
\item Similarly, we can cast the remaining two terms as regression problems.

\item An additional fact that we can exploit is that both $M_2$ and $M_3$ are low rank, so we can use low-rank regression to recover estimates $\hat M_2$ and $\hat M_3$ efficiently from data.
  \end{itemize}
\end{frame}

\begin{frame}
  \frametitle{Rates for Recovery}
  \begin{itemize}
    \item {\bf The algorithm is consistent, and requires $O(?)$ samples.}
\item The rate of convergence for the spectral experts algorithm to the
true parameters breaks into two parts; the rates for learning the
moments, which feeds into the rates for learning the parameters.
\item For low rank regression, we have the following bound on recovery by (Tomioka2011);
  $$ \| \hat M_p - M_p \|_F \le \frac{32 \lambda^{(p)}_n \sqrt{k}}{\kappa(\opX_p)}, $$
where $\kappa(\opX_p)$ is the (restricted) strong convexity constant, and $\lambda^{(p)} > \|\opX^*_p(\eta)\|$. 
\item Because we assume our noise is bounded, it is easy to show that the error concentrates. 
\item In the tensor recovery case, we will need to whiten $M_3$ before
applying the tensor decomposition and unwhiten it afterwards; this
modifies the error bounds slightly.
  \end{itemize}
\end{frame}

\begin{frame}
  \frametitle{Experimental Results}
  \begin{itemize}
    \item {\bf Spectral Experts provides a good initialization to EM.}
      \item We simulated the performance of spectral experts and compared it to EM. 
\item In a non-trivial number of cases, EM did not converge to the right parameters and got stuck in local optima.
\item The parameters estimated by the spectral method were not very good even with $O(10^5)$ samples.
\item However, EM initialized with these parameters did extremely well. 
\item This finding that spectral methods should be a good initialization for EM is not surprising.
  \begin{itemize}
    \item The biggest sell for spectral methods is that they give a global guarantee on where the parameters are.
    \item The parameters might not be at the global optima, but will hopefully lie in the potential well around it.
    \item EM will then converge to the global optima.
  \end{itemize}
  \end{itemize}
\end{frame}

\begin{frame}
  \frametitle{Conclusions}
  \begin{itemize}
    \item {\bf We can learn the parameters of conditional models where the observed moments on their own have such sparse information.}
    \item The key intuition is that we could construct a regression problem to learn the moments of the parameters from their projections onto the data.
    \item We found that while the parameters learned this way are usually not better than those learned via local methods like EM, etc., they are a good way to initialize these local methods.
  \end{itemize}
\end{frame}

\end{document}

% TikZ
%\begin{tikzpicture}
%  \node (img1) at (0,0) {\includegraphics[height=0.6\textheight,width=0.4\linewidth,keepaspectratio]{}};
%\end{tikzpicture}

% Notes 
% \note[item]{}

% 2-column
% \begin{columns}
%   \begin{column}{0.48\textwidth}
%     \begin{itemize}
%         \item
%     \end{itemize}
%   \end{column}
%   \hfill
%   \begin{column}{0.48\textwidth}
%     \begin{itemize}
%         \item
%     \end{itemize}
%   \end{column}
% \end{columns}


\end{document}

% TikZ
%\begin{tikzpicture}
%  \node (img1) at (0,0) {\includegraphics[height=0.6\textheight,width=0.4\linewidth,keepaspectratio]{}};
%\end{tikzpicture}

% Notes 
% \note[item]{}

% 2-column
% \begin{columns}
%   \begin{column}{0.48\textwidth}
%     \begin{itemize}
%         \item
%     \end{itemize}
%   \end{column}
%   \hfill
%   \begin{column}{0.48\textwidth}
%     \begin{itemize}
%         \item
%     \end{itemize}
%   \end{column}
% \end{columns}

