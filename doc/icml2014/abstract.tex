\begin{abstract} 

Recent work on the method of moments provide consistent estimates
  for latent-variable models, avoiding local optima issues,
  but are applicable only to certain types of graphical models.
  On the other hand, pure likelihood objectives are difficult to optimize.
In this work, we show that using the method of moments in conjunction
  with composite marginal likelihood
  yields consistent parameter estimates for a much broader
  class of discrete directed and undirected graphical models,
  including loopy graphs with high treewidth.
%In particular, our algorithm applies to any graphical model wherein each
  %hidden variable has at least three conditionally independent observed
  %variables.
  %that sidestep local optima issues for a general class of
  %directed and undirected graphical models.
%Parameter estimation in latent-variable models based on EM
  %is prone to local optima.
  %due to the non-convexity of the marginal log-likelihood. 
%In this paper, we extend recent advances in the method of moments 
  %to provide consistent parameter estimates
  %that sidestep local optima issues for a general class of
  %directed and undirected graphical models.
%In particular, our algorithm applies to any graphical model wherein each
  %hidden variable has at least three conditionally independent observed
  %variables.
%This family includes loopy graphs with high treewidth.
Specifically, we use tensor factorization to reveal partial information
about the hidden variables.  This renders the otherwise non-convex
likelihood objective convex, and EM can then be used to attain the global optimum.
%We use the method of moments to tie each hidden to some observed
  %variables, and show that the rest of the problem can be cast as convex
  %optimization of the likelihood.
%Our approach gracefully extends to models outside our class by
  %incorporating the partial information via posterior regulraization.
\end{abstract} 

%\paragraph{Last updated:} \today, \currenttime
%
%\paragraph{Errata / Incomplete}
%\begin{enumerate}
%  \item Use \texttt{\textbackslash left(}.
%  \item Use $\ell$ instead of $L$.
%  \item convex and negative log likelihood.
%\end{enumerate}
%

