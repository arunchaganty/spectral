\section{Discussion}
\label{sec:discussion}

% Summarize paper
For latent-variable models,
there has been tension between
local optimization of likelihood,
which is broadly applicable but offers no global theoretical guarantees,
and the spectral method of moments, which provides consistent estimators
but are limited to models with special structure.
The purpose of this work is to show that the two methods
can be used synergistically to produce consistent estimates
for a broader class of directed and undirected models.
%for example, undirected log-linear models with high treewidth.

% Bottleneck is key: previous work is preprocessing
Our approach provides consistent estimates for
a family of models in which each hidden variable is a \emph{bottleneck}---that is,
it has three conditionally independent observations.
This bottleneck property of \citet{anandkumar13tensor}
has been exploited in many other contexts,
including latent Dirichlet allocation \cite{anandkumar12lda},
mixture of spherical Gaussians \cite{hsu13spherical},
probabilistic grammars \cite{hsu12identifiability},
noisy-or Bayesian networks \cite{halpern2013unsupervised},
mixture of linear regressions \cite{chaganty13regression},
and others.
Each of these methods can be viewed as ``preprocessing'' the given model into a
form that exposes the bottleneck or tensor factorization structure.
The model parameters correspond directly to the solution of the factorization.

% Our work: post-processing
In contrast, the bottlenecks in our graphical models are given by assumption,
but the conditional distribution of the observations given the bottleneck
can be quite complex.
Our work can therefore be viewed as ``postprocessing'',
where the conditional moments recovered from tensor factorization
are used to further obtain the hidden marginals and eventually the parameters.
Along the way, we developed the notion of exclusive views and bidependent sets,
which characterize conditions under which the conditional moments can reveal
the dependency structure between hidden variables.
We also made use of custom likelihood functions which were constructed to be
easy to optimize.

%We also show that bottlenecks lead to a notion of \emph{exclusive views}
%(\definitionref{exclusive-views}), the key property that allows us to learn the
%correlation structure between hidden variables.

%Much recent work on method of moments for estimating latent-variable models
%has focused on increasing applicability to broader range of models,

%In this paper, we we've described an algorithm that can consistently
%  estimate parameters for the class of directed and undirected graphical
%  models for which each hidden variable has at least three conditionally
%  independent observed variables (\propertyref{bottleneck}).

%This condition is sufficiently general to capture a variety of popular
  %models like latent trees and grids (like a pairwise latent Markov random field).
%The undirected model setting applies to log-linear models, allowing
%  for discriminative featurization; furthermore, in practice, the
%  bottleneck condition can be enforced on a model if the observed
%  features can be split into two uncorrelated sets.

% Correlation structure
%Other authors \citep{anandkumar12lda, anandkumar2013linear,
%  halpern13noisyor} learn parameters for models outside our
%  family, but they do so with additional knowledge of the correlation structure
%  between hidden variables, allowing them to ``subtract'' out these correlations and
%  obtain a graph that is bottlenecked.
% TOO BOLD
%Thus, while \propertyref{bottleneck} is much stronger than
%  \propertyref{exclusive-views}, we expect that without any knowledge of
%  the correlation between variables, \propertyref{bottleneck} is
%  necessary for identifiability.
%Our work uses the \TensorFactorize algorithm presented by
%  \citet{anandkumar13tensor}, first
%  studied the consistent parameter estimation algorithm for the
%  three-view mixture model.
%
%\todo{What about Mossel and Roch?}
%For example,
%\citet{anandkumar12lda} are able to recover parameters for correlated
%  discrete mixtures, e.g. latent Dirichlet allocation; their approach
%  follows that of the three-view mixture model, with a pre-processing step
%  that subtracts the correlation between the views.
%\citet{anandkumar2013linear} study a special family of Bayesian networks
%  in which the observed variables are linearly related to a sparse set of
%  hidden variables. Once again, the key step is to identify the
%  correlations between observed variables and exclude them from the views.
%\citet{halpern13noisyor} propose an algorithm to recover factors for
%  a more densely connected bipartite noisy-or network; they also use
%  bottlenecks to learn parameters, but exploit properties of the model
%  to negate correlations from learned edges, allowing for more
%  bottlenecks to be discovered. 
%Our work makes a different tradeoff: we do not assume any special correlation structure,
%but do rely on every hidden variable having an exclusive view so that we can
%learn the underlying structure.

%work to settings that do not meet the
  %criterion \propertyref{bottleneck}, but rely on knowing more about
  %the underlying correlation structure of the problem.
%In comparison, our work applies to a general family of models, including
%  latent tree models, directed and undirected grids, with no assumptions
%  on the correlation structure aside from independences encoded in the
%  graph. 

% Observable operator
Another prominent line of work in the method of moments community has
  focused on recovering {\em observable operator
  representations} \citep{jaeger2000observable,hsu09spectral,bailly2010spectral,balle12automata}.
  These methods allow prediction of new observations, but do not
  recover the actual parameters of the model, making them difficult to use
  in conjunction with likelihood-based models. % which are driven by parameters.
\citet{song2011spectral} proposed an algorithm to learn observable
  operator representations for latent tree graphical models, like the
  one in \figureref{examples-tree}, assuming the graph is bottlenecked. 
Their approach is similar to our first step of learning conditional moments,
  but they only consider trees.
\citet{parikh12spectral} extended this approach to general graphical
  models which are bottlenecked using a latent junction tree representation. 
  Consequently, the size of the observable representations is exponential in
  the treewidth.
  In contrast, our algorithm only constructs moments of the order of size of the cliques
  (and sub-neighborhoods for pseudolikelihood), which can be much smaller.
%In some sense, an observable operator re-formulation of our algorithm
  %subsumes this work.

% Identifiability
An interesting direction is to examine the necessity of the bottleneck
property.  Certainly, three views is in general needed to ensure
identifiability \cite{kruskal77three}, but requiring \emph{each} hidden variable to be
a bottleneck is stronger than what we would like.  We hope that by judiciously
leveraging likelihood-based methods in conjunction with the method of moments,
we can generate new hybrid techniques for estimating
even richer classes of latent-variable models.
