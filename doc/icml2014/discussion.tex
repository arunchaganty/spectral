\section{Discussion}
\label{sec:discussion}

% Summarize paper
For latent-variable models,
there exists a tension between
local optimization of likelihood,
which is broadly applicable but has no theoretical guarantees,
and the method of moments, which provides consistent estimators
but are limited to models with special structure.
The purpose of this work is to show that the two methods
can be used synergistically to produce consistent estimates
for a broader class of directed and undirected models.
A notable example are undirected log-linear models with hidden variables of
high treewidth.
% Classes of models

% Repeat
%Our approach can be summarized: (i) perform tensor factorization \citep{anandkumar13tensor} to
%obtain information about each hidden variable in the form of conditional moments;
%In some sense, the presence of these views helps tie down the hidden
%  variables to the data. 
%(ii) use composite likelihood to estimate the marginal distributions of latent variables;
%and (iii) in the case of undirected models,
%use the marginal distributions to formulate a convex optimization to
%recover the model parameters.

% Bottleneck is key
Our approach provides consistent estimates for
models in which each hidden variable is a ``bottleneck''---that is,
it has three conditionally independent observations,
(although the dependence on the observations could be quite complex as a function of the parameters).
This bottleneck property of \citep{anandkumar13tensor}
has been exploited in many other contexts,
including Latent Dirichlet Allocation \cite{anandkumar12lda},
parsing \cite{hsu12identifiability},
noisy-or Bayesian networks \cite{halpern13noisyor},
mixture of linear regressions \cite{chaganty13regression},
and others.
We also introduce the idea of \emph{exclusive views} (\propertyref{exclusive-views}),
the key property that allows us to learn the correlation structure between hidden variables.

%Much recent work on method of moments for estimating latent-variable models
%has focused on increasing applicability to broader range of models,

%In this paper, we we've described an algorithm that can consistently
%  estimate parameters for the class of directed and undirected graphical
%  models for which each hidden variable has at least three conditionally
%  independent observed variables (\propertyref{bottleneck}).

%This condition is sufficiently general to capture a variety of popular
  %models like latent trees and grids (like a pairwise latent Markov random field).
%The undirected model setting applies to log-linear models, allowing
%  for discriminative featurization; furthermore, in practice, the
%  bottleneck condition can be enforced on a model if the observed
%  features can be split into two uncorrelated sets.

% Correlation structure
Other authors, e.g \citet{anandkumar12lda, anandkumar2013linear,
  halpern13noisyor}, learn parameters for models outside this
  family, they do so with some knowledge of the correlation structure
  between hidden variables. 
In a pre-processing step, they ``subtract'' out the correlations and
  recover a graph that satisfies the bottleneck property
  (\propertyref{bottleneck}).
% TOO BOLD
%Thus, while \propertyref{bottleneck} is much stronger than
%  \propertyref{exclusive-views}, we expect that without any knowledge of
%  the correlation between variables, \propertyref{bottleneck} is
%  necessary for identifiability.
%Our work uses the \TensorFactorize algorithm presented by
%  \citet{anandkumar13tensor}, first
%  studied the consistent parameter estimation algorithm for the
%  three-view mixture model.
%
%\todo{What about Mossel and Roch?}
For example,
\citet{anandkumar12lda} are able to recover parameters for correlated
  discrete mixtures, e.g. latent Dirichlet allocation; their approach
  follows that of the three-view mixture model, with a pre-processing step
  that subtracts the correlation between the views.
\citet{anandkumar2013linear} study a special family of Bayesian networks
  in which the observed variables are linearly related to a sparse set of
  hidden variables. Once again, the key step is to identify the
  correlations between observed variables and exclude them from the views.
\citet{halpern13noisyor} propose an algorithm to recover factors for
  a more densely connected bipartite noisy-or network; they also use
  bottlenecks to learn parameters, but exploit properties of the model
  to negate correlations from learned edges, allowing for more
  bottlenecks to be discovered. 
Our work makes a different tradeoff: we do not assume any special correlation structure,
but do rely on every hidden variable having an exclusive view so that we can
learn the underlying structure.

%work to settings that do not meet the
  %criterion \propertyref{bottleneck}, but rely on knowing more about
  %the underlying correlation structure of the problem.
%In comparison, our work applies to a general family of models, including
%  latent tree models, directed and undirected grids, with no assumptions
%  on the correlation structure aside from independences encoded in the
%  graph. 

% Observable operator
Another prominent line of work in the method of moments community has
  focused on recovering an {\em observable operator
  representation} of the observations.
  These methods which allows for prediction of new observations, but does not
  identify the actual parameters of the model, which makes them difficult to use
  in conjunction with likelihood-based models which are driven by parameters.
\citet{song2011spectral} propose an algorithm to learn an observable
  operator representations for latent tree graphical models, like the
  one in \figureref{examples-tree}, assuming the bottleneck property (\propertyref{bottleneck}). 
Their approach is similar to our first step of learning conditional moments,
  but only consider trees.
\citet{parikh12spectral} extend this approach to general graphical
  models in which \propertyref{bottleneck} holds, but consider the latent
  junction tree representation. 
  Consequently, the size of the observable representations is exponential in
  the treewidth.
  In contrast, our algorithm only constructs moments of the order of size of the cliques
  (and neighborhoods for pseudolikelihood), which can be much smaller.
For example, an $n\times n$ grid model has a treewidth of $n$, but each
  clique is of size at most $2$ and the degree is at most $2$.
%In some sense, an observable operator re-formulation of our algorithm
  %subsumes this work.

% Identifiability
An interesting direction of study is to examine the necessity of the bottleneck
property.  Certainly, to some extent, three views are necessary to ensure
identifiability \cite{kruskal77three}, but requiring \emph{each} hidden variable to be
a bottleneck is still stronger than we would like.  However, we hope that judiciously
leveraging likelihood based methods in conjunction with the method of moments,
as we have done in this paper, can generate new hybrid techniques for estimating
an even richer class of latent-variable models.
