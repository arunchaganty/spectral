\subsection{Recovering conditional moments}
\label{app:assumption-proof}

% Define up front that we will focus on h, x_1, x_2, x_3
In step 1 of \LearnMarginals, we used the bottleneck property of a hidden
  variable $h_i$ to learn conditional moments $\mOpp{v}{i} \eqdef
  \Pr(x_v | h_i)$ for every view $x_v \in \sV_{h_i}$ using
  \TensorFactorize. 
In order to do so, we require that \assumptionref{full-rank} holds, i.e.
\begin{assumption*}
  Given a bottleneck $h_1$ with views $x_1, x_2, x_3$, the conditional
  moments $\mOpp{1}{1}, \mOpp{2}{1}, \mOpp{3}{1}$ have full column rank
  $k$, and $\pi\oft 1 \succ 0$.
\end{assumption*}

To interpret this assumption better, we would like find an equivalent
condition on the model parameters $Z_\sC$ that imply the above condition
on $\mOpp{v}{i}$. In this section, we will show that
\assumptionref{full-rank} is actually implied by the following condition
(\assumptionref{full-rank-plus});
\begin{assumption*}
For every clique $\sC \in \sG$ (including ones involving observed
  variables) and for every {\em conditioning} of the clique's marginals,
  every unfolding has full rank. 
\end{assumption*}

\paragraph{A recursive construction of $\mOpp{v}{i}$}

Without loss of generality, let $i = 1$. We will express the conditional
distribution $\mOpp{v}{1} \eqdef \Pr(x_v | h_1)$ recursively through the
parents of $x_v$, its parents, and so on. The approach is analogous to
message passing, and the key idea is to express the same in the form of
a tensor multiplication. \figureref{message-proof} outlines the
procedure.

\begin{figure}
  \caption{A recursive construction of the conditional moments $\mOpp{v}{i}$.}
  \label{fig:message-proof}
\end{figure}

Recall that $x_v$ is the observed variable in consideration; let $h_t$
be its unique parent. Then, 
\begin{align}
  \mOpp{v}{1} &\eqdef \Pr( x_v \given h_1 ) \nonumber \\
              &= \sum_{h_t}  \Pr( h_t \given h_1 ) \Pr( x_v \given h_t ) \nonumber\\
              &= \mOpp{v}{t} \mYpp{t}{1}, \label{eqn:expanding-O}
\end{align}
where $\mYpp{i}{j} \eqdef \Pr( h_i \given h_j )$. 
More generally, for two sets of hidden variables $C \eqdef \{h_{C_1}
\cdots h_{C_m} \}$ and $C' \eqdef \{h_{C'_1} \cdots h_{C'_n} \}$, 
define $\mYpp{C}{C'} \eqdef \Pr( h_{C_1} \cdots h_{C_m} \given h_{C'_1}
\cdots h_{C'_n} )$. 
With this representation, $\mYpp{C}{C'}$ is a $|C| + |C'|$-th order
tensor.
  
Going forward, we will construct $\mYpp{v}{1}$ recursively.
Without loss of generality, let $h_1, h_2, \cdots, h_t$ be a topological
  ordering rooted at $h_1$,
and let $\Pa(h)$ be the parents of a hidden variable $h$ in
  this topological ordering.
Then $\mYpp{t}{1}$ can be expressed as,
\begin{align*}
  \mYpp{t}{1} &\eqdef \Pr( h_t \given h_1 )  \\
  &= \sum_{\vh \in H_{\Pa(h_t)}} \Pr( \Pa(h_t) \given h_1 ) \Pr( h_t \given \Pa(h_t) ) \\
  &= \mYpp{t}{\Pa(h_t)} \times_{\Pa(h_t)} \mYpp{ \Pa(h_t) }{1},
\end{align*}
where $A \times_{C} B$ refers to summation along the indices $C$. We
refer to the operation $A \times_C B$ as tensor multiplication.
\appendixref{tensor-multiplication} proves some key properties of this
operation. 

Note that $\mYpp{t}{\Pa(h_t)}$ is the conditional probability $\Pr(h_t
 | \Pa(h_t))$, which is assumed to be full rank, and can be easily
 computed from the hidden marginals $Z_\sC$ that contain $h_t$. 

Now, let's further expand $\mYpp{ \Pa(h_t) }{1}$ in terms of {\em its
  parents}, recursively doing so until we reach $h_1$.
Let $C$ be an intermediate set of hidden variables. Let $h_c \neq h_1$
  be some hidden variable in $C$ which we wish to eliminate. 
Let $C' = C \union \Pa(h_c) \setminus \{h_c\}$, the new set of hidden
  variables containing the parents of $h_c$ instead of $h_c$ and $\del
  C = C' \setminus (C \union \{h_1\})$, the variables in the interface
  between $C$ and $C'$.
Note that $\Pa(h_c) \subseteq \del C$.
Then,
\begin{align}
  \mYpp{C}{1} &\eqdef \Pr( h_c \given h_1 ) \nonumber \\
  &= \sum_{\vh \in H_{\del C}} \Pr( h_c \given \Pa(h_c) ) \Pr( C' \given h_1 ) \nonumber \\
  &= \mYpp{c}{\Pa(h_c)}  \times_{\del C} \mYpp{ C' }{ 1 }. \label{eqn:recursive-step}
\end{align}
This process is repeated until the base case, $\mYpp{1}{1} = \ones$.
\algorithmref{Y} summarizes the procedure.

\begin{algorithm}
  \caption{$\mYpp{C}{1}$}
  \label{algo:Y}
  \begin{algorithmic}
    \REQUIRE The root $h_1$, a set of hidden variables $C$.
    \ENSURE The hidden moments distribution $\mYpp{C}{1}$.
    \IF{ $C = \{h_1\}$ }
      \STATE $\mYpp{1}{1} = \ones$.
    \ELSE
      \STATE Let $h_c \neq h_1$ be some hidden variable in $C$.
      \STATE Let $C' = C \union \Pa(h_c) \setminus \{h_c\}$.
      \STATE $\mYpp{C}{1} = \mYpp{C'}{1} \times_{\Pa(h_c)} \mYpp{c}{\Pa(h_c)}$.
    \ENDIF
  \end{algorithmic}
\end{algorithm}

\paragraph{Constructing $\mYpp{c}{\Pa(h_c)}$}

The algorithm constructs an expression for $\mYpp{t}{1}$ entirely as
  tensor multiplications of terms of the form $\mYpp{c}{\Pa(h_c)}$ for
  some hidden variable $h_c$. 
The individual $\mYpp{c}{\Pa(h_c)}$ can be easily computed from the
  clique marginals $Z_\sC$ as follows.

Without loss of generality, let $\sC \eqdef \{c\} \union \Pa(h_c)$ be a clique
  in the graph $\sG$.
If instead $\sC \subseteq \sC_1 \union \sC_2$ for some $\sC_1, \sC_2 \in
  \sG$, we can simply construct $Z_{\sC}$ from the union of $\sC_1$ and
  $\sC_2$,
\begin{align*}
  Z_\sC &= Z_{\sC_1} \times_{\del \sC} Z_{\sC_2},
\end{align*}
where $\del \sC = (\sC_1 \union \sC_2) \setminus \sC$.

Proceeding,
\begin{align*}
\mYpp{c}{\Pa(h_c)} 
  &\eqdef \Pr(h_c \given \Pa(h_c)) \\
  &= \frac{\Pr(h_c, \Pa(h_c))}{\Pr(\Pa(h_c))} \\
  &= Z_\sC \times_{\Pa(h_c)} \diag{Z_\sC(\ones, \cdot, \ldots, \cdot)}^{-1},
\end{align*}
where $\diag(\sP)$ is a ``diagonal'' tensor constructed by taking the
tensor product with $I^{\otimes p}$:
\begin{align*}
  \diag(\sP)[\vi,\vj] &\eqdef \sP[\vi] \delta[\vi,\vj].
\end{align*}

\paragraph{Sample complexity results}

With an expression for $\mOpp{v}{i}$, we are finally ready to describe
sufficient conditions for $\mOpp{v}{i}$ to be full rank, along with the
sample complexity for estimation.

\begin{lemma}[Sufficiency of \assumptionref{full-rank-plus}]
  \label{lem:full-rank-suff}
  Given the full rank conditions in \assumptionref{full-rank-plus}, then
  for any hidden variable $h_i$ and observed variable $x_v$,
  $\mOpp{v}{i}$ has full column rank.
\end{lemma}
\begin{proof}
Again, wlog, let $i=1$. 
The key step of the proof will be to use the property that the tensor
  multiplication in \equationref{recursive-step} preserves rank.

\theoremref{tensor-multiplication} shows that for any two tensors $A,
  B$ and an index set $C$, the unfoldings $I$ of $A \times_C B$ are full
  rank when the projected unfoldings $I_A$ of $A$ and $I_B$ of $B$ are
  full rank:
\begin{align*}
  \sigma_{\min}( (A \times_C B)\munf{I} )
    &\ge \sigma_{\min}(A\munf{I_A}) \sigma_{\min}(B\munf{I_B}).
\end{align*}

By the assumption, any unfolding of $\mYpp{c}{\Pa(h_c)}$ is full-rank. 
From \equationref{recursive-step} and
  \theoremref{tensor-multiplication}, this implies that $\mYpp{C}{1}$ is
  full rank for any $C$, including $C = \{h_t\}$.

Thus, $\mYpp{t}{1}$ is a full rank matrix. 
Again, by assumption,
  $\mOpp{v}{t}$ is full rank. From \equationref{expanding-O}, we finally
  get that $\mOpp{v}{1}$ is full rank.
\end{proof}

Next, we turn to the question of the sample complexity of estimating
$\mOpp{v}{i}$, which depends mainly on smallest singular value,
$\sigma_{\min}(\mOpp{v}{i})$. 

\begin{lemma}[Singular values of $\mOpp{v}{1}$]
  \label{lem:mopp-singular-values}

Let $h_t$ be the unique parent of $x_v$, as above.
Let $h_1 \succ h_2 \cdots \succ h_t$ be a topological ordering of
  variables according to the graph $\sG$.
  Then, the smallest singular value of $\mOpp{v}{1}$ is at least the product of the smallest singular values of $\mYpp{c}{\Pa(h_c)}$.
\begin{align*}
  \sigma_{\min}(\sP) &\ge \sigma_{\min}(\mOpp{v}{t}) 
        \prod_{c \in [t]} \sigma_{\min}(\mYpp{c}{\Pa(h_c)}),
\end{align*}
where the parents $\Pa(h_c)$ are decided by the topological ordering. 
\end{lemma}
\begin{proof}
  The proof follows directly by application of
  \theoremref{tensor-multiplication} to \equationref{expanding-O} and
  \equationref{recursive-step}.
\end{proof}

Finally, using the tensor power method of \citep{anandkumar13tensor}, we
  get the following result on sample complexity.
\begin{theorem}[Sample complexity for $\mOpp{v}{i}$]
  \label{thm:sample-complexity-1}
  Wlog, let $v=1$ and $i=1$. Let $x_1, x_2, x_3$ be three views for
  $h_1$.  
  If 
\begin{align*}
  \|\hat M_{1,2} - M_{1,2}\|_{\op} &\le \epsilon & \|\hat M_{1,2,3} - M_{1,2,3}\|_{\op} &\le \epsilon,
\end{align*}
for some $\epsilon < \half$, then
with probability at least $1 - \delta$,
\begin{align*}
  \|\mOpphat{1}{1} - \mOpp{1}{1}\|_F 
    &\le  \\
    &
      O\left( k 
      \frac{{\pi\oft{1}}_{\max}/{\pi\oft{1}}_{\min}} 
      {(\sigma_{\min}(\mOpp{1}{1}) \sigma_{\min}(\mOpp{1}{2}))^{5/2}} \right) \epsilon.
\end{align*}

Furthermore, if $\sigma_{\min}(\mYpp{c}{\Pa(h_c)}) \le \sigma$ for every
  such
$\mYpp{c}{\Pa(h_c)}$, we get,
\begin{align*}
  \|\mOpphat{1}{1} - \mOpp{1}{1}\|_F 
    &\le 
      O\left( k 
      \frac{{\pi\oft{1}}_{\max}/{\pi\oft{1}}_{\min}} 
      {\sigma^{5t}} \right) \epsilon,
\end{align*}
where $t$ is the length of the topological ordering, $h_1 > \cdots
> h_t$ as defined above.
\end{theorem}
\begin{proof}
  The first statement follows directly from Theorem 5.1 of
  \citet{anandkumar13tensor} by noting that $\|M_{1,2}\|_\op \le 1$ and
  $\|M_{1,2,3}\|_\op \le 1$ (as they represent probability
  distributions). The second statement follows directly from
  \lemmaref{mopp-singular-values}.
\end{proof}

