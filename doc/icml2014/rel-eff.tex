\subsection{Comparing the pseudoinverse and composite likelihood estimators}
\label{app:rel-eff}

In \lemmaref{mom-pw-variance}, we derived concrete expressions for the
asymptotic variances of the pseudoinverse and composite likelihood
estimators, $\Sigmamom$ and $\Sigmaml$ respectively. 
In this section, we will use the asymptotic variances to compare the two
estimators for a few special cases.

We look at relative efficiency as a quantitative measure: $e^\mom
= \frac{1}{\bbk} \Tr(\Sigmaml \Sigmamomi)$, where $\bbk = k-1$. The
relative efficiency $e^\mom$ lies between $0$ and $1$. When $e^\mom
= 1$, the pseudo-inverse estimator is said to be efficient.

First, let us consider the case where $d = k$,
\begin{lemma}[Relative efficiency when $d = k$]
  When $d = k$, the asymptotic variances of the pseudoinverse and
  composite likelihood estimators are equal, $\Sigmaml = \Sigmamom$, and the relative efficiency
  is $1$.
\end{lemma}
\begin{proof}
Under our assumptions, when $d=k$, $\tO$ is invertible. This allows us
to simplify the expression of the asymptotic variance of the composite
likelihood estimator, $\Sigmaml$, as follows, 
\begin{align*}
    \Sigmaml 
      &= \left( \tO^\top(\tD\inv + \td\inv \ones \ones^\top) \tO \right)\inv \\
      &= \tO\inv (\tD\inv - \td\inv \ones \ones^\top)\inv \tO\tinv \\
      &= \tO\inv (\tD - \frac{\tD \ones \ones^\top \tD}{\td + \ones^\top \tD \ones}) \tO\tinv.
\end{align*}
Note that $\tD \ones = \tm$ and $\td = 1 - \ones^\top \tm$. This gives us,
\begin{align*}
    \Sigmaml 
      &= \tO\inv (\tD - \frac{\tm \tm^\top}{1 - \ones^\top \tm + \ones^\top \tm}) \tO\tinv \\
      &= \tO\inv (\tD - \tm \tm^\top) \tO\tinv \\
      &= \Sigmamom.
\end{align*}
\end{proof}

Next, we consider the worst case where the observed moments $\mu$ is the
uniform distribution.

\begin{lemma}[Relative efficiency with uniform observed moments]
  Let the observed moments, $\mu$, be uniform, $\mu = \frac{1}{d} \ones$. 
  The efficiency of the pseudo-inverse estimator is, 
  \begin{align}
    e^\mom &= 
    1 - \frac{1}{\bbk}\frac{\|\ones_U\|^2}{1 + \|\ones_U\|^2} \left( 1 - \frac{1}{d - \|\ones_U\|^2} \right),
  \end{align}
  where $\ones_U \eqdef \tOi\tO \ones$, the projection of $\ones$ onto
  the column space of $\tO$. Note that $0 \le \|\ones_U\|_2 \le \sqrt{\bbk}$.

  When $\|\ones_U\|_2 = 0$, the pseudoinverse estimator is efficient:
  $e^\mom = 1$. When $\|\ones_U\|_2 > 0$ and $d > k$, the pseudoinverse
  estimator is strictly inefficient. In particular, when $\|\ones_U\|_2
  = \frac{d-1}{2}$ the efficiency attains its minimum, 
  \begin{align*}
    e^\mom 
    &= 1 - \frac{1}{\bbk} \left( \frac{d-1}{d+1} \right)^2.
      %&= 1 - \frac{1}{\bbk +1} \left( 1 - \frac{1}{d - \bbk} \right).
  \end{align*}
\end{lemma}
\begin{proof}
Next, let us consider the case where the moments are the uniform
distribution, where $\mu = \frac{\ones}{d}$ and $\tD = \frac{1}{d} I$.
The expressions for $\Sigmaml$ can be simplified as
follows,
\begin{align*}
    \Sigmaml 
      &= \left( \tOt(d I + d \ones \ones^\top) \tO \right)\inv \\
      &= \frac{1}{d} \left( \tOt\tO + \tOt\ones \ones^\top \tO \right)\inv \\
      &= \frac{1}{d} \left(
      (\tOt\tO)\inv - \frac{(\tOt\tO)\inv \tOt\ones \ones^\top \tO (\tOt\tO)\inv}
        {1 + \ones^\top \tO (\tOt\tO)\inv \tOt\ones} 
        \right)\\
      &= \frac{1}{d} \left(
      \tOi\tOit - \frac{\tOi\tOit\tOt \ones \ones^\top \tO\tOi\tOit}
        {1+ \ones^\top \tO \tOi \tOit \tOt\ones} 
        \right)
        \\
      &= \frac{1}{d} \left(
      \tOi\tOit - \frac{\tOi \ones \ones^\top \tOit}
        {1+ \|\tO \tOi \ones\|_2^2} 
        \right) \\
      &= \frac{1}{d} \left(
      \tOi\tOit - \frac{\tOi \ones \ones^\top \tOit}
        {1+ \|\ones_U\|_2^2} 
        \right),
\end{align*}
where $\ones_U \eqdef \tO \tOi \ones = \tOit \tOt \ones$ is the projection of $\ones$ onto
the column space of $\tO$. 

Next, we can simplify the expression for $\Sigmamomi$,
\begin{align*}
    \Sigmamom &= 
      \tOi \left( \frac{I}{d} - \frac{\ones \ones^\top}{d^2} \right) \tOit \\
    \Sigmamomi &=
        \left(
        \frac{1}{d} 
          \tOi\tOit - \frac{1}{d^2} \tOi \ones \ones^\top \tOit
        \right)\inv \\
        &=
        d
          (\tOi\tOit)\inv \\
        &\quad
          + d^2
          \frac{
              (\tOi\tOit)\inv \tOi \ones \ones^\top \tOit (\tOi\tOit)\inv 
          }{
            d - \ones^\top \tOit (\tOi\tOit)\inv \tOi \ones 
          } \\
          &=
        d
          \tOt\tO + d^2
          \frac{
              \tOt\tO \tOi \ones \ones^\top \tOit \tOt\tO
          }{
            d - \ones^\top \tOit \tOt\tO \tOi \ones 
          } \\
          &=
        d \left(
          \tOt\tO +
          \frac{
              \tOt \ones \ones^\top \tO
          }{
            d - \|\tOi \tO \ones\|_2^2 
          } \right) \\
          &=
        d \left(
          \tOt\tO +
          \frac{
              \tOt \ones \ones^\top \tO
          }{
            d - \|\ones_U\|_2^2 
          } \right).
\end{align*}

Now, we are ready to study the relative efficiency. 
\begin{align*}
  e^\mom &= \frac{1}{\bbk} \Tr(\Sigmaml \Sigmamomi) \\
    &= \frac{1}{\bbk}
    \Tr \Bigg( \frac{1}{d} \left(
      \tOi\tOit - \frac{\tOi \ones \ones^\top \tOit}
        {1+ \|\ones_U\|_2^2} 
        \right) \\
    &\quad \hphantom{\bbk \Tr}
        d
          \left(
          \tOt\tO +
          \frac{
              \tOt \ones \ones^\top \tO
          }{
            d - \|\ones_U\|_2^2 
          }
          \right)
          \Bigg).
%%%%
\end{align*}

Next, applying the property that the trace is invariant under cyclic
permutation, we get,
\begin{align*}
  e^\mom
    &= 
    \frac{1}{\bbk} \Tr(I) + \frac{1}{\bbk} \Tr\left(
          \frac{
              \tOi\tOit \tOt \ones \ones^\top \tO
          }{
            d - \|\ones_U\|_2^2 
          } \right) \\
    &\quad
    - \frac{1}{\bbk} \Tr\left(
      \frac{\tOi \ones \ones^\top \tOit \tOt\tO}{1+ \|\ones_U\|_2^2} 
      \right) \\
    &\quad
    - \frac{1}{\bbk} \Tr\left(
        \frac{\tOi \ones \ones^\top \tOit \tOt \ones \ones^\top \tO
}{(d - \|\ones_U\|_2^2 )(1+ \|\ones_U\|_2^2)} 
          \right) \\
%%%%
    &= 1 + \frac{1}{\bbk}
          \frac{
              \|\tOit \tOt \ones\|_2^2
          }{
            d - \|\ones_U\|_2^2 
          } 
        - \frac{1}{\bbk}
        \frac{\|\tO \tOi \ones\|_2^2}{1+ \|\ones_U\|_2^2} \\
    &\quad
    - \frac{1}{\bbk}
    \frac{\|\tOit \tOt \ones\|_2^2  \|\tO \tOi \ones\|_2^2}
        {(d - \|\ones_U\|_2^2 ) (1+ \|\ones_U\|_2^2)} \\
&= 1
+ \frac{1}{\bbk} \frac{\|\ones_U\|^2}{d - \|\ones_U\|^2}
    -  \frac{1}{\bbk}\frac{\|\ones_U\|^2}{1 + \|\ones_U\|^2}\\
&\quad
    -  \frac{1}{\bbk}\frac{\|\ones_U\|^4}{(1 + \|\ones_U\|^2)(d - \|\ones_U\|^2)} \\
    &= 1 - \frac{\|\ones_U\|^2}{\bbk (1 + \|\ones_U\|^2)} \left( 1 - \frac{1}{d - \|\ones_U\|^2} \right).
\end{align*}

Note that $\ones_U$ is the projection of $\ones$ on to a $k$-dimensional
subspace, thus, $0 \le \|\ones_U\|_2 \le \sqrt{\bbk}$.
When $\ones_U = \zeros$, the relative efficiency $e^\mom$ is $1$: the pseudo-inverse estimator is inefficient.
When $\ones_U = \sqrt{\bbk}$, the relative efficiency is,
\begin{align*}
  e^\mom 
    &= 1 - \frac{1}{\bbk +1} (1 - \frac{1}{d - \bbk}) \\
    &= 1 - \frac{1}{k +1} (1 - \frac{1}{1 + d - k}).
\end{align*}

This tells us that the pseudo-inverse estimator is strictly inefficient
when $d > k$. The inefficiency grows as $d$ becomes larger for a fixed
$k$ and decreases if $k$ increases (for a fixed $d$).

If $k > d/2$, then the efficiency is at its minimum when $\|\ones_U\|^2
= \frac{1}{2} (d-1)$, i.e.
\begin{align*}
  e^\mom 
  &\ge 1 - \frac{1}{\bbk}\left(\frac{d-1}{d+1}\right)^2.
\end{align*}

\end{proof}

% When is $\|\ones_U\| = \sqrt{\bbk}$? When $\ones$ is in the column space of $\tO$. 
% When is $\|\ones_U\| = 0$?

% Firstly, if $\ones$ lies in the null space of $\tO$, then the variances
% of the two are equal. 

% Next, consider the trace of the two, which gives
% us a measure of $\|\hat{z}^\mom\|^2 - \|\hat{z}^\ml\|^2$,
% \begin{align*}
%     \Tr(\Sigmamom - \Sigmaml)
%     &= \frac{1}{d} 
%     \Tr(\tOi \ones \ones^\top \tOit) \left(
%     \frac{1}{1+ \| \tO \tOi \ones\|^2} - \frac{1}{d}
%     \right) \\
%     &= \frac{1}{d} \|\tOi \ones\|^2 \left(
%     \frac{1}{1+ \| \tO \tOi \ones\|^2} - \frac{1}{d}
%     \right) \\
%     &= \frac{1}{d} \|\tOi \tO \tOi \ones\|^2 \left(
%     \frac{1}{1+ \| \tO \tOi \ones\|^2} - \frac{1}{d}
%     \right) \\
%     &\ge \frac{\sigma_{k-1}(\tO)}{d} \|\tO \tOi \ones\|^2 \left(
%     \frac{1}{1+ \| \tO \tOi \ones\|^2} - \frac{1}{d}
%     \right).
% \end{align*}
% 
% It remains for us to bound the projection $\|\tO \tOi \ones\|$. Observe,
%  \begin{align*}
%    O^\top \ones 
%      &= \sum_{\bh = \ones}^{\bh \prec k} \sum_{x = \ones}^{d} \Pr(x | \bh) e_\bh  \\
%      &= \ones \\
%    \tOt \ones
%        &= (O_{\neg d, k}^\top - \ones O_{\neg d, k}^\top) \ones  \\
%        &= \sum_{\bh = \ones}^{\bh \prec k} \sum_{x = \ones}^{x \prec d} \Pr(x | \bh) e_\bh - (\sum_{x = \ones}^{x \prec d} \Pr(x | k)) \ones^\top  \\
%        &= \sum_{\bh = \ones}^{\bh \prec k} (1 - \Pr(d | \bh)) e_\bh - (1 - \Pr(d | k)) \ones  \\
%        &= \sum_{\vh = \ones}^{\vh \prec k} (\Pr(d | k) - \Pr(d | \vh))e_\bh \\
%        &= \tO_{d,k} \ones^\top - \tO_{d,\neg k}.
%  \end{align*}
%  Let $c = \|\tO^\top \ones\|_2 \le \sqrt{k}$.
%  Let $\tO = U W V^\top$ be the SVD of $\tO$. We have,
%  \begin{align*}
%    \ones^\top U W V^\top 
%        &= \ones^\top \tO \\
%    \ones^\top U 
%      &= \ones^\top \tO V W\inv \\
%    \|\ones^\top U\|_2 
%        &\le \frac{\|\ones^\top \tO\|_2}{\sigma_{k}(\tO)} \\
%        &\le \frac{c}{\sigma_{k}(\tO)} \\
%    \|\ones^\top U\|_2 
%        &\ge \frac{c}{\sigma_{1}(\tO)}.
%  \end{align*}


% With concrete expressions for $\Sigmamom$ and $\Sigmaml$, we can quantitatively evaluate their asymptotic efficiency,
% \begin{proof}[Proof for Corollary X] %\corollaryref{efficiency}]
%   Let $\bar k = k -1$ and $\bar d = d -1$. 
%   We have that
%   \begin{align*}
%     \Sigmamom 
%       &= \tOi \tD \tOit - \tOi \tD \ones \ones^\top \tD \tOit, \\
%     \Sigmaml &= 
%          \tOi \tD \tOit - \frac{\tOi \tD v v^\top \tD \tOit }{1 + v\top \tD v - \ones\top \tD \ones}.
%   \end{align*}
% 
%   We use the following identity,
%   \begin{align*}
%     \Tr( A_1 A_2\inv ) 
%       &= \Tr( (A - x x^\top) (A - yy^\top)\inv ) \\
%       &= \Tr( (A - x x^\top) (A\inv + \frac{(A\inv y)(A\inv y)^\top }{1 - y^\top A\inv y}) ) \\
%       &= \Tr( I - xx^\top A\inv + \frac{ y (A\inv y)^\top - x x^\top (A\inv y)(A\inv y)^\top}{1 - y^\top A\inv y} ) \\
%       &= k - x^\top A\inv x + \frac{ y^\top A\inv y - (x^\top A\inv y)^2 }{1 - y^\top A\inv y}.
%   \end{align*}
% 
%   Let us now consider their relative efficiencies, 
%   \begin{align*}
%     e^\mom 
%         &\eqdef \frac{1}{\bbk} \Tr(\Sigmaml \Sigmamomi ) \\
%         \Tr( \tOi \tS\inv \tOit \tOt \tilde\Sigma\inv \tO ) \\
%       &\quad - \frac{1}{\bar k} \frac{s \Tr( \tOi \tS\inv J \ones \ones^\top J \tS\inv \tOit \tOt  \tilde\Sigma\inv \tO  )}
%       {1 + s \ones^\top J \tS\inv J \ones} \\
%         &= \frac{1}{\bar k} \Tr( J \tS\inv J \tilde\Sigma\inv ) - \frac{1}{\bar k} \frac{s \ones^\top J \tS\inv J \tilde\Sigma\inv J \tS\inv J \ones }
%       {1 + s \ones^\top J \tS\inv J \ones},
%   \end{align*}
%   where $J \eqdef \tO \tOi = U U^\top$ as before. 
%   
%   Note that $\Tr(J) = \rank(J) = {\bar k}$.
%   Next we use H\"{o}lder's inequality for the trace: $\Tr(D A) \le
%   \|D\|_\infty \Tr(A)$ if $D$ is diagonal. Thus,
%   \begin{align*}
%       \Tr( J \tS\inv J \tilde\Sigma\inv ) 
%         &= \Tr( \tS\inv J \tilde\Sigma\inv J ) \\
%         &\le \|\tS\inv\|_\infty \Tr( J \tilde\Sigma\inv J ) \\
%         &= \|\tS\inv\|_\infty \Tr( \tilde\Sigma\inv J ) \\
%         &\le \|\tS\inv\|_\infty \|\tilde\Sigma\inv\|_\infty \Tr( J ) \\
%         &= {\bar k} \|\tS\inv\|_\infty \|\tilde\Sigma\inv\|_\infty.
%   \end{align*}
% 
%   To bound the next term, we need bound the projection of the ones
%   vector, $\ones$, through $J \eqdef U U^\top$,
%   \begin{align*}
%       \ones^\top J \tS\inv J \tilde\Sigma\inv J \tS\inv J \ones \\
%       &=
%       (\ones^\top U) (U^\top \tS\inv U) (U^\top \tilde\Sigma\inv U) \\
%       &\quad (U^\top \tS\inv U) (U^\top \ones)  \\
%       &\ge 
%       (\tS\inv)^2_{\min}
%       (\tilde\Sigma\inv)_{\min}
%       \|\ones^\top U\|^2_2 \\
%       &\ge 
%       \frac{\|\ones^\top U\|^2_2}{\|\tS\|^2_\infty \|\tilde\Sigma\|_\infty},
%   \end{align*}
%   where we have used the property that each of the bracketed terms is an
%   eigen-decomposition, and $U^\top \ones$, lies in the subspace $U$.
%   We similarly have that
%   \begin{align*}
%       \ones^\top J \tS\inv J \ones \\
%       &=
%       (\ones^\top U) (U^\top \tS\inv U) (U^\top \ones) \\
%       &\le 
%       \|\tS\inv\|_{\infty}
%       \|\ones^\top U\|^2_2.
%   \end{align*}
% 
%   Finally, to bound $\|\ones^\top U\|_2$, we have:
%   \begin{align*}
%     \ones^\top O 
%       &= \sum_{\bh = \ones}^{\bh \prec k} \sum_{x = \ones}^{d} \Pr(x | \bh) e_\bh  \\
%       &= \ones \\
%     \ones^\top \tO 
%         &= \ones^\top (O_{\neg \neg d, k} - O_{\neg d, k}\ones^\top) \\
%         &= \sum_{\bh = \ones}^{\bh \prec k} \sum_{x = \ones}^{x \prec d} \Pr(x | \bh) e_\bh - (\sum_{x = \ones}^{x \prec d} \Pr(x | k)) \ones^\top  \\
%         &= \sum_{\bh = \ones}^{\bh \prec k} (1 - \Pr(d | \bh)) e_\bh - (1 - \Pr(d | k)) \ones^\top  \\
%         &= \sum_{\vh = \ones}^{\vh \prec k} (\Pr(d | k) - \Pr(d | \vh))e_\bh \\
%         &= \tO_{d,k} \ones^\top - \tO_{d,\neg k}^\top.
%   \end{align*}
%   Let $c = \|\tO_{d,k} \ones^\top - \tO_{d,\neg
%   k}^\top\|_2 \le \sqrt{k}$.
%   Let $\tO = U W V^\top$ be the SVD of $\tO$. We have,
%   \begin{align*}
%     \ones^\top U W V^\top 
%         &= \ones^\top \tO \\
%     \ones^\top U 
%       &= \ones^\top \tO V W\inv \\
%     \|\ones^\top U\|_2 
%         &\le \frac{\|\ones^\top \tO\|_2}{\sigma_{k}(\tO)} \\
%         &\le \frac{c}{\sigma_{k}(\tO)} \\
%     \|\ones^\top U\|_2 
%         &\ge \frac{c}{\sigma_{1}(\tO)}.
%   \end{align*}
% 
%   Thus,
%   \begin{align*}
%       \ones^\top J \tS\inv J \tilde\Sigma\inv J \tS\inv J \ones 
%       &\ge \frac
%           {c^2}
%           { \|\tS\|_{\infty} \|\tilde\Sigma\|_{\infty}
%           \sigma_{1}(\tO)} \\
%       \ones^\top J \tS\inv J \ones  
%       &\le \frac
%           {c^2 \|\tS\inv\|_{\infty}}
%           {\sigma_{k}(\tO)}.
%   \end{align*}
% 
%   Finally, we have the result,
%   \begin{align*}
%     e^\mom 
%     &\le \|\tS\inv\|_\infty  \|\tilde\Sigma\inv\|_\infty \\
%     &\quad - 
%         \frac{1}{\bar k} 
%     \frac{
%         s c^2 /(\|\tS\|^2_{\infty} \|\tilde\Sigma\|_{\infty}
%             \sigma_{1}(\tO))
%     }
%     {1 + (s c^2 \|\tS\inv\|_{\infty})/
%           \sigma_{k}(\tO)
%     }.
%   \end{align*}
% 
%   For the uniform distribution $M[x] = \frac{1}{\bar d + 1}$,
%   $\|\tS\|_\infty = \bbd$, $\|\tilde
%   \Sigma\|_\infty = \frac{\bbd}{(\bbd + 1)^2}$:
%   \begin{align*}
%     e^\mom 
%     &\le \frac{1}{\bbd} \frac{(\bbd + 1)^2}{\bbd} 
%     - \frac{1}{\bbk} \frac{
%     \bbd c^2/(\bbd^2 \frac{\bbd}{(\bbd +1)^2} \sigma_{k}(\tO))
%     }{
%     1 + c^2 \bbd \frac{1}{\bbd} / \sigma_{k}(\tO)
%     } \\
%     &= (1 + \frac{1}{\bbd})^2 - \frac{1}{\bbk} (1 + \frac{1}{\bbd})^2 \frac{c^2/\sigma_{1}(\tO)}
%       {1 + c^2 / \sigma_{k}(\tO)} \\
%     &= (1 + \frac{1}{\bbd})^2 (1 - \frac{1}{\bbk} \frac{c^2/\sigma_{1}(\tO)}{1 + c^2 / \sigma_{k}(\tO)}).
%   \end{align*}
% 
%   This shows that the pseudoinverse estimator is strictly less efficient
%   than the composite likelihood estimator for finite $k$ and $d$.
%   Furthermore, for a fixed $\bbk$, the pseudoinverse estimator is most
%   efficient for large $\bbd$, and in general it is more efficient for
%   larger $\bbk$.
% \end{proof}
% 

% \begin{align*}
%   \Tr(\Sigmamom - \Sigmaml) 
%   &= 
%       \Tr\left( 
%       \frac{\tOi \tD v v^\top \tD \tOit }
%         {1 - \ones^\top \tD \ones + v^\top \tD v } 
%       - \tOi \tD \ones \ones^\top \tD \tOit 
%       \right) \\
%   &= 
%       \frac{v^\top \tD \tOit \tOi \tD v}
%         {1 - \ones^\top \tD \ones + v^\top \tD v } 
%       - \ones^\top \tD \tOit  \tOi \tD \ones \\
% \end{align*}



% ARUN: This is wrong because pseudo-inverses don't work that way.
%
% The above expression is the inversion of an invertible matrix plus
% a rank-one component. The Sherman-Morrison formula allows us to write
% the inverse in closed form:
% \begin{align*}
%   (A + \alpha uv^\top)\inv &= A\inv - \frac{A\inv uv^\top A\inv}{\alpha\inv + v^\top A\inv u}.
% \end{align*}
% 
% This gives us,
% \begin{align}
%     \Sigmaml
%     &= (\tOt \tD\inv \tO)\inv
%       - \frac{
%       (\tOt \tD\inv \tO)\inv (\tOt \ones) (\tOt \ones)^\top (\tOt \tD\inv \tO)\inv }
%       {\td + (\tOt \ones)^\top (\tOt \tD\inv \tO)\inv (\tOt \ones)} \nonumber \\
%     &= \tOi \tD \tOit 
%       - \frac{\tOi \tD (\tOit \tOt \ones) (\tOit \tOt \ones)^\top  \tD \tOit }
%       {\td + (\tOit \tOt \ones)^\top \tD (\tOit \tOt \ones)} \nonumber \\
%     &= \tOi \tD \tOit 
%       - \frac{\tOi \tD v v^\top \tD \tOit }
%       {\td + v^\top \tD v} \nonumber \\
%     &= \tOi \tD \tOit 
%       - \frac{\tOi \tD v v^\top \tD \tOit }
%       {1 - \ones^\top \tD \ones + v^\top \tD v }, \label{eqn:asymp-var-ml}
% \end{align}
% where $v \eqdef (\tO \tOi)^\top \ones$, the projection of $\ones$ into the
% column space of $\tO$. 
% 
% If $k = d$, then by assumption $\tO$ is invertible and $v = \ones$. In this case,
% \begin{align*}
%     \Sigmaml
%     &= \tOi \tD \tOit 
%       - \frac{\tOi \tD \ones \ones^\top \tD \tOit }
%       {1 - \ones^\top \tD \ones + \ones^\top \tD \ones } \\
%     &= \tOi \tD \tOit 
%       - \tOi \tD \ones \ones^\top \tD \tOit.
% \end{align*}
% In other words, when $k = d$, the asymptotic variance of the
% pseudoinverse estimator and the composite likelihood estimator are
% equal.


