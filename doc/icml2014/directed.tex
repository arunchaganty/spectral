\section{Consistent Parameter Estimation in Directed Models}
\label{sec:directed}

The tensor factorization method attacks the heart of the non-convexity
  in latent variable models.
The technique is central in solving several other problems with more
  complex correlation structure, for example latent Dirichlet allocation
  \citep{anandkumar12lda}, mixture of Gaussians \citep{hsu13spherical},
  etc.\findcite{more things}.

We now show that once we recover the observation
  potentials, identifying other clique potentials in a directed graphical
  model can be consistently estimated in a systematic manner.
In section \sectionref{undirected} we generalize our results to
  undirected log-linear models.

\subsection{Example: Directed grid model}

\begin{figure}
  \centering
  % vim:ft=tex
\documentclass[tikz,convert={outfile=grid.pdf}]{standalone}
%\usetikzlibrary{...}% tikz package already loaded by 'tikz' option
\usepackage{scabby-diag}
\begin{document}

\begin{tikzpicture}

% Hidden nodes
   \node[style=node, scale=0.8] (h1) at (0,0) {$h_1$};
   \node[style=node, scale=0.8, below left= 0.5cm of h1] (h2) {$h_2$};
   \node[style=node, scale=0.8, below right= 0.5cm of h1] (h3) {$h_3$};
   \node[style=node, scale=0.8, below right= 0.5cm of h2] (h4) {$h_4$};

   \draw[-latex] (h1) -- (h2);
   \draw[-latex] (h1) -- (h3);
   \draw[-latex] (h2) -- (h4);
   \draw[-latex] (h3) -- (h4);

% Observed nodes
   \node[style=obsnode, scale=0.7, above left=0.3cm of h1] (x1a) {$x^a_1$};
   \node[style=obsnode, scale=0.7, above right=0.3cm of h1] (x1b) {$x^b_1$};
   \draw[-latex] (h1) -- (x1a);
   \draw[-latex] (h1) -- (x1b);

   \node[style=obsnode, scale=0.7, above left=0.3cm of h2] (x2a) {$x^a_2$};
   \node[style=obsnode, scale=0.7, below left=0.3cm of h2] (x2b) {$x^b_2$};
   \draw[-latex] (h2) -- (x2a);
   \draw[-latex] (h2) -- (x2b);

   \node[style=obsnode, scale=0.7, above right=0.3cm of h3] (x3a) {$x^a_3$};
   \node[style=obsnode, scale=0.7, below right=0.3cm of h3] (x3b) {$x^b_3$};
   \draw[-latex] (h3) -- (x3a);
   \draw[-latex] (h3) -- (x3b);
    
   \node[style=obsnode, scale=0.7, below left=0.3cm of  h4] (x4a) {$x^a_4$};
   \node[style=obsnode, scale=0.7, below right=0.3cm of h4] (x4b) {$x^b_4$};
   \draw[-latex] (h4) -- (x4a);
   \draw[-latex] (h4) -- (x4b);

\end{tikzpicture}

\end{document}

  \caption{A directed grid model.}
  \label{fig:grid}
\end{figure}

Let us illustrate our approach with an example. 
Consider the directed grid model shown in \figureref{grid} which
  captures the different dependency structures possible in a Bayesian network.
The model has eight observed variables $x^a_1, x^b_1 \cdots, x^a_4, x^b_4$ and four
  hidden variables $h_1, \cdots, h_4$.
The parameters of this model are the conditional probability tables
  $\pi, T, V$ and $O$; for the sake of simplicity, we assume that all
  the observation potentials are the same, i.e. $O \eqdef \Pr(x^\cdot_i
  | h_i)$ for all $i$, and that $\Pr(h_2 | h_1) = \Pr(h_3 | h_1) = T$.
Finally, we assume $O$ and $T$ have full column rank.

\paragraph{Estimating $O$}

Note that the observed variables $x^a_1, x^b_1, x^a_2$ are
  conditionally independent given $h_1$, allowing us to use
  \TensorFactorize as reviewed in \sectionref{setup}, to recover
  the moments $O$.
We can aggregate statistics from additional sets of independent observed
  variables, e.g. $\{x^a_2, x^b_2, x^a_3\}$, to arrive at a better
  estimate of $O$.

%Intuitively, we expect the observations to tell us about the underlying
%  hidden state, which should make learning each factor easier. 
%For example, when $O = I$ the observed variables
%  tell us exactly which hidden state we are in; the optimal procedure to
%  estimate factors is simply counting the co-occurrences of the hidden
%  states.

\paragraph{Estimating $\pi$}

We can relate the state of $h_1$ to the moments of $x^a_1$; $M_1
  \eqdef \Pr(x^a_1) = O \pi$. From \assumptionref{full-rank}, $O$ has full
  column rank and thus can be inverted to recover $\pi$; $\pi = \pinv O
  M_1$.

\paragraph{Estimating $T$}

Similarly, the we can write down the moments of $x^a_1, x^a_2$ in terms
of the pairwise marginals $P_{12} \eqdef \Pr(h_1, h_2)$ and solve for
$P_{12}$;
\begin{align*}
  M_{12} &= O P_{12} O^T \\
  P_{12} &= \pinv O M_{12} (\pinv O)^T.
\end{align*}

$T$ can be recovered from the $P_{12}$ by suitable normalization.

\paragraph{Estimating $V$}

Finally, we can estimate $V$ from the third-order moments $M_{234}$ as
follows,
\begin{align*}
  M_{234} &= P_{234}(O, O, O) \\
  P_{234} &= M_{234}(\pinv O, \pinv O, \pinv O).
\end{align*}

\subsection{General algorithm}

\renewcommand{\algorithmicrequire}{\textbf{Input:}}
\renewcommand{\algorithmicensure}{\textbf{Output:}}
\begin{algorithm}
  \caption{\LearnFactors}
  \label{algo:directed}
  \begin{algorithmic}
    \REQUIRE A graphical model $\sG$ satisfying \textbf{(P1)}, \textbf{(P2)}, data $\sD$
    \ENSURE Parameters $\theta$ for $\sG$

      \FOR{$h \in H$} 
        \STATE Apply \TensorFactorize for each bottleneck $h$ and learn parameters $\Pr(x \mid h) ~ \forall x \in \sB(h)$.

%    \COMMENT{Recover observation potentials $O$ using bottlenecks}
      \ENDFOR
%      \COMMENT{\textbf{Step 2:} Recover clique potentials from the piecewise likelihood.}
      \FOR{every clique $\sC = \{h_1, \cdots, h_m\} \in \sG$} 
      \STATE Let $x_\sC = \{x_1, \cdots, x_m\}$.
      \STATE $P_\sC = M_{x_\sC}( \pinv O_{x_1}, \cdots, \pinv O_{x_m} )$.
%      Run expectation-maximization to convergence on the piecewise likelihood \eqref{eqn:piecewise}, over data $\{\vec x_\sC : x \in \sD\}$
      \ENDFOR
  \end{algorithmic}
\end{algorithm}

\algorithmref{directed} presents the natural extension of this approach.
Let us now describe the general properties for which the algorithm,
  $\LearnFactors$ works.

For the first step of the algorithm, we wish to identify conditional
  moments between every hidden variable and atleast one observed
  variable.
We call hidden variables with this property ``bottlenecks'',
  and define them as follows.
\begin{definition}(Bottleneck)
  A hidden variable $h$ is said to be a bottleneck if there exists at
    least three conditionally independent observed variables $o_1, o_2,
    o_3$. 
  Let $\vec x_\sB(h)$ be the set of conditionally independent observed
    variables for $h$.
\end{definition}

We will further assume that each observed variable has a single parent. 
This leads to the following two properties.
\begin{property}
  \label{prop:unique-parent}
  Every observed variable $o$ has exactly one parent, $|\Pa(o)| = 1$.
\end{property}
\begin{property}
  \label{prop:bottleneck}
  Every hidden variable is a bottleneck.
\end{property}

We note that \propertyref{unique-parent} can be relaxed if another
  algorithm is used to recover the conditional moments. For example,
  \citet{halpern13noisyor} propose an algorithm that can recover the
  conditional moments for a broad, but less general class of noisy-or
  networks.
\propertyref{bottleneck} can also be relaxed if factors share the same
  parameters. 
For example, because we can identify the observation potential from just
  $x^a_1, x^b_1$ and $x^a_2$, we do not require that $h_2, h_3$ or $h_4$
  be bottlenecks, and can correspondingly omit the observations $x^b_2,
  x^b_3$ and $x^b_4$.

Given \propertyref{unique-parent} and \propertyref{bottleneck}, step
  1 of \algorithmref{directed} is guaranteed to recover potentials
  $O^{(h)}_{x}$ for every hidden variable $h$ and observed variable $x \in
  \vec x_\sB(h)$.
\propertyref{bottleneck} guarantees that for every clique $\sC
  = \{h_1, \cdots, h_m\}$ in the graphical model $\sG$, there exists at
  least one bottleneck observable that depends on exactly one $h_i \in
  \sC$; let $x_{h_i}$ be that observed variable\verify. 
Finally, define $\vec x_\sC = \{x_{h_1}, \cdots x_{h_m}\}$ to be the
  participating observed variables in the clique $\sC$.

Finally, we have the following moment condition,
\begin{align*}
  M_{x_\sC} &= P_{\sC}(O_{x_{h_1}}, \cdots, O_{x_{h_m}}) \\
  P_{\sC} &= M_{x_\sC}(\pinv O_{x_{h_1}}, \cdots, \pinv O_{x_{h_m}}).
\end{align*}

This algorithm is simple and intuitive. Next, we characterize its sample complexity.

\begin{lemma}(Sample complexity for $\sC$)
We require some number of samples to learn the factor for $\sC$.
\end{lemma}

This gives us our final result for a graphical model in our model
family.
\begin{theorem}(Learnability of Bottleneck Graphs)
The parameters for a directed graphical $\sG$ with tree-width $k$
satisfying \propertyref{unique-parent}, \propertyref{bottleneck} and
\assumptionref{full-rank} can be learned with $O(\cdot)$ samples.
\end{theorem}


\paragraph{Aggregating observations}

When we have multiple observed variables for a hidden variable, it is
  always beneficial to aggregate over them.
Note that $O_h$ will have rank atleast that of $O$; this encodes the
  intuitive fact that we might be able to identify $h$ from multiple
  observed variables even if we can not identify it from a single
  observed variable. 
Suppose $x_\sB(h) = \{x_1, x_2, \cdots, x_l\}$, let $O_h \in
  \Re^{D^l \times K}$ be an
  aggregated observation potential, such that $O_h(x_1, x_2, \cdots,
  x_l) = P(x_1|h) P(x_2|h) \cdots P(x_l|h)$.
Given \assumptionref{full-rank}, $O_h$ have rank $K$.
After step 1, $O_h$ is a known quantity.

\subsection{More examples}

We will now instantiate our algorithm for several examples.

\paragraph{Hidden Markov Model}

\paragraph{Latent Tree Structure}

\paragraph{Directed Grid Model}

