\section{Consistent Parameter Estimation in Directed Models}
\label{sec:directed}

The tensor factorization method attacks the heart of the non-convexity
  in latent variable models.
The technique is central in solving several other problems with more
  complex correlation structure, for example latent Dirichlet allocation
  \citep{anandkumar12lda}, mixture of Gaussians \citep{hsu13spherical},
  etc.\findcite{more things}.

Once we recover the conditional moments $O\oft{i,j} \eqdef \Pr(x_i
  | h_j)$ for some $i,j$, we present a systematic approach to learn the
  conditional probability tables for every clique for a class of directed
  graphical models.
We begin by illustrating our approach with an example, followed by
  describing the simple conditions for which the approach works and some
  theoretical properties.

\subsection{Example: Directed grid model}

\begin{figure}
  \centering
  % vim:ft=tex
\documentclass[tikz,convert={outfile=grid.pdf}]{standalone}
%\usetikzlibrary{...}% tikz package already loaded by 'tikz' option
\usepackage{scabby-diag}
\begin{document}

\begin{tikzpicture}

% Hidden nodes
   \node[style=node, scale=0.8] (h1) at (0,0) {$h_1$};
   \node[style=node, scale=0.8, below left= 0.5cm of h1] (h2) {$h_2$};
   \node[style=node, scale=0.8, below right= 0.5cm of h1] (h3) {$h_3$};
   \node[style=node, scale=0.8, below right= 0.5cm of h2] (h4) {$h_4$};

   \draw[-latex] (h1) -- (h2);
   \draw[-latex] (h1) -- (h3);
   \draw[-latex] (h2) -- (h4);
   \draw[-latex] (h3) -- (h4);

% Observed nodes
   \node[style=obsnode, scale=0.7, above left=0.3cm of h1] (x1a) {$x^a_1$};
   \node[style=obsnode, scale=0.7, above right=0.3cm of h1] (x1b) {$x^b_1$};
   \draw[-latex] (h1) -- (x1a);
   \draw[-latex] (h1) -- (x1b);

   \node[style=obsnode, scale=0.7, above left=0.3cm of h2] (x2a) {$x^a_2$};
   \node[style=obsnode, scale=0.7, below left=0.3cm of h2] (x2b) {$x^b_2$};
   \draw[-latex] (h2) -- (x2a);
   \draw[-latex] (h2) -- (x2b);

   \node[style=obsnode, scale=0.7, above right=0.3cm of h3] (x3a) {$x^a_3$};
   \node[style=obsnode, scale=0.7, below right=0.3cm of h3] (x3b) {$x^b_3$};
   \draw[-latex] (h3) -- (x3a);
   \draw[-latex] (h3) -- (x3b);
    
   \node[style=obsnode, scale=0.7, below left=0.3cm of  h4] (x4a) {$x^a_4$};
   \node[style=obsnode, scale=0.7, below right=0.3cm of h4] (x4b) {$x^b_4$};
   \draw[-latex] (h4) -- (x4a);
   \draw[-latex] (h4) -- (x4b);

\end{tikzpicture}

\end{document}

  \caption{A directed grid model.}
  \label{fig:grid}
\end{figure}

Consider the directed grid model shown in \figureref{grid} which
  captures all the different dependency structures possible in
  a Bayesian network and is hence a good example for intuition\reword.
The model has eight observed variables $x^a_1, x^b_1 \cdots, x^a_4, x^b_4$ and four
  hidden variables $h_1, \cdots, h_4$.
The parameters of this model are the conditional probability tables
  $\pi \eqdef \Pr(h_1), T \eqdef \Pr(h_2 | h_1) = \Pr(h_3 | h_1),
  V \eqdef \Pr(h_4 | h_2, h_3)$ and $O \eqdef \Pr(x^a_i | h_i)
  =  \Pr(x^b_i | h_i)$. 
Finally, we assume $O$ and $T$ have full column rank.

\paragraph{Estimating $~O$}

Note that the observed variables $x^a_1, x^b_1, x^a_2$ are
  conditionally independent given $h_1$, allowing us to use
  \TensorFactorize, as reviewed in \sectionref{setup}, to recover
  the $O$.
We can exploit parameter sharing between the different $x$ to aggregate
  statistics from additional sets of independent
  observed variables, or {\em views}, e.g. $\{x^a_2, x^b_2, x^a_3\}$, to
  arrive at a better estimate of $O$.

%Intuitively, we expect the observations to tell us about the underlying
%  hidden state, which should make learning each factor easier. 
%For example, when $O = I$ the observed variables
%  tell us exactly which hidden state we are in; the optimal procedure to
%  estimate factors is simply counting the co-occurrences of the hidden
%  states.

\paragraph{Estimating $~\pi$}

The moments of $x^a_1$,$\mO_1 \eqdef \Pr(x^a_1)$ are directly related to
  $\pi$; $\mO_1 = O \pi$. 
By \assumptionref{full-rank}, $O$ has full column rank and thus can be
  inverted to recover $\pi$; $\pi = \pinv O \mO_1$.

\paragraph{Estimating $~T$}

Similarly, we can write down the moments of $x^a_1, x^a_2$, $\mO_{12}
  \eqdef \Pr(x^a_1, x^a_2)$ in terms of the pairwise marginals $\mH_{12}
  \eqdef \Pr(h_1, h_2)$ and solve for $\mH_{12}$;
\begin{align*}
  \mO_{12} &= O \mH_{12} O^T \\
  \mH_{12} &= \pinv O \mO_{12} \pinvt O.
\end{align*}

$T$ can be recovered from the $\mH_{12}$ by suitable normalization.

\paragraph{Estimating $V$}

Finally, we can estimate $V$ from the third-order moments $\mO_{234} \eqdef \Pr(x^a_2, x^a_3, x^a_4)$ as
follows,
\begin{align*}
  \mO_{234} &= \mH_{234}(O, O, O) \\
  \mH_{234} &= \mO_{234}(\pinv O, \pinv O, \pinv O).
\end{align*}

\subsection{General algorithm}

Fundamentally, our algorithm takes advantage of having a view for every
  hidden variable in a clique to help identify its hidden state. The
  ideal property we wish our graphical models to possess is the
  following:
\begin{property}(Exclusive views)
  \label{prop:exclusive-views}
For every clique $\sC$ of $\sG$ and every hidden variable $h \in \sC$,
  there is some observed variable, $x_{v_h}$, which is conditionally
  independent of the rest of the clique given $h$ and whose conditional
  moment $\mOpp{v_h}{h} \eqdef \Pr(x_{v_h} | h)$ is known. 
We call $x_{v_h}$ the exclusive view of $h$ in $\sC$.
\end{property}

Before we identify a family of models which entail this property, let us
  show that it is sufficient to learn the parameters of $\sG$.

Consider any clique $\sC = \{h_{i_1}, \cdots, h_{i_m}\} \in \sG$. Let $x_{v_j}$ be the exclusive
  view for $h_{i_j}$ in $\sC$ and $\sV = \{x_{v_1}, \cdots, x_{v_m}\}$
  be the set of views for the clique $\sC$.
We can write down the moments $\mO_\sV$,
\begin{align*}
  \mO_\sV 
  &\eqdef \Pr(\Sx{\sV}) \\
  &= \sum_{h_{i_1} = 1}^{d} \cdots \sum_{h_{i_1} = 1}^{d} 
      \Pr(\Sh{\sC}) \Pr(\Sx{\sV} \given \Sh{\sC}) \\
      &= \sum_{h_{i_1} = 1}^{d} \cdots \sum_{h_{i_1} = 1}^{d} \Pr(\Sh{\sC}) 
          \Pr(x_{i_1} | h_{v_1}) \cdots \Pr(x_{i_m} | h_{v_m}) \\
    &= Z_{\sC}(\mOpp{v_1}{i_1},\cdots,\mOpp{v_m}{i_m}).
\end{align*}

If each $\mOpp{v_j}{i_j}$ has full column rank, then we can recover the
hidden marginals $Z_\sC$ by inverting the conditional moments,
\begin{align*}
  Z_{\sC} &= \mO_\sV(\pinv{\mOpp{v_1}{i_1}},\cdots,\pinv{\mOpp{v_m}{i_m}}).
\end{align*}
Finally, the conditional probability tables for $\sC$ can easily be gotten from
  $Z_\sC$ via conditioning.

For each $\mOpp{v_j}{i_j}$ to have full column rank, it is equivalent
  to the following generalized assumption of
  \assumptionref{full-rank},

\begin{assumption} 
  \label{asm:full-rank-plus}
  For every clique $\sC = \{h_{i_1}, \cdots, h_{i_m}\} \in \sG$, the
  marginal distribution $Z_\sC$ derived by the parameters is full rank
  for every mode-unfolding of $Z_\sC$.
\end{assumption}

Intuitively, this assumption says that the marginal probability of every state for any hidden variable is non-zero.

\begin{lemma}
  \label{lem:goodness}

  Given \assumptionref{full-rank-plus}, (a) the marginal distribution
  $\Pr(h_i) \succ 0$ for every $h_i$ in the graph and (b)
  $\mOpp{v_j}{i_j}$ has full column rank.
\end{lemma}
\begin{proof}
  By variable elimination.
\end{proof}

\todo{Talk about aligning views}

\paragraph{Required structural properties}

We will now provide a sufficient condition for
\propertyref{exclusive-views} to hold. 
Recall that given three observed
variables, $x_1, x_2, x_3$ that are independent conditioned on $h$,
\TensorFactorize allows us to recover the conditional moments $\mOp{1},
\mOp{2}, \mOp{3}$. We define a hidden variable that has this capability to be a bottleneck.
\begin{definition}(Bottleneck)
  A hidden variable $h$ is said to be a bottleneck if there exists at
    least three conditionally independent observed variables, or views,
    $x_1, x_2, x_3$. 
  Let $\xB{h}$ be the set of views for $h$.
\end{definition}

The only property a graphical models must have is the following,
\begin{property}
  \label{prop:bottleneck}
  Every hidden variable is a bottleneck.
\end{property}

We note that \propertyref{bottleneck} can be relaxed if some cliques
  share parameters;
For example, because we can identify the observation potential from just
  $x^a_1, x^b_1$ and $x^a_2$, we do not require that $h_2, h_3$ or $h_4$
  be bottlenecks, and can correspondingly omit the observations $x^b_2,
  x^b_3$ and $x^b_4$.
In this case, we only require that the hidden variables in distinct
  cliques share parameters.

It is not immediately obvious that \propertyref{bottleneck} implies
  \propertyref{exclusive-views}. 
Another consideration is that while
  \TensorFactorize gives us the conditional moments for every
  bottleneck, it can not recover the conditional probability table when
  there are multiple parents for an observed variable.
The following lemma shows us how to handle this case.

\begin{lemma}(Canonical form) 
  Every directed graphical model can be transformed into one in which
  the observed variables are leaves with exactly one parent.
\end{lemma}
\begin{proof}
  \begin{figure}
    \label{fig:reduction}
    \centering
    \subimport{figures/}{reduction.tikz}
    \caption{Reduction to canonical form.}
  \end{figure}

  Let $x$ be an observed variable with parents $h_1, \cdots, h_p$ and
  children $h_{p+1}, \cdots, h_c$. Replace $x$ with a phantom hidden
  variable $h_x$, and make $x$ the child of $h_x$ (\figureref{reduction}). The domain of $h_x$
  has the same as $x$, and $\mOpp{x}{h_x} = I$.

  The parameters for $h_x$ are the same as $x$.
\end{proof}

Applying \TensorFactorize to every bottleneck gives us candidate views
  for every hidden variable. Now we show that every clique also has
  exclusive views.

\begin{lemma}(Bottlenecks guarantee exclusive views)
  Provided \propertyref{bottleneck} holds, every hidden variable
    $h_{i_1}, \cdots, h_{i_m}$ in a clique $\sC$ has a exclusive view.
\end{lemma}
\begin{proof}
  Let $h$ be a hidden variable in $\sC$, let $\sV_{h}$ be the set of
    views for it and let $\sC^- \eqdef \sC \setminus \{h\}$.
  By definition, the views of $h$ are conditionally independent, viz.
    every d-connecting path between two views $x_1, x_2 \in \sV_h$ must pass through
    $h$.
  A view $x \in \sV_{h}$ is exclusive for $h$ iff $x \not\in \sV_{h'} ~ \forall h'
  \in \sC^-$

  Suppose $h$ had no unique view, then $\sV_{h} \subseteq \Union_{h' \in \sC^-} \sV_{h'}$. 
  However, this would imply that for every $x_1, x_2 \in \sV_{h}$, there
    exists some $h_1, h_2 \in \sC^-$ (not necessarily distinct) such that
    $x_1$ is d-connected to $h_1$ and $x_2$ is d-connected $h_2$.
  By participation in the clique, either $h_1 = h_2$ or $h_1$ is
    d-connected to $h_2$, implying that $x_1$ and $x_2$ are d-connected
    via a path that does not pass through $h$, contradicting the
    statement that $x_1, x_2 \in \sV$. 
    
  Thus it must be the case that $\sV_{h} \subsetneq \Union_{h' \neq
    h \in \sC} \sV_{h'}$ and hence there exists a unique view for $h$.

  Algorithmically, finding the unique views can be done by subtracting
    out the remaining set, i.e. $x_{v_h} \in \sV_h \setminus \Union_{h'
    \in \sC} \sV_{h'}$
\end{proof}

\paragraph{Aligning the views}
\todo{How do we make sure all the recovered $O$ are align? Shouldn't be Hard}

With this lemma in place, we present the full algorithm, \LearnMarginals
in \algorithmref{directed}.

\renewcommand{\algorithmicrequire}{\textbf{Input:}}
\renewcommand{\algorithmicensure}{\textbf{Output:}}
\begin{algorithm}
  \caption{\LearnMarginals}
  \label{algo:directed}
  \begin{algorithmic}
    \REQUIRE A graphical model $\sG$ satisfying \textbf{(P1)}, \textbf{(P2)}, data $\sD$
    \ENSURE Marginals $Z_\sC$ for every clique $\sC \in \sG$

      \FOR{$h_i \in H$} 
        \STATE Apply \TensorFactorize to learn conditional moments
        $\mOpp{v}{i}$ for every $x_v \in \xB h$.

%    \COMMENT{Recover observation potentials $O$ using bottlenecks}
      \ENDFOR
%      \COMMENT{\textbf{Step 2:} Recover clique potentials from the piecewise likelihood.}
\FOR{every clique $\sC = \{h_{i_1}, \cdots, h_{i_m}\} \in \sG$} 
\STATE Identify unique views $\Sx{\sV} = \{x_{v_1}, \cdots, x_{v_m}\}$.
\STATE $\mH_\sC \gets \mO_{\Sx{\sV}}( \pinv{\mOpp{v_1}{i_i}}, \cdots, \pinv{\mOpp{v_m}{i_m}} )$.
      \ENDFOR
  \end{algorithmic}
\end{algorithm}

\paragraph{Theoretical properties}

As each step of \LearnMarginals is a consistent estimator, it is
  straightforward that as whole it is a consistent estimator for $\sG$.
We describe its asymptotic complexity to see what the dependence on the
  parameters of the problem are.

\begin{lemma}(Sample complexity for $\sC$)
  \label{lem:mom-complexity}  
We require some number of samples to learn the factor for $\sC$.
\end{lemma}
\begin{proof}

For the sake of simplicity, let us assume that our dataset of $n$
  samples is partitioned into two sets.
  The first $\bytwo{n}$ samples are used to estimate $\mOpphat{v}{i}$ for
  every $h_i \in H$ and $x_v \in \sV_{h}$ in step 1 of \algorithmref{directed}.
\citet{anandkumar13tensor} show that 
\begin{align*}
  \|\mOpphat{v}{i} - \mOpp{v}{i}\|_F &= O( \frac{k \pi_{\max}^2}{\sigma_k(M_2)^5} ). 
\end{align*}

The next $\bytwo{n}$ samples are used to compute the moments $M_\sV$ for
  every clique in step 2.

We have for each clique,
\begin{align*}
  Z_{\sC} &= \mO_\sV(\pinv{\mOpp{v_1}{i_1}},\cdots,\pinv{\mOpp{v_m}{i_m}}) \\
\end{align*}

Choosing to represent $Z_\sC$ as a vector and $\mOpp{\cdot}{\cdot}$ as a matrix,
let $\pinv{\mOpp{\sV}{\sC}} \eqdef \pinv{\mOpp{v_1}{i_1}} \otimes \cdots \otimes \pinv{\mOpp{v_m}{i_m}}$,
\begin{align*}
  Z_{\sC} &= \pinv{\mOpp{\sV}{\sC}} \mO_\sV.
\end{align*}

We know the asymptotic distribution of $M_\sV$ goes as,
\begin{align*}
  \sqrt{n}(\hat M_\sV - M_\sV) \convind \sN(0, \Sigma_\sV),
\end{align*}
where $\Sigma_\sV$ is the variance. Because $M_\sV$ describes the discrete marginal distribution, we have 
\begin{align*}
  \Sigma_\sV &= \diag  M_\sV  - M_\sV M_\sV^T,
\end{align*}

Thus, $Z_\sC$ goes as
\begin{align*}
  \sqrt{n}(\hat Z_\sC - Z_\sC) \convind \sN(0, \pinvt{\mOpp{\sV}{\sC}} \Sigma_\sV \pinv{\mOpp{\sV}{\sC}}).
\end{align*}

\todo{Convert from $Z_\sC$ to the actual CPTs}.

\end{proof}

