% DONE: too wordy
%\section{Consistent Parameter Estimation in Directed Models}
\section{Directed models and exclusive views}
\label{sec:directed}

%\todo{we are using clique to refer to only hidden cliques; be explicit about using "hidden"}

% DONE: need more setup of what the goal is
%The tensor factorization method attacks the heart of the non-convexity
  %in latent variable models.
  In this section, we will develop a consistent parameter estimate for a class of directed graphical models.
  In general directed models, the conditional moments $\mOpp{v}{i} \eqdef \Pr(x_v \mid h_i)$
  do not correspond directly to the underlying parameters.
  %Once we recover the conditional moments $\mOpp{v}{j} \eqdef \Pr(x_v
  %| h_j)$ for some $v, j$, we present a systematic approach to learn the
  %conditional probability tables for every clique for a class of directed
  %graphical models.
  Let us see how we can estimate the parameters with an example (Section~\ref{sec:directedExample}).
  Section~\ref{sec:directedGeneral} will describe the algorithm in full generality.
For ease of exposition we make the following simplifications to our presentation:
(i) we describe our algorithm entirely in the context of directed
  graphical models, though it generalizes to factor graphs;
(ii) we derive our algorithm assuming infinite data; in \sectionref{sampleComplexity} we 
  describe error bounds for the finite sample regime; and
(iii) we present the algorithm solely in terms of linear algebra operations; in
  \sectionref{piecewise}, we show how we can get a more statistically
  efficient estimator using composite likelihood.

\subsection{Example: Directed grid model}
\label{sec:directedExample}

%\begin{figure}
%  \centering
%  % vim:ft=tex
\documentclass[tikz,convert={outfile=gridoutline.pdf}]{standalone}
%\usetikzlibrary{...}% tikz package already loaded by 'tikz' option
\usepackage{scabby-diag}

\begin{document}

\begin{tikzpicture}
%\draw[step=1.0,black,thin] (-3,-3) grid (2,2);

% Hidden nodes
   \node[style=node, scale=0.8] (h1) at (0,0) {$h_1$};
   \node[style=node, scale=0.8, below left= 0.5cm of h1] (h2) {$h_2$};
   \node[style=node, scale=0.8, below right= 0.5cm of h1] (h3) {$h_3$};
   \node[style=node, scale=0.8, below right= 0.5cm of h2] (h4) {$h_4$};

   \draw[-latex] (h1) -- (h2);
   \draw[-latex] (h1) -- (h3);
   \draw[-latex] (h2) -- (h4);
   \draw[-latex] (h3) -- (h4);

% Observed nodes
   \node[style=obsnode, scale=0.7, above left=0.3cm of h1] (x1a) {$x^a_1$};
   \node[style=obsnode, scale=0.7, above right=0.3cm of h1] (x1b) {$x^b_1$};
   \draw[-latex] (h1) -- (x1a);
   \draw[-latex] (h1) -- (x1b);

   \node[style=obsnode, scale=0.7, above left=0.3cm of h2] (x2a) {$x^a_2$};
   \node[style=obsnode, scale=0.7, below left=0.3cm of h2] (x2b) {$x^b_2$};
   \draw[-latex] (h2) -- (x2a);
   \draw[-latex] (h2) -- (x2b);

   \node[style=obsnode, scale=0.7, above right=0.3cm of h3] (x3a) {$x^a_3$};
   \node[style=obsnode, scale=0.7, below right=0.3cm of h3] (x3b) {$x^b_3$};
   \draw[-latex] (h3) -- (x3a);
   \draw[-latex] (h3) -- (x3b);
    
   \node[style=obsnode, scale=0.7, below left=0.3cm of  h4] (x4a) {$x^a_4$};
   \node[style=obsnode, scale=0.7, below right=0.3cm of h4] (x4b) {$x^b_4$};
   \draw[-latex] (h4) -- (x4a);
   \draw[-latex] (h4) -- (x4b);

% Draw outline   
\begin{pgfonlayer}{background}
\draw[rounded corners,line width=1pt, dotted, black] 
                ($(x1b.north east) + (45:0.3cm)$) -- 
                ($(x1a.north west) + (135:0.3cm)$) -- 
                ($(x2a.north west) + (135:0.3cm)$) -- 
                ($(x2a.south west) + (-135:0.3cm)$) -- 
                ($(h1.south) + (-90:0.3cm)$) -- 
                ($(x1b.east) + (0:0.3cm)$) -- 
                cycle;
\end{pgfonlayer}

\end{tikzpicture}

\end{document}

%  \caption{A directed grid model.}
%  \label{fig:grid}
%\end{figure}

%Consider the directed grid model shown in \figureref{grid} which
Consider the directed grid model shown in \figureref{approach}(a).
  % DONE: all the different dependency => too strong
  %captures the local dependency structures possible in
  %a Bayesian network.
The model has eight observed variables $x^a_1, x^b_1 \cdots, x^a_4, x^b_4$ and four
  hidden variables $h_1, \cdots, h_4$.
The parameters of this model are the conditional probability tables
$\pi \eqdef \Pr(h_1) \in \Re^k, T \eqdef \Pr(h_2 | h_1) = \Pr(h_3 | h_1) \in \Re^{k \times k},
V \eqdef \Pr(h_4 | h_2, h_3) \in \Re^{k \times k \times k}$ and $O \eqdef \Pr(x^a_i | h_i)
=  \Pr(x^b_i | h_i) \in \Re^{d \times k}$. 
Assume $O$ and $T$ have full column rank.

\paragraph{Estimating $O$}
Note that the observed variables $x^a_1, x^b_1, x^a_2$ are
  conditionally independent given $h_1$; we can use
  $\TensorFactorize$ (\sectionref{setup}) to recover $O$.

\paragraph{Estimating $\pi$}
The moments of $x^a_1$, $\mO_1 \eqdef \Pr(x^a_1)$ are directly related to
  $\pi$ by $\mO_1 = O \pi$. 
By \assumptionref{full-rank}, $O$ has full column rank and thus can be
  inverted to recover $\pi$: $\pi = \pinv O \mO_1$.

\paragraph{Estimating $T$}
Similarly, we can write down the moments of $x^a_1, x^a_2$, $\mO_{12}
  \eqdef \Pr(x^a_1, x^a_2)$, in terms of the hidden marginals $\mH_{12}
  \eqdef \Pr(h_1, h_2)$ and solve for $\mH_{12}$:
\begin{align*}
\mO_{12} = O \mH_{12} O^\top \quad\Rightarrow\quad
  \mH_{12} = \pinv O \mO_{12} \pinvt O.
\end{align*}
$T$ can be recovered from the $\mH_{12}$ by suitable renormalization.

\paragraph{Estimating $V$}
Finally, we can estimate $V$ by renormalizing the hidden marginals
$\mH_{234} \eqdef \Pr(h_2, h_3, h_4)$ from the third-order moments
$\mO_{234} \eqdef \Pr(x^a_2, x^a_3, x^a_4)$:
\begin{align*}
  \mO_{234} = \mH_{234}(O, O, O) \quad\Rightarrow\quad
  \mH_{234} = \mO_{234}(\pinv O, \pinv O, \pinv O).
\end{align*}

%%%%%%%%%%%%%%%%%%%%%%%%%%%%%%%%%%%%%%%%%%%%%%%%%%%%%%%%%%%%
\subsection{General algorithm}
\label{sec:directedGeneral}

Note that in each of the above cases,
  for any clique $\sC$, each hidden variable $h_i \in \sC$ has its own
  view which is conditionally independent given the other hidden variables in
  the clique.
%Fundamentally, algorithm takes advantage of having a view for every
  %hidden variable in a clique to help identify its hidden state.
We capture this general property as follows:
\begin{property}(Exclusive views)
  \label{prop:exclusive-views}
A hidden sub-graph $\sG' \subseteq \sG$ (typically a clique $\sC \in
  \sG$) is said to possess the \textbf{exclusive views property} if for
  every hidden variable $h_i \in \sG'$, there is some observed variable,
  $x_{v}$, which is conditionally independent of the rest of the clique
  given $h_i$, and whose conditional moment $\mOpp{v}{i} \eqdef
  \Pr(x_{v} \mid h_i)$ can be estimated consistently.
We call $x_v$ the exclusive view of $h_i$ in $\sG'$.
\end{property}

For the rest of this section, we will only consider sub-graphs which are
  cliques. 
In \sectionref{undirected}, we will show how the generalization
  to sub-graphs allows us to efficiently estimate the pseudo-likelihood.

% \begin{property}(Exclusive views)
%   \label{prop:exclusive-views}
%   A hidden clique $\sC$ of $\sG$ is said to possess the \textbf{exclusive views
%   property} if for every hidden variable $h_i \in \sC$,
%   there is some observed variable, $x_{v}$, which is conditionally
%   independent of the rest of the clique given $h_i$,
%   and whose conditional
%   moment $\mOpp{v}{i} \eqdef \Pr(x_{v} \mid h_i)$ can be estimated consistently.
% We call $x_v$ the exclusive view of $h_i$ in $\sC$.
% \end{property}

%%%%%%%%%%%%%%%%%%%%%%%%%%%%%%
\paragraph{Estimating hidden clique marginals}

We now show this condition is sufficient to learn the marginal
  distribution of the clique.
Consider any hidden clique $\sC = \{h_{i_1}, \cdots, h_{i_m}\} \in \sG$ for which this property holds. Let
  $x_{v_j}$ be any exclusive view for $h_{i_j}$ in $\sC$ and $\sV
  = \{x_{v_1}, \cdots, x_{v_m}\}$ be a set of views for the clique $\sC$.
We can write down the moments $\mO_\sV$ as follows:
\begin{align*}
  \mO_\sV 
  &\eqdef \Pr(\Sx{\sV}) \\
  %&= \sum_{\vh \in \sH}
      %\Pr(\Sh{\sC}) \Pr(\Sx{\sV} \given \Sh{\sC}) \\
      &= \sum_{\vh \in \sH} \Pr(\Sh{\sC}) 
          \Pr(x_{i_1} | h_{v_1}) \cdots \Pr(x_{i_m} | h_{v_m}) \\
    &= Z_{\sC}(\mOpp{v_1}{i_1},\cdots,\mOpp{v_m}{i_m}).
\end{align*}

If each $\mOpp{v_j}{i_j}$ has full column rank, then we can recover the
hidden marginals $Z_\sC$ by inverting:
\begin{align*}
  Z_{\sC} &= \mO_\sV(\pinv{\mOpp{v_1}{i_1}},\cdots,\pinv{\mOpp{v_m}{i_m}}).
\end{align*}
Finally, the conditional probability tables for $\sC$ can easily be gotten from
  $Z_\sC$ via renormalization.
\algorithmref{learnclique} summarizes the procedure, \LearnClique.

\begin{algorithm}
  \caption{\LearnClique~(pseudoinverse)}
  \label{algo:learnclique}
  \begin{algorithmic}
    \REQUIRE Hidden clique $\sC$ with exclusive views (\propertyref{exclusive-views}).
    \ENSURE Marginal distribution of the clique $Z_\sC$.
      \STATE Identify exclusive views $\Sx{\sV} = \{x_{v_1}, \cdots, x_{v_m}\}$.
      \STATE Return $\mH_\sC \gets \mO_{\Sx{\sV}}( \pinv{\mOpp{v_1}{i_i}}, \cdots, \pinv{\mOpp{v_m}{i_m}} )$.
  \end{algorithmic}
\end{algorithm}
Of course, for $Z_{\sC}$ to be identifiable, we need various rank conditions,
which will be established in \sectionref{sampleComplexity}.

%%%%%%%%%%%%%%%%%%%%%%%%%%%%%%
\paragraph{Structural properties}

% DONE: transition the content more gradually (checkpoint: what have we done so far, what next)
% DONE: refer to properties by name in addition to by number
We have shown how to estimate the parameters of any clique that possesses the exclusive
  views property (\propertyref{exclusive-views}). But in a general graph, which cliques have this property?
In the example above, we were able to guarantee \propertyref{exclusive-views}
  by identifying a hidden variable, $h_1$ along with three conditionally
  independent observed variables, $x^a_1, x^b_1, x^a_2$.
  %followed by using \TensorFactorize to learn the conditional moments.
We define such a hidden variable to be a bottleneck:
\begin{definition}[Bottleneck]
  A hidden variable $h$ is said to be a \textbf{bottleneck} if there exists at
    least three conditionally independent observed variables, or views,
    $x_1, x_2, x_3$. 
  Let $\sV_{h}$ denote a set of views for $h$ (not necessarily unique).
\end{definition}

% DONE: don't talk about parameter recovery yet, just establish exclusive views
%Finally, the claim we make is that we can recover parameters for any
  %graphical model where every hidden variable is a bottleneck:
We will prove that the following simple property guarantees that all cliques
have the exclusive views property:
\begin{property}[Uniformly bottlenecked]
  \label{prop:bottleneck}
  Every hidden variable is a bottleneck.
\end{property}

%Applying \TensorFactorize to every bottleneck gives us a set of views
%  for every hidden variable. 
%To proceed, we need to check that \
%%(i) understand the assumptions under which
%  %\assumptionref{full-rank} holds for each bottleneck, and (ii)
%  that the candidate views actually imply that \propertyref{exclusive-views}
%  holds for every clique in the graph.

Before we establish this relationship,
we describe a reduction of any graphical model to
  a canonical form to make our arguments simpler.
  % DONE: don't say this yet since reader can't appreciate it.
  %that will allow us to handle the case where an observed
  %variable has multiple parents.

\begin{lemma}[Canonical form]
  \label{lem:reduction}
Every directed (undirected) graphical model can be transformed into one in which
  the observed variables are leaves with exactly one parent (neighbour). 
There is a one-to-one correspondence between the parameters of this
  transformed model and the original one.
\end{lemma}
\begin{proof}
  \begin{figure}
    \centering
    \subimport{figures/}{reduction.tikz}
    \caption{Reduction to canonical form.}
    \label{fig:reduction}
  \end{figure}

  \providecommand{\hp}{\ensuremath{h_\text{\rm new}}}

  Let $x_v$ be an observed variable with parents $\Pa(x_v)$ and children $\text{Ch}(x_v)$.
  Consider the following transformation.
  Replace $x_v$ with a new hidden variable \hp\ with the same
  parents $\Pa(x_v)$ and children $\text{Ch}(x_v) \union \{x_{v_1}, x_{v_2}, x_{v_3}\}$,
  where $x_{v_1},x_{v_2},x_{v_3}$ are three copies of $x$
  (\figureref{reduction}). 
  The domain of $\hp$ is the same as $x$,
    and $\mOpp{v}{\textrm{new}} = I$.\footnote{
      Though we have assumed that all hidden variables share the
      same domain, $[k]$, our work generalizes to hidden domains of
      different sizes, provided \assumptionref{full-rank-plus} is satisfied.
      }

  Then, there is a one-to-one correspondence between every value of
  $\hp$ and $x_v$. Consequently, for any clique $\sC \contains \hp$, the
  parameters in the original graphical model can be obtained by
  substituting $\hp$ with $x_v$.

  This procedure applies straightforwardly for undirected graphical
  models, considering the neighbors $\sN(x_v)$ instead of its parents
  and children.
\end{proof}

\begin{lemma}[Bottlenecks guarantee exclusive views]
  \label{lem:bottleneck-views}  
Let $\sG' = \{ h_{i_1}, \cdots, h_{i_m} \} \subseteq \sG$ be
  a connected sub-graph of hidden variables. If every hidden variable
  $h_i \in \sG'$ is a bottleneck, then they each have an exclusive view.
\end{lemma}
\begin{proof}
  By \lemmaref{reduction}, we can assume that every observed
  variable is a leaf in the graph, causing any v-structures in the graph
  to be active. This allows us to reason about independence simply by
  checking whether or not two nodes are connected or not. Note that this
  proof applies for undirected graphs as well.

  Let $h$ be a hidden variable in $\sG'$, let $\sV_{h}$ be the set of
  views for it and let ${\sG'}^- \eqdef \sG' \setminus \{h\}$.
  By definition, the views of $h$ are conditionally independent, viz.
    every connecting path between two views $x_1, x_2 \in \sV_h$ must pass through
    $h$.
  A view $x \in \sV_{h}$ is exclusive for $h$ iff $x \not\in \sV_{h'} ~ \forall h'
  \in {\sG'}^-$.

  Suppose $h$ had no unique view, then $\sV_{h} \subseteq \Union_{h' \in {\sG'}^-} \sV_{h'}$. 
  However, this would imply that for every $x_1, x_2 \in \sV_{h}$, there
  exists some $h_1, h_2 \in {\sG'}^-$ (not necessarily distinct) such that
    $x_1$ is connected to $h_1$ and $x_2$ is connected $h_2$.
  By participation in the sub-graph, either $h_1 = h_2$ or $h_1$ is
    connected to $h_2$, implying that $x_1$ and $x_2$ are connected
    via a path that does not pass through $h$, contradicting the
    statement that $x_1, x_2 \in \sV_h$. 
    
  Thus, it must be the case that $\sV_{h} \subsetneq \Union_{h' \neq
    h \in \sC} \sV_{h'}$ and hence there exists a unique view for $h$.

  Algorithmically, finding the unique view for $h$ can be done by
  subtracting the views of all other variables in the sub-graph from
  $\sV_h$,i.e. $x_{v_h} \in \sV_h \setminus \Union_{h' \in \sC} \sV_{h'}$
\end{proof}

With this lemma in place, we present the full algorithm, $\LearnMarginals$,
in \algorithmref{directed}.

\renewcommand{\algorithmicrequire}{\textbf{Input:}}
\renewcommand{\algorithmicensure}{\textbf{Output:}}
\begin{algorithm}
  \caption{\LearnMarginals}
  \label{algo:directed}
  \begin{algorithmic}
    \REQUIRE Graphical model $\sG$ satisfying \propertyref{bottleneck}, data $\sD$
    \ENSURE Marginals $Z_\sC$ for every clique $\sC \in \sG$

      \FOR{each hidden variable $h_i \in H$} 
        \STATE Apply $\TensorFactorize$ to learn conditional moments
        $\mOpp{v}{i}$ for every $\sV_{h_i}$.

%    \COMMENT{Recover observation potentials $O$ using bottlenecks}
      \ENDFOR
%      \COMMENT{\textbf{Step 2:} Recover clique potentials from the piecewise likelihood.}
\FOR{every clique $\sC = \{h_{i_1}, \cdots, h_{i_m}\} \in \sG$} 
\STATE Apply $\LearnClique$ to learn the marginals $\mH_\sC$.
\ENDFOR
  \end{algorithmic}
\end{algorithm}

% DONE: move here because not central to story
\paragraph{Remark.} We note that \propertyref{bottleneck} can be relaxed if some cliques
  share parameters.
In the directed grid model, we can identify the conditional moments $O$ from just
  $x^a_1, x^b_1$ and $x^a_2$.
  Therefore, we do not require that $h_2, h_3$ or $h_4$
  be bottlenecks, and can omit the observations $x^b_2, x^b_3$ and $x^b_4$.
%  \todo{make this a footnote if running out of space}
% PL: don't understand this comment
%In this case, we only require that the hidden variables in distinct
  %cliques share parameters.

\begin{figure}
  \centering
  \subfigure[Hidden Markov model] {
    \includegraphics[width=0.45\columnwidth]{figures/hmm.pdf}
    \label{fig:examples-hmm}
  }
%  \subfigure[Directed grid model] {
%    \label{fig:examples-grid}
%    \includegraphics{figures/grid.pdf}
%  }
  \subfigure[Tree model] {
    \includegraphics[width=0.45\columnwidth]{figures/tree.pdf}
    \label{fig:examples-tree}
  }
  \subfigure[Noisy-or model] {
    \includegraphics[height=5em]{figures/non-example.pdf}
    \label{fig:examples-noisy-or}
  }
  \caption{(a) and (b): graphs that satisfy the exclusive views property; (c) does not.}
  \label{fig:examples}
\end{figure}


\paragraph{Example: hidden Markov model.}

In the HMM (\figureref{examples-hmm}), the parameters
are $O \eqdef \Pr(x_i|h_i)$  and that $T \eqdef \Pr(h_{i+1} | h_i)$
for all $i$. % (i.e. we have parameter sharing).
While the first and last hidden variables $h_1, h_M$ in the
  sequence are not bottlenecks, they still have exclusive views ($x_1$ and
  $x_M$ respectively) whose parameters we know because they share
  parameters, $O$.
%We first use the bottleneck $h_2$ with views $x_1, x_2, x_3$ to estimate $O$,
%and then recover $\pi$ from $\{h_1\}$ and recover $T$ from the clique $\{h_{1}, h_{2}\}$.

\paragraph{Example: latent tree model.}

In the latent tree model (\figureref{examples-tree}), the parameters
are $\Pr(h_i) = \pi$, $\Pr(h_i | h_1) = T$ for $i \in \{2,3,4\}$
and $\Pr(x^a_i | h_i) = \Pr(x^b_i | h_i) = O$ for $i \in \{2,3,4\}$.
Note that while $h_1$ is not directly connected to an observed variable,
  it is still a bottleneck, with views $x^a_2, x^a_3, x^a_4$.
We can recover $T$ from the clique $\{h_1, h_2\}$ by using views $x^a_2$
  (exclusive to $h_2$) and $x^a_3$ (exclusive to $h_3$).
  %Note that while
  %$x^a_3$ is also a view for $h_2$, $x^a_3$ is independent of $h_2$ given
  %$h_1$.

%In step 1, we recover the parameters $O$ from the bottleneck $h_2$ with
%  views $\{x^a_2, x^b_2, x^a_3\}$. We also recover the conditional moments
%  $\mOpp{2}{1}$, $\mOpp{3}{1}$, $\mOpp{4}{1}$ for $h_1$. 
%In step 2, we can recover $\pi$ from the clique $\{h_1\}$, using any
%  one of views (they are all exclusive). 
%To recover $T$ from the clique $\{h_1, h_2\}$, we use the views $x^a_2$
%  (exclusive to $h_2$) and $x^a_3$ (exclusive to $h_3$). Note that while
%  $x^a_3$ is also a view for $h_2$, $x^a_3$ is independent of $h_2$ given
%  $h_1$.

\paragraph{Non-examples}
\label{sec:non-example}

The simplest example of a model that cannot be recovered by our algorithm is
  a mixture model with a single view, i.e. \figureref{three-view} with
  just the variables $h_1$ and $x_1$.
However, without further assumptions the model itself is not
  identifiable\footnote{To see this, observe that probability-mass
  can be arbitrarily exchanged between $\Pr(h_1)$ and $\Pr(x_1 | h_1)$ while preserving the marginals $\Pr(x_1)$}.
A class of binary-valued noisy-or networks
(see \figureref{examples-noisy-or} for an example)
which do not possess the
  bottleneck property are nonetheless identifiable;
  see \citet{halpern13noisyor} for an algorithm.

%%%%%%%%%%%%%%%%%%%%%%%%%%%%%%
\subsection{Sample complexity}
\label{sec:sampleComplexity}
% PL: "combine" is too vague to be useful
%\LearnMarginals combines two consistent algorithms, \TensorFactorize and
%\LearnCliqueNs, and is thus consistent itself.

In this section, we provide formal conditions under which $\LearnMarginals$
will produce consistent estimates.
We let
$\mOpphat{v}{i}$,
$\hat Z_\sC$,
and $\hat M_\sV$,
denote estimators for
$\mOpp{v}{i}$,
$Z_\sC$,
and $M_\sV$,
respectively.

To apply $\TensorFactorize$, we need that each $\mOpp{v}{i}$ has full
  column rank and that $\pi^{(i)} \eqdef \Pr(h_i) \succ 0$.
But $\mOpp{v}{i}$ is a product of
tensors on the path from $h_i$ to $x_v$, marginalizing out the rest of the graph.
To make this property more interpretable, let us assume the following property:%\verify
\begin{assumption} 
\label{asm:full-rank-plus}
For every clique $\sC \in \sG$ (including ones involving observed variables),
  every $\Re^{k \times k}$ slice of the marginals $Z_\sC$ has full column
  rank $k$ constructed by summing out the remaining indices.
\end{assumption}
In our directed grid example (\sectionref{directedExample}), for the clique $\sC = \{ h_1, h_2 \}$ this
  condition implies $\pi = \Pr(h_1) \succ 0$ and $T = \Pr(h_2 \mid h_1)$ has column rank $k$; 
for the clique $\sC = \{ h_1, x_1 \}$, it implies $O = \Pr(x_1 \mid h_1)$ has column rank $k$.

%\todo{did they prove this; if so just say "they show", not "we can show"}
%Using results from
%  \citet{anandkumar12moments,anandkumar13tensor} we can show that
%  learning $\mOpp{v_1}{i}$ for the bottleneck $h_i$ with views $x_{v_1},
%  x_{v_2}, x_{v_3}$ has the following sample complexity,
%  \todo{define hat O}
%  \todo{max subscript is misaligned}
%\begin{align*}
%  \|\mOpphat{v_1}{i} - \mOpp{v_1}{i}\|^2_F &= \frac{1}{\sqrt{n}} O\left( \frac{k {\pi\oft{i}}_{\max}^2}{\sigma_k(M_{v_1,v_2})^5} \right). 
%\end{align*}

From \citet{anandkumar13tensor} given \assumptionref{full-rank-plus},
we have that $\mOpphat{v}{i}$ converges to $\mOpp{v}{i}$ at a rate of $n^{-\frac12}$ with a constant that
depends polynomially on the $k$-th singular value of $\mOpp{v}{i}$.
%Of course, the singular value of $\mOpp{v}{i}$ become exponentially worse as $x_v$ and $h_i$ become farther.
%So while we can obtain consistent estimates, the sample complexity
%is exponential in the mimimum distance from $h_i$ to any three views, as to be expected.
%(\todo{say something
%quantitative?  annoying part is that we would have to bring in the bounds, so leave it out probably}).
% DONE: need to setup the asymptotic - it's not fresh in everyone's mind.
% Easy case
Next, we need to show that the hidden marginals $\hat Z_\sC$ converge to $Z_\sC$.
First, $\hat M_\sV$ is just an empirical average of multinomials over the data points.
Abusing notation slightly, we let $M_\sV$ also denote its $d^{m}$-dimensional vectorized version.
We also represent $\mH_\sC$ as
  a vector in $\Re^{k^m}$, and represent $\mOppAll \eqdef
  \mOpp{v_1}{i_1} \otimes \cdots \otimes
  \mOpp{v_m}{i_m}$ as a matrix in $\Re^{d^m \times
  k^m}$.
By the central limit theorem, we have:
$\sqrt{n} (\hat M_\sV - M_\sV) \convind \sN(0, \Sigma_\sV)$,
where $\Sigma_\sV$ is the \emph{asymptotic variance} of $\hat M_\sV$: 
% ARUN: I don't think so - \todo{doesn't this need to be multiplied by something?}
\begin{align*}
\Sigma_\sV \eqdef \dM_\sV - M_\sV M_\sV^\top, \quad \dM_\sV \eqdef \text{diag}(M_\sV).
\end{align*}

% PL: this is not obviously correct
%To address the first, we will need that the following assumption on the
%  parameters of the model:
%\begin{assumption} 
%  \label{asm:full-rank-plus}
%  For every clique $\sC = \{h_{i_1}, \cdots, h_{i_m}\} \in \sG$, let
%    $Z_\sC$ be the marginal distribution for that clique. 
%  Then every mode-$i$ unfolding of $Z_\sC$ has full column rank.\verify
%\end{assumption}
%Intuitively, the assumption guarantees that the marginal probability of
%  any hidden state is non-zero.

We can now use the delta-method to convert the above result into 
the asymptotic variance for the pseudoinverse version of $\LearnClique$:
\begin{lemma}[Asymptotic variance of pseudoinverse estimator for $Z_\sC$]
  \label{lem:mom-variance}  
  Assume $\hat M_{\sV}$ has asymptotic variance $\Sigma_\sV$ defined above.
  Then the asymptotic variance of $\hat Z_\sC$ is:
  \begin{align*}
    \Sigma^{\mom} &= \mOppAlli \Sigma_\sV \mOppAllit.
  \end{align*}
  %where $\pinv{\mOppAll} \eqdef \pinv{\mOpp{v_1}{i_1}} \otimes
  %\cdots \otimes \pinv{\mOpp{v_m}{i_m}}$, a $d^m \times k^m$ matrix.
\end{lemma}
\begin{proof}
For clique $\sC$, recall we have
  $Z_{\sC} = \mO_\sV(\mOppi{v_1}{i_1},\cdots,\mOppi{v_m}{i_m})$.
%Choosing to represent $Z_\sC$ and $M_\sV$ as vectors and
%$\mOppi{v_1}{i_1} \otimes \cdots \otimes \mOppi{v_m}{i_m}$ as the
%matrix,
We can rewrite this representation more compactly as: $Z_{\sC} = \pinv{\mOppAll} \mO_\sV$.
By the delta-method \cite{vaart98asymptotic},
% DONE: need to shorten
%we have that the asymptotic variance of $M_\sV$ is:
%\begin{align*}
%  \sqrt{n}(\hat M_\sV - M_\sV) \convind \sN(0, \Sigma_\sV),
%\end{align*}
%where $\Sigma_\sV$ is the variance of the observations. 
we immediately get:
$\sqrt{n}(\hat Z_\sC - Z_\sC) \convind \sN(0, \mOppAlli \Sigma_\sV \mOppAllit)$.
\end{proof}

% DONE: too brazen to say
%Note that extending these results to finite sample bounds can be done
  %via a straightforward application of perturbation bounds.
