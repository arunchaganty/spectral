\documentclass[xcolor={svgnames}]{beamer}

\setbeameroption{hide notes} 

%\usetheme{NLP}
\usetheme{boxes}
\useoutertheme{infolines}

\usepackage{graphicx}
\usepackage{lmodern}
\usepackage{calc}

\usepackage{soul}

\usepackage{amsmath,amsthm,amssymb}   

\usepackage{listings}
\usepackage[style=authoryear,babel=hyphen]{biblatex}
\addbibresource{ref.bib}
\addbibresource{pliang.bib}

%\usepackage{algorithm,algorithmic}

\usepackage{tikz}
\usepackage[debug,debugmarks]{scabby}
%\usepackage{scabby}

\usepackage[customcolors]{hf-tikz}

\usepackage{mathtools}

\providecommand{\Pa}{\textrm{Pa}}
\providecommand{\TensorFactorize}{\textsc{TensorFactorize}~}
\providecommand{\LearnFactors}{\textsc{LearnFactors}~}

% Matrix Perturbation
\newcommand{\inv}[1] {#1^{-1}}
\newcommand{\pinv}[1] {#1^{-\dagger}}
\newcommand{\Ap} {\hat{A}}
\newcommand{\Bp} {\hat{B}}
\newcommand{\Up} {\hat{U}}
\newcommand{\Vp} {\hat{V}}
\newcommand{\Xp} {\hat{X}}
\newcommand{\Wp} {\hat{W}}
\newcommand{\Zp} {\hat{Z}}
\newcommand{\vp} {\hat{v}}
\newcommand{\lambdap} {\hat{\lambda}}
\newcommand{\cnd}[1] {\kappa(#1)}
\newcommand{\aerr}[1] {\varepsilon_{#1}}
\newcommand{\rerr}[1] {\delta_{#1}}
\newcommand{\serr}[1] {\alpha_{#1}}
\newcommand{\gap}[1] {\Delta_{#1}}

% Keywords
\newcommand{\Pairs}{\mathrm{Pairs}}
\newcommand{\Triples}{\mathrm{Triples}}



\newcommand{\drawgen}[1]{%
  \mkcircle{h}{}{#1}{1.5ex};
  \node[left=0.1cm of h] {$h$};
  \mkcircle{x}{fill=green,opacity=0.7}{($#1 - (0,1cm)$)}{1.5ex};
  \node[left=0.1cm of x] {$x$};
  \draw[-latex] (h) -- (x);
}



% these will be used later in the title page
\title[Moments and Likelihoods]{Estimating Latent Variable Graphical Models with Moments and Likelihoods}
\author[Chaganty, Liang]{%
    Arun Tejasvi Chaganty\\
    Percy Liang
}
\institute{Stanford University}

\begin{document}

% "Beamer, do the following at the start of every section"
\AtBeginSection[] 
{%
\begin{frame}<beamer> 
\frametitle{Outline} % make a frame titled "Outline"
\tableofcontents[currentsection]  % show TOC and highlight current section
\end{frame}
}

\begin{frame}
  \titlepage
\end{frame}

\section{Introduction}

\begin{frame}
  \frametitle{Latent Variable Graphical Models}

  \splitcolumn{%
      \begin{itemize}
        \item Gaussian Mixture Models \tikzmark{gmm}
        \item Latent Dirichlet Allocation
        \item Hidden Markov Models \tikzmark{hmm}
        \item PCFGs
        \item \dots
      \end{itemize}
  }{%
  \begin{canvas}
    \point{mark}{(1cm,0)};
    \point{gmm}{($(mark) + (0,+0cm)$)};
    \point{hmm}{($(mark) + (0,-1cm)$)};

    \node[anchor=south west] (mog) at (gmm) {%
      \includegraphics[width=0.45\textwidth,height=3cm,keepaspectratio]{figures/mog.pdf}
    };

    %\drawgen{($(gmm) + (0,0.0cm)$)};
    \drawhmm{(hmm)};
  \end{canvas}
  }

\end{frame}

\begin{frame}
  \frametitle{Parameter Estimation is Hard}

  \begin{tikzpicture}
    % x, y
    \llhood{0}{0};
    \node<2->[scale=0.3,circle,fill=black] at (mle) {};
    \node<2-> at ($(mle) + (0.6cm,0)$) {$\mathmb{\textrm{MLE}}$};
    \node<3->[scale=0.3,circle,fill=black] at (em1) {};
    \node<3-> at ($(em1) + (0.5cm,0)$) {$\mathmr{\textrm{EM}}$};
    \node<3->[scale=0.3,circle,fill=black] at (em2) {};
    \node<3-> at ($(em2) + (0.5cm,0)$) {$\mathmr{\textrm{EM}}$};

    \node<3->[scale=0.3,circle,fill=black] at (spec) {};
    \node<3-> at ($(spec) + (0.5cm,0.3cm)$) {$\mathmg{\textrm{MoM}}$};
   % \draw<4->[latex-latex,DarkGreen,line width=1pt] ($(mle) + (-0.8cm,0.8cm)$) -- node[above]{$\mathmg{\epsilon}$} ($(mle) + (+0.8cm,0.8cm)$);
  \end{tikzpicture}

  % Simple message: MLE is consistent but intractable, EM is efficient not but consistent. Can we get something in between.

  \begin{itemize}
    \item<1-> Log-likelihood function is non-convex.
    \item<2-> MLE is consistent but intractable.
    \item<3-> Local methods (EM, gradient descent, etc.) are tractable but inconsistent\alt<3-4>{.}{\em and generalize easily.}
    \item<4-> {\em Method of moments} estimators can be consistent and
      computationally-efficient, but more data. 
         \uncover<6->{\bf Thus far, applicable to a limited set of models.}
  \end{itemize}
\end{frame}

\begin{frame}
  \frametitle{Main contributions}

  \splitcolumn{%
    \begin{itemize}
     % \item {\em Before our work}
     % \begin{itemize}
     %   \item Gaussian Mixture Models \tikzmark{gmm}
     %   \item Hidden Markov Models 
     %   \item Latent Dirichlet Allocation
     % \end{itemize}
     \item<2-> Succinct conditions and an algorithm for a {\bf broader model
       family}. \tikzmark{grid}
    \item<3-> Efficiently learn models with low-degree but {\bf high-treewidth}.
    \item<3-> Extends to {\bf log-linear models}.
    \item<4-> Combine moment estimators with composite
      likelihood estimators.
    \end{itemize}
  }{%
  \begin{canvas}
    \point{mark}{(4cm,0)};
    %\point{gmm}{({pic cs:gmm} -| mark)};
    %\point{grid}{({pic cs:grid} -| mark)};
    \point{gmm}{($(mark) + (0,3cm)$)};
    \point{grid}{($(mark) + (0,-1cm)$)};

%    \node[anchor=south west] (mog) at (gmm) {%
%      \includegraphics[width=0.45\textwidth,height=3cm,keepaspectratio]{figures/mog.pdf}
%    };

    \drawgen{($(gmm) + (0,0.0cm)$)};
    \drawgrid<2->{(grid)};
  \end{canvas}
  }


%  \cornertext<1->{\cite{AnandkumarGeHsu2012}}
%
%  \begin{canvas}
%    % Tasks.
%
%    \node<1->[anchor=west] (diag) at (-3cm, 1cm) {%
%      \includegraphics[width=0.45\textwidth,height=3cm,keepaspectratio]{figures/mog.pdf}
%    };
%    %\drawgen{(-3cm,1cm)}
%    \node[below=0.6cm of diag.south] {Before};
%
%    % Highlight
%    \draw<2>[scale=0.8,fill=green,opacity=0.4,dashed] (1cm,2.5cm) rectangle (6.5cm,-2.5cm);
%      \drawgrid{(3cm,1cm)}
%      \node[below=0.1cm of h4.south] {After};
%  \end{canvas}

\end{frame}

% \begin{frame}
%   \frametitle{Related Work}
%   \begin{itemize}
%     \item<1-> Method of Moments [Pearson, 1894]
%     \item<2-> Observable operators
%     \begin{itemize}
%       \item Control Theory [Ljung, 1987]
%       \item Observable operator models [Jaeger, 2000; {\small{Littman/Sutton/Singh, 2004}}]
%       \item Hidden Markov models [Hsu/Kakade/Zhang, 2009]
%       \item Low-treewidth graphs [Parikh et al., 2012]
%       \item Weighted finite state automata [Balle \& Mohri, 2012]
%     \end{itemize}
%      \item<3-> Parameter Estimation
%   \begin{itemize}
%     \item Mixture of Gaussians [Kalai/Moitra/Valiant, 2010]
%     \item \alert{Mixture models, HMMs [Anandkumar/Hsu/Kakade, 2012]}
%     \item Latent Dirichlet Allocation [Anandkumar/Hsu/Kakade, 2012]
%     \item Stochastic block models [Anandkumar/Ge/Hsu/Kakade, 2012]
%     \item Linear Bayesian networks [Anandkumar/Hsu/Javanmard/Kakade, 2012]
%   \end{itemize}
%   \end{itemize}
% \end{frame}

\section{Three-view Mixture Models}

\begin{frame}
  \frametitle{Three-view Mixture Models}
  \cornertext<1->{\cite{AnandkumarGeHsu2012}}
  \begin{canvas}
    % The model
    \point{mark}{(1cm,0)};
    \point{start}{(1cm,1cm)}; %{pic cs:gen} -| mark)};
    \drawgen<1->{($(start) + (-4cm,1cm)$)};
    %\node[anchor=west] (diag1) at ($(start)$) {%
    %  \includegraphics[width=0.45\textwidth,height=2cm,keepaspectratio]{figures/gen.png}
    %};
    \node<1->[anchor=west] (diag) at ($(start) + (0cm,0.0cm)$) {%
      \includegraphics[width=0.45\textwidth,height=3cm,keepaspectratio]{figures/mog.pdf}
    };

    \node<6-> (bottleneck) at ($(x2.south) - (0,0.4cm)$) {\textmb{\bf Bottleneck}};

    \uncover<2>{
    \vectorfactorization{(-3cm,-2cm)}
    }
    \uncover<3>{
    \matrixfactorization{(-3cm,-2cm)}
    }


    \node<5-> (tf) at ($(t3.north) + (-1cm,0.5cm)$) {Tensor eigen-decomposition};

    \uncover<4->{
    \tensorfactorization{(-3cm,-2cm)}
    }

  \end{canvas}
\end{frame}

\begin{frame}
  \frametitle{}

  \begin{canvas}
    % Tasks.
    \drawgen{(-3cm,1cm)}
    \node[below=0.1cm of x2.south] {Bottlenecked};

    % Highlight
    \draw<2>[scale=0.8,fill=green,opacity=0.4,dashed] (1cm,3.0cm) rectangle (6.5cm,-3.0cm);
      \drawgrid{(3cm,1cm)}
      \node[below=0.8cm of h4.south] {Uniformly bottlenecked};

  \end{canvas}

\end{frame}

\section{Uniformly Bottlenecked Models}
% Example 
% General algorithm
% Conditions
% Efficiency 1: EM (+diagram).

\section{Log-linear models}
% Convergence in the limit.
% Pseudo-likelihood and optimizations.

\section{Conclusions}

\begin{frame}
  \frametitle{Conclusions}
  \begin{itemize}
      \item Uniformly bottlenecked models
      \item Scales with the size of each clique, not the tree-width
      \item Solving bottlenecks breaks problem into convex pieces; can be solved more accurately
      \item The marginals make the log-linear recovery problem convex.
  \end{itemize}
\end{frame}

\begin{frame}
  \frametitle{}
    Thank you!
\end{frame}

\end{document}

