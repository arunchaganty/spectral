\documentclass[xcolor={svgnames}]{beamer}

\setbeameroption{hide notes} 

%\usetheme{NLP}
\usetheme{boxes}
\useoutertheme{infolines}

\usepackage{graphicx}
\usepackage{lmodern}
\usepackage{calc}

\usepackage{soul}

\usepackage{amsmath,amsthm,amssymb}   

\usepackage{listings}
\usepackage[style=authoryear,babel=hyphen]{biblatex}
\addbibresource{ref.bib}
\addbibresource{pliang.bib}

%\usepackage{algorithm,algorithmic}

\usepackage{tikz}
%\usepackage[debug,debugmarks]{scabby}
\usepackage[nodefinetheorems]{scabby}
\usepackage[debug,beamer]{scabby-diag}

\usepackage[customcolors]{hf-tikz}

\usepackage{mathtools}

\providecommand{\Pa}{\textrm{Pa}}
\providecommand{\TensorFactorize}{\textsc{TensorFactorize}~}
\providecommand{\LearnFactors}{\textsc{LearnFactors}~}

%% Matrix Perturbation
\newcommand{\inv}[1] {#1^{-1}}
\newcommand{\pinv}[1] {#1^{-\dagger}}
\newcommand{\Ap} {\hat{A}}
\newcommand{\Bp} {\hat{B}}
\newcommand{\Up} {\hat{U}}
\newcommand{\Vp} {\hat{V}}
\newcommand{\Xp} {\hat{X}}
\newcommand{\Wp} {\hat{W}}
\newcommand{\Zp} {\hat{Z}}
\newcommand{\vp} {\hat{v}}
\newcommand{\lambdap} {\hat{\lambda}}
\newcommand{\cnd}[1] {\kappa(#1)}
\newcommand{\aerr}[1] {\varepsilon_{#1}}
\newcommand{\rerr}[1] {\delta_{#1}}
\newcommand{\serr}[1] {\alpha_{#1}}
\newcommand{\gap}[1] {\Delta_{#1}}

% Keywords
\newcommand{\Pairs}{\mathrm{Pairs}}
\newcommand{\Triples}{\mathrm{Triples}}



\newcommand{\drawgen}[1]{%
  \mkcircle{h}{}{#1}{1.5ex};
  \node[left=0.1cm of h] {$h$};
  \mkcircle{x}{fill=green,opacity=0.7}{($#1 - (0,1cm)$)}{1.5ex};
  \node[left=0.1cm of x] {$x$};
  \draw[-latex] (h) -- (x);
}



% these will be used later in the title page
\title[Moments and Likelihoods (M58)]{Estimating Latent Variable Graphical Models with Moments and Likelihoods}
\author[Chaganty, Liang]{%
    Arun Tejasvi Chaganty\\
    Percy Liang
}
\institute{Stanford University}

\begin{document}

\begin{frame}
  \titlepage
\end{frame}

\section{Introduction}

% \begin{frame}
%   \frametitle{Latent Variable Graphical Models}
% 
%   \splitcolumn{%
%       \begin{itemize}
%         \item Gaussian Mixture Models \tikzmark{gmm}
%         \item Latent Dirichlet Allocation
%         \item Hidden Markov Models \tikzmark{hmm}
%         \item PCFGs
%         \item \dots
%       \end{itemize}
%   }{%
%   \begin{canvas}
%     \point{mark}{(1cm,0)};
%     \point{gmm}{($(mark) + (0,+0cm)$)};
%     \point{hmm}{($(mark) + (0,-1cm)$)};
% 
%     \node[anchor=south west] (mog) at (gmm) {%
%       \includegraphics[width=0.45\textwidth,height=3cm,keepaspectratio]{figures/mog.pdf}
%     };
% 
%     %\drawgen{($(gmm) + (0,0.0cm)$)};
%     \drawhmm{(hmm)};
%   \end{canvas}
%   }
% 
% \end{frame}
% 
% \begin{frame}
%   \frametitle{Parameter Estimation is Hard}
% 
%   \begin{tikzpicture}
%     % x, y
%     \llhood{0}{0};
%     \node<2->[scale=0.3,circle,fill=black] at (mle) {};
%     \node<2-> at ($(mle) + (0.6cm,0)$) {$\mathmb{\textrm{MLE}}$};
%     \node<3->[scale=0.3,circle,fill=black] at (em1) {};
%     \node<3-> at ($(em1) + (0.5cm,0)$) {$\mathmr{\textrm{EM}}$};
%     \node<3->[scale=0.3,circle,fill=black] at (em2) {};
%     \node<3-> at ($(em2) + (0.5cm,0)$) {$\mathmr{\textrm{EM}}$};
% 
%     \node<4->[scale=0.3,circle,fill=black] at (spec) {};
%     \node<4-> at ($(spec) + (0.5cm,0.3cm)$) {$\mathmg{\textrm{MoM}}$};
%    % \draw<4->[latex-latex,DarkGreen,line width=1pt] ($(mle) + (-0.8cm,0.8cm)$) -- node[above]{$\mathmg{\epsilon}$} ($(mle) + (+0.8cm,0.8cm)$);
%   \end{tikzpicture}
% 
%   % Simple message: MLE is consistent but intractable, EM is efficient not but consistent. Can we get something in between.
% 
%   \begin{itemize}
%     \item<1-> Log-likelihood function is non-convex.
%     \item<2-> MLE is consistent but intractable.
%     \item<3-> Local methods (EM, gradient descent, \dots) are tractable but inconsistent.
%     \item<4-> {\em Method of moments} estimators can be consistent and
%       computationally-efficient, but require more data. 
%   \end{itemize}
% \end{frame}
% 
% \begin{frame}
%   \frametitle{Consistent estimation for general models}
% 
%   \begin{itemize}
%     \item<+-> Several estimators based on the method of moments.
%       \begin{itemize}
%         \item {\bf Phylogenetic trees:} \cite{mossel2005learning}.
%         \item {\bf Hidden Markov models:} \cite{hsu09spectral}
%         \item {\bf Latent Dirichlet Allocation:} \cite{anandkumar12lda}
%         \item {\bf Latent trees:} \cite{anandkumar11tree}
%         \item {\bf PCFGs:} \cite{hsu12identifiability}
%         \item {\bf Mixtures of linear regressors} \cite{chaganty13regression}
%         \item {\bf \ldots}
%       \end{itemize}
%     \item<+-> These estimators are applicable only to a specific type of model. 
%     \item<+-> In contrast, EM and gradient descent apply for almost any model.
%     \item<+-> Note: some work in the observable operator framework does apply to a more general model class.
%       \begin{itemize}
%         \item {\bf Weighted automata:} \cite{balle12automata}.
%         \item {\bf Junction trees:} \cite{song2011spectral}
%         \item {\bf \ldots}
%         \item \todo{Check that this list is representative}
%       \end{itemize}
%     \item<+-> {\bf How can we apply the method of moments to estimate {\em parameters efficiently} for a {\em general} model?}
%   \end{itemize}
% \end{frame}
% 
% \begin{frame}
%   \frametitle{Setup}
%   \splitcolumn{%
%     \begin{itemize}
%       \item<1-> Discrete models with $k$ hidden and $d \ge k$
%         observed values.
%       \item<2-> Parameters and marginals can be represented as matrices
%         and tensors.
%       \item<5-> Presented in terms of infinite data and exact moments.
%     \end{itemize}
%   }{%
%     \begin{canvas}
%         % The model
%         \point{start}{(2cm,0cm)}; %{pic cs:gen} -| mark)};
%         \drawgensquigglescale<1->{($(start) + (0cm,3cm)$)}{0.6};
%         \node[right=0.5cm of h,scale=0.6] {$h \in \{1, 2, \cdots, k\}$};
%         \node[right=0.5cm of x3,scale=0.6] {$x_i \in \{1, 2, \cdots, d\}$};
% 
%         \uncover<2->{
%         % Matrices
%         % - M12
%         \node[scale=0.8] (m12) at ($(start) + (0.5cm,0.25cm)$) {\objw{3cm}{
%           \begin{align*}
%             M_{12} &\eqdef \Pr(x_1, x_2) \\
%             {\color{blue} (M_{12})_{ij}} &\eqdef {\color{blue} \Pr(x_1 = i, x_2 = j)}
%           \end{align*}
%           }
%         };
% %        \point{m12c}{($(m12) + (1.5cm,0.75cm)$)};
%          \tikzrect{m12r}{black,fill=white} {($(m12) + (3.2cm,0.75cm)$)}{1}{1};
%          \tikzrect{m12ijr}{black,fill=blue}{($(m12) + (3.2cm,0.75cm)$)}{0.2}{0.2};
%          }
%          \uncover<3-> {
%         % - M123
%         \node[below=0.25cm of m12, scale=0.8] (m123) {\objw{3cm}{
%           \begin{align*}
%             M_{123} &\eqdef \Pr(x_1, x_2, x_3) \\
%             {\color{blue} (M_{123})_{ijk}} &\eqdef {\color{blue} \Pr(x_1 = i, x_2 = j, x_3=k)}
%           \end{align*}
%           }
%         };
% %        \point{m123-c}{($(m123) + (1.75cm,0.75cm)$)};
%         \tikzcube{m123r}{black,fill=white} {($(m123) + (3.2cm,0.75cm)$)}{1}{1}{1};
%         \tikzcube{m123ijr}{black,fill=blue}{($(m123) + (3.2cm,0.75cm)$)}{0.2}{0.2}{0.2};
%         }
% 
%         \uncover<4-> { 
%         % - O11
%         \node[below=0.25cm of m123, scale=0.8] (o11) {\objw{3cm}{
%           \begin{align*}
%             \mOpp{1}{1} &\eqdef \Pr(x_1 \given h_1) \\
%             {\color{DarkGreen} (\mOpp{1}{1})_{ij}} &\eqdef {\color{DarkGreen} \Pr(x_1 = i \given h_1 = j)}
%           \end{align*}
%           }
%         };
% %        \point{m123-c}{($(m123) + (1.75cm,0.75cm)$)};
%         \tikzrect{o11r}{black,fill=white} {($(o11) + (3.2cm,0.75cm)$)}{1}{1};
%         \tikzrect{o11ijr}{black,fill=DarkGreen}{($(o11) + (3.2cm,0.75cm)$)}{0.2}{0.2};
%         }
% 
%       \end{canvas}
%   }
% \end{frame}
% 
% \begin{frame}
%   \frametitle{Setup}
%   \splitcolumn{%
%     \begin{itemize}
%       \item<1-> Directed models parameterized by conditional probability
%         tables.
%       \item<2-> Undirected models parameterized as a log-linear model.
%         Identify modulo $A(\theta)$.
%       \item<3-> Focus on directed models, but return to undirected
%         models later.
%     \end{itemize}
%   }{%
%     \begin{canvas}
%       \point{stuff}{(2cm,0cm)};
%       \drawbridge<1->{($(stuff)+(0,2cm)$)};
%       \node[scale = 0.5, anchor=south] at (h1h2) {
%         \begin{tabular}{r | l l}
%           \diaghead{aaaaaa}{$h_2$}{$h_1$} &
%           \thead{$0$} & \thead{$1$} \\ \hline 
%           $0$ & \quad & \quad \\ 
%           $1$ & \quad & \quad 
%         \end{tabular}
%       };
%       \drawubridge<2->{($(stuff) + (0,-1cm)$)};
%       \node[scale = 1.0, anchor=south] at (h1h2u) {$\theta$};
%     \end{canvas}
%   }
% \end{frame}
% 
% \begin{frame}
%   \frametitle{Background: Three-view mixture models aka bottlenecks}
%   \cornertext<1->{}
%   \splitcolumn{%
%     \begin{definition}[Bottleneck]
%       A hidden variable $h$ is a {\bf bottleneck} if there exist three
%       observed variables ({\bf views}) $x_1, x_2, x_3$ that are
%       {\em conditionally independent} given $h$.
%     \end{definition}
% 
%     \begin{itemize}
%       \item<2-> \cite{anandkumar13tensor} provide an algorithm to
%         estimate conditional moments $\mOpp{i}{1} \eqdef \Pr(x_i \given
%         h_1)$ based on tensor eigendecomposition.
%       \item<2-> In general, three views are necessary for identifiability
%         (\cite{kruskal77three}).
%     \end{itemize}
%   }{%
%     \begin{canvas}
%         % The model
%         \point{start}{(2cm,0cm)}; %{pic cs:gen} -| mark)};
%         \drawgensquiggle<1->{($(start) + (1cm,1cm)$)};
%         %\node[anchor=west] (diag1) at ($(start)$) {%
%         %  \includegraphics[width=0.45\textwidth,height=2cm,keepaspectratio]{figures/gen.png}
%         %};
%         %\node<1->[anchor=west] (diag) at ($(start) + (0cm,-1cm)$) {%
%         %  \includegraphics[width=0.45\textwidth,height=3cm,keepaspectratio]{figures/mog.pdf}
%         %};
%       \end{canvas}
%   }
% \end{frame}
% 
% \begin{frame}
%   \frametitle{Example: a bridge, take I}
%   \splitcolumn{%
%     \begin{itemize}
%       \item<1-> Each edge has a set of parameters.
%       \item<2-> $h_1$ and $h_2$ are bottlenecks.
%       \item<3-> We can learn $\Pr(x_1^a | h_1), \Pr(x_1^b | h_1), \cdots$.
%       \item<6-> However, we can't learn $\Pr(h_2 | h_1)$ this way.
%     \end{itemize}
%   }{%
%   \begin{canvas}
%     \point{mark}{(3cm,0)};
%     \drawbridge{(mark)};
% 
%   \begin{pgfonlayer}{background}
%   \draw<2>[draw=black,fill=green!70,rounded corners,line width=1pt, dotted] 
%                   ($(x1a.west) + (180:0.3cm)$) -- 
%                   ($(h1.north) + (90:0.3cm)$) -- 
%                   ($(x2a.east) + (0:0.3cm)$) -- 
%                   ($(x2a.south) + (-90:0.3cm)$) -- 
%                   ($(x1b.south) + (-90:0.3cm)$) -- 
%                   ($(x1a.south) + (-90:0.3cm)$) -- 
%                   cycle;
%   \draw<2>[dashed,-latex] (h1) -- (x2a);
%   \end{pgfonlayer}
% 
% 
%    \uncover<3->{
%    \draw[-latex,green] (h1) -- (x1a);
%    \draw[-latex,green] (h1) -- (x1b);
% 
%    \draw[-latex,green,dashed] (h1) -- (x2a);
%    }
% 
%   \begin{pgfonlayer}{background}
%   \draw<4>[draw=black,fill=green!70,rounded corners,line width=1pt, dotted] 
%                   ($(x2b.east) + (0:0.3cm)$) -- 
%                   ($(h2.north) + (90:0.3cm)$) -- 
%                   ($(x1b.west) + (180:0.3cm)$) -- 
%                   ($(x1b.south) + (-90:0.3cm)$) -- 
%                   ($(x2a.south) + (-90:0.3cm)$) -- 
%                   ($(x2b.south) + (-90:0.3cm)$) -- 
%                   cycle;
%   \draw<4>[dashed,-latex] (h2) -- (x1b);
%   \end{pgfonlayer}
% 
%    \uncover<5->{
%    \draw[-latex,green] (h2) -- (x2a);
%    \draw[-latex,green] (h2) -- (x2b);
%    \draw[-latex,green,dashed] (h2) -- (x1b);
%    }
%    \uncover<6->{
%    \draw[-latex,red,line width=1.3pt] (h1) -- (h2);
%    }
% 
%   \end{canvas}
%   }
% \end{frame}
% 
% \begin{frame}
%   \frametitle{Example: a bridge, take II}
%   %\fontsize{8pt}{8.2pt}\selectfont
%   \splitcolumn{%
%     \begin{itemize}
%       \item<1-> Observe the joint distribution of $x_1, x_2$,
%         \begin{align*}
%           \underbrace{\Pr(x_1^b, x_2^a)}_{M_{12}} &= \sum_{h_1, h_2} 
%           \mathmg{\underbrace{\Pr(x_1^b \given h_1)}_{\mOpp{1}{1}}}
%           \mathmg{\underbrace{\Pr(x_2^a \given h_2)}_{\mOpp{2}{2}}}
%           \mathmb{\underbrace{\Pr(h_1, h_2)}_{Z_{12}}}.
%         \end{align*}
%       \item<2-> {\bf Observed moments} $\Pr(x_1^b, x_2^a)$ are {\em linear} in the {\bf hidden marginals} $\Pr(h_1, h_2)$.
%         %\begin{align*}
%         %  M_{12} &= \mOpp{1}{1} Z_{12} \mOppt{2}{1}
%         %\end{align*}
%       \item<3-> Solve for $\Pr(h_1, h_2)$ by pseudoinversion.
%         %\begin{align*}
%         %  Z_{12} &= \mOppi{1}{1} M_{12} \mOppit{2}{1}
%         %\end{align*}
%       \item<4-> Normalize for $\Pr(h_2 \given h_1)$.
%     \end{itemize}
%   }{%
%   \begin{canvas}
%     \point{mark}{(1cm,0)};
%     \point{start-bridge}{($(mark) + (1cm,3.0cm)$)};
%    \node[style=node, scale=0.8] (h1) at (start-bridge) {$h_1$};
%    \node[style=node, scale=0.8, right= 1.0cm of h1] (h2) {$h_2$};
%    \draw[-latex] (h1) -- (h2);
% 
%    %\point{V}{($(h4.north) + (0,0.1cm)$)};
% 
% % Observed nodes
%    \node[style=obsnode, scale=0.6, below left=0.3cm of h1] (x1a) {$x_1^a$};
%    \node[style=obsnode, scale=0.6, below=0.3cm of h1] (x1b) {$x_1^b$};
%    \node[style=obsnode, scale=0.6, below=0.3cm of h2] (x2a) {$x_2^a$};
%    \node[style=obsnode, scale=0.6, below right=0.3cm of h2] (x2b) {$x_2^b$};
%    \draw[-latex,gray] (h1) -- (x1a);
%    \draw[-latex,green] (h1) -- (x1b);
%    \draw[-latex,green] (h2) -- (x2a);
%    \draw[-latex,gray] (h2) -- (x2b);
% 
% % Draw the matrices
% \uncover<2->{
%   \point{eq}{($(mark) + (0.5cm,0.0cm)$)};
%   \tikzrect{m12}{black,fill=white} {($(eq) + (0.0cm,0cm)$)}{1.0}{1.0};
%   \node at ($(eq) + (0.5cm,-0.5cm)$) {$=$};
%   \tikzrect{o11}{black,fill=green} {($(eq)+(1.5cm,0)$)}{0.5}{1.0};
%   \tikzrect{z12}{black,fill=blue} {($(eq)+(2.5cm,0)$)}{0.5}{0.5};
%   \tikzrect{o22}{black,fill=green} {($(eq)+(4cm,0)$)}{1.0}{0.5};
%   \node at (m12) {\small $M_{12}$};
%   \node at (o11) {\small $\mOpp{1}{1}$};
%   \node at (o22) {\small $\mOpp{2}{2}$};
%   \node at (z12) {\small $Z_{12}$};
%   }
% 
%   \uncover<3->{
%   \point{ieq}{($(mark) + (0.5cm,-1.5cm)$)};
%   \tikzrect{z12i}{black,fill=blue} {($(ieq) + (0cm,-0.25cm)$)}{0.5}{0.5};
%   \node at ($(ieq) + (0.5cm,-0.5cm)$) {$=$};
%   \tikzrect{o11i}{black,fill=green} {($(ieq)+(2.0cm,0)$)}{1.0}{0.5};
%   \tikzrect{m12i}{black,fill=white} {($(ieq)+(3.5cm,0)$)}{1.0}{1.0};
%   \tikzrect{o22i}{black,fill=green} {($(ieq)+(4.5cm,0)$)}{0.5}{1.0};
% 
%   \node at (z12i) {\small $Z_{12}$};
%   \node at (m12i) {\small $M_{12}$};
%   \node at (o11i) {\small $\mOppi{1}{1}$};
%   \node at (o22i) {\small $\mOppi{2}{2}$};
%   }
% 
%   \end{canvas}
%   }
% \end{frame}
% 
% 
% \begin{frame}
% \frametitle{Outline} 
% \begin{canvas}
%   \uncover<1>{
%   \initialoverview{(-2cm,3cm)};
%   }
%   \uncover<2->{
%   \overview{(-2cm,3cm)};
%   }
%   \uncover<3-> {
%     \node[right=0.5cm of obs-edge] {
%     \alt<3>{\color{DarkGreen}}{} 1. Solve bottlenecks
%     };
%   }
%   \uncover<4-> {
%     \node[right=0.5cm of cond-edge] {\obj{
%     \color{DarkGreen} 2a. Pseudoinverse\\
%     \hphantom{2b. Composite likelihood}
%     }};
%   }
% \end{canvas}
% \end{frame}
% 
% \section{Estimating Hidden Marginals}
% 
% \begin{frame}
%   \frametitle{Exclusive Views}
%   \splitcolumn{%
%   \begin{definition}[Exclusive views]
%     We say $h_i \in S \subseteq \bh$ has an {\bf exclusive view} $x_v$
%       if
%       \begin{enumerate}
%         \item<1-> There exists {\em some observed variable $x_{v}$} which is
%           {\em conditionally independent of the others} ($S \backslash \{ h_i \}$)
%           given $h_i$.
%         \item<2-> The conditional moment matrix $\mOpp{v}{i} \eqdef
%           \Pr(x_{v} \mid h_i)$ has full column rank $k$ and can be
%           recovered.
%         \item<3-> A set has exclusive views if each $h_i \in S$ has an
%           exclusive view.
%       \end{enumerate}
%   \end{definition}
%   }{%
%   \begin{canvas}
%     \point{mark}{(4cm,0)};
%     \node at (mark) {%
%       \includegraphics[width=\textwidth,height=4cm,keepaspectratio]{figures/exclusive-views.pdf}
%       };
%   \end{canvas}
%   }
% \end{frame}

\begin{frame}
  \frametitle{Exclusive views give parameters}
  %\splitcolumn{%
  \begin{itemize}
    \item Given {\em exclusive views}, $\Pr(x \given h)$,
      learning cliques is solving a linear equation!
%        \todo{Use cartoon tensors}
      \begin{align*}
        \underbrace{\Pr(x_1, \ldots, x_m)}_{M} &=
        \sum_{h_1, \ldots, h_m}
        \underbrace{P(x_1 | h_1)}_{\mOpp{1}{1}} \cdots \underbrace{\mathmb{P(h_1, \cdots, h_m)}}_{Z}
      \end{align*}
  \end{itemize}
  %}{%
  \begin{tikzpicture}
    %\point{mark}{(4cm,0)};
    % 
    \uncover<2->{
    \point{eq}{(0,0)};
    \tikzcube{m}{black,fill=white} {($(eq) + (0.0cm,0cm)$)}{1.0}{1.0}{1.0};
    \node at ($(eq) + (0.75cm,-0.5cm)$) {$=$};
    \tikzrect{o11}{black,fill=green} {($(eq)+(1.75cm,-0.25cm)$)}{0.5}{1.0};
    \tikzcube{z}{black,fill=blue} {($(eq)+(2.5cm,-0.25cm)$)}{0.5}{0.5}{0.5};
    \tikzrect{o22}{black,fill=green} {($(eq)+(4cm,0cm)$)}{1.0}{0.5};
    \tikzrect{o33}{black,fill=green} {($(eq)+(2.75cm,1.25cm)$)}{0.5}{1.0};
    \node at (m) {\small $M$};
    \node at (o11) {\small $\mOpp{1}{1}$};
    \node at (o22) {\small $\mOpp{2}{2}$};
    \node at (o33) {\small $\mOpp{3}{3}$};
    \node at (z) {\small $Z$};
    }


    \uncover<3->{
    \point{ieq}{(6cm,0)};
    \node at ($(ieq) - (1.25cm,0.25cm)$) {$\to$};

    \tikzcube{zi}{black,fill=blue} {($(ieq) + (0.0cm,-0.25cm)$)}{0.5}{0.5}{0.5};
    \node at ($(ieq) + (0.50cm,-0.5cm)$) {$=$};
    \tikzrect{o11i}{black,fill=green} {($(ieq)+(1.75cm,-0.25cm)$)}{1.0}{0.5};
    \tikzcube{mi}{black,fill=white} {($(ieq)+(3.0cm,0.0cm)$)}{1.0}{1.0}{1.0};
    \tikzrect{o22i}{black,fill=green} {($(ieq)+(4.5cm,0.25cm)$)}{0.5}{1.0};
    \tikzrect{o33i}{black,fill=green} {($(ieq)+(3.25cm,1.25cm)$)}{1.0}{0.5};
    \node at (mi) {\small $M$};
    \node at (o11i) {\small $\mOppi{1}{1}$};
    \node at (o22i) {\small $\mOppi{2}{2}$};
    \node at (o33i) {\small $\mOppi{3}{3}$};
    \node at (zi) {\small $Z$};

    }

  \end{tikzpicture}
  %}
\end{frame}

% \begin{frame}
%   \frametitle{Bottlenecked graphs}
%   \splitcolumn{%
%   \begin{itemize}
%     \item<1-> When are we assured exclusive views?
%     \item<2-> {\bf Theorem:} A clique in which {\bf each hidden variable is
%       a bottleneck} has exclusive views. 
%       \begin{itemize}
%         \item<3-> Follows by graph independence conditions.
%         \item<4-> We say that the clique is ``bottlenecked''.
%       \end{itemize}
%   \end{itemize}
%   }{%
%   \begin{canvas}
%     \point{mark}{(4cm,0)};
%     \node at (mark) {%
%       \includegraphics[width=\textwidth,height=4cm,keepaspectratio]{figures/exclusive-views.pdf}
%       };
%   \end{canvas}
%   }
% \end{frame}
% 
% \begin{frame}
% \frametitle{Outline} 
% \begin{canvas}
%   \overview{(-2cm,3cm)};
%   \node[right=0.5cm of obs-edge] {1. Solve bottlenecks};
%   \node[right=0.5cm of cond-edge] {\obj{
%   {\alt<1>{\color{DarkGreen}}{} 2a. Pseudoinverse\\}
%   \uncover<2->{\color{DarkGreen} 2b. Composite likelihood}
%   }};
% \end{canvas}
% \end{frame}
% 
% \begin{frame}
%   \frametitle{Example}
%   \begin{canvas}
%     \point{mark}{(3cm,0)};
%     \point{start-grid}{(mark)};
%    \node[style=node, scale=0.8] (h1) at (start-grid) {$h_1$};
%    \node[style=node, scale=0.8, below left= 0.5cm of h1] (h2) {$h_2$};
%    \node[style=node, scale=0.8, below right= 0.5cm of h1] (h3) {$h_3$};
%    \node[style=node, scale=0.8, below right= 0.5cm of h2] (h4) {$h_4$};
% 
%    \point{pi}{($(h1.north) + (0,0.1cm)$)};
%    \draw[-latex] (h1) -- node[scale=0.7,above] (T1) {} (h2);
%    \draw[-latex] (h1) -- node[scale=0.7,above] (T2) {} (h3);
%    \draw[-latex] (h2) -- (h4);
%    \draw[-latex] (h3) -- (h4);
%    \point{V}{($(h4.north) + (0,0.1cm)$)};
%   % Observed nodes
%    \node[style=obsnode, scale=0.7, above left=0.3cm of h1] (x1a) {$x^a_1$};
%    \node[style=obsnode, scale=0.7, above right=0.3cm of h1] (x1b) {$x^b_1$};
%    \draw[-latex] (h1) -- (x1a);
%    \draw[-latex] (h1) -- (x1b);
% 
%    \node[style=obsnode, scale=0.7, above left=0.3cm of h2] (x2a) {$x^a_2$};
%    \node[style=obsnode, scale=0.7, below left=0.3cm of h2] (x2b) {$x^b_2$};
%    \draw[-latex] (h2) -- node[scale=0.7,above] (O1) {} (x2a);
%    \draw[-latex] (h2) -- node[scale=0.7,below] (O2) {} (x2b);
% 
%    \node[style=obsnode, scale=0.7, above right=0.3cm of h3] (x3a) {$x^a_3$};
%    \node[style=obsnode, scale=0.7, below right=0.3cm of h3] (x3b) {$x^b_3$};
%    \draw[-latex] (h3) -- (x3a);
%    \draw[-latex] (h3) -- (x3b);
%     
%    \node[style=obsnode, scale=0.7, below left=0.3cm of  h4] (x4a) {$x^a_4$};
%    \node[style=obsnode, scale=0.7, below right=0.3cm of h4] (x4b) {$x^b_4$};
% 
%    \draw[-latex] (h4) -- (x4a);
%    \draw[-latex] (h4) -- (x4b);
% 
%    % Story - 
%   \begin{pgfonlayer}{background}
%   \draw<3>[draw=black,fill=green!70,rounded corners,line width=1pt, dotted] 
%                   ($(x2a.north west) + (135:0.3cm)$) -- 
%                   ($(x2b.south west) + (-135:0.3cm)$) -- 
%                   ($(x4a.south east) + (-45:0.3cm)$) -- 
%                   ($(h2.east) + (0:0.3cm)$) -- 
%                   ($(x2a.north east) + (45:0.3cm)$) -- 
%                   cycle;
%   %  \draw<4>[draw=black,fill=green!70,rounded corners,line width=1pt, dotted] 
%   %                  ($(x1a.north west) + (135:0.3cm)$) -- 
%   %                  ($(x1a.north east) + (45:0.3cm)$) -- 
%   %                  ($(h1.south east) + (-45:0.3cm)$) -- 
%   %                  ($(h1.south west) + (-135:0.3cm)$) -- 
%   %                  cycle;
%   \draw<6>[draw=black,fill=green!70,rounded corners,line width=1pt, dotted] 
%                   ($(x1b.north west) + (135:0.3cm)$) -- 
%                   ($(x3a.north east) + (45:0.3cm)$) -- 
%                   ($(h3.south east) + (-45:0.3cm)$) -- 
%                   ($(h1.south west) + (-135:0.3cm)$) -- 
%                   cycle;
%   \draw<9>[draw=black,fill=green!70,rounded corners,line width=1pt, dotted] 
%                   ($(h2.north west) + (135:0.3cm)$) -- 
%                   ($(x2b.south west) + (-135:0.3cm)$) -- 
%                   ($(x4a.south west) + (-135:0.3cm)$) -- 
%                   ($(x4b.south east) + (-45:0.3cm)$) -- 
%                   ($(x3b.north east) + (45:0.3cm)$) -- 
%                   ($(h3.north east) + (45:0.3cm)$) -- 
%                   cycle;
%   \end{pgfonlayer}
% 
%   \draw<4>[DarkGreen,-latex, line width=1.3pt] (h2) -- (x2a);
%   \draw<4>[DarkGreen,-latex, line width=1.3pt] (h2) -- (x2b);
%   \draw<5->[DarkGreen,-latex] (h2) -- (x2a);
%   \draw<5->[DarkGreen,-latex] (h2) -- (x2b);
% 
%   \draw<5>[DarkGreen,-latex,line width=1.3pt] (h1) -- (x1a);
%   \draw<5>[DarkGreen,-latex,line width=1.3pt] (h1) -- (x1b);
%   \draw<5>[DarkGreen,-latex,line width=1.3pt] (h3) -- (x3a);
%   \draw<5>[DarkGreen,-latex,line width=1.3pt] (h3) -- (x3b);
%   \draw<5>[DarkGreen,-latex,line width=1.3pt] (h4) -- (x4a);
%   \draw<5>[DarkGreen,-latex,line width=1.3pt] (h4) -- (x4b);
%   \draw<6->[DarkGreen,-latex] (h1) -- (x1a);
%   \draw<6->[DarkGreen,-latex] (h1) -- (x1b);
%   \draw<6->[DarkGreen,-latex] (h3) -- (x3a);
%   \draw<6->[DarkGreen,-latex] (h3) -- (x3b);
%   \draw<6->[DarkGreen,-latex] (h4) -- (x4a);
%   \draw<6->[DarkGreen,-latex] (h4) -- (x4b);
% 
% 
% 
%   \draw<7>[DarkGreen,-latex, line width=1.3pt] (h1) -- (h3);
%   \draw<8>[DarkGreen,-latex, line width=1.3pt] (h1) -- (h2);
%   \draw<8->[DarkGreen,-latex] (h1) -- (h3);
%   \draw<9->[DarkGreen,-latex] (h1) -- (h2);
% 
%   \draw<10>[DarkGreen,-latex, line width=1.3pt] (h2) -- (h4);
%   \draw<10>[DarkGreen,-latex, line width=1.3pt] (h3) -- (h4);
%   \draw<11->[DarkGreen,-latex] (h2) -- (h4);
%   \draw<11->[DarkGreen,-latex] (h3) -- (h4);
% 
%   \end{canvas}
% \end{frame}
% 
% \begin{frame}
%   \frametitle{More Bottlenecked Examples}
% 
%   \begin{canvas}
%     \drawhmm<1->{(-4cm, 2cm)};
%     \node at ($(start-hmm) + (0, 0.6cm)$) {Hidden Markov models};
%     \drawtree<1->{(2cm, 2cm)};
%     \node at ($(start-tree) + (0, 0.6cm)$) {Latent Tree models};
%     \drawnoisyor<3->{(0cm, -2cm)};
%     \node<3-> at ($(start-nor) + (0, 0.7cm)$) {Noisy Or (non-example)
%       ({\small \cite{halpern2013unsupervised}})
%     };
%   \end{canvas}
% 
% \end{frame}
% 
% 
% % Efficiency 1: EM (+diagram).
% \section{Combining moments with likelihood estimators}
% 
% \begin{frame}
% \frametitle{Outline} 
% \begin{canvas}
%   \overview{(-2cm,3cm)};
%   \node[right=0.5cm of obs-edge] {1. Solve bottlenecks};
%   \node[right=0.5cm of cond-edge] {\obj{
%   \alt<1>{\color{DarkGreen}}{}2a. Pseudoinverse \\
%   \uncover<2->{\color{DarkGreen} 2b. Composite likelihood}
%   }};
% \end{canvas}
% \end{frame}
% 
% 
% \begin{frame}
%   \frametitle{Convex marginal likelihoods}
%   \splitcolumn{%
%     \begin{itemize}
%       \item<1-> The MLE is statistically
%         most efficient, but usually non-convex. 
%       \item<2-> If we fix the conditional moments, $-\log \Pr(x)$ is convex in $\theta$.
%       \item<3-> No closed form solution, but a local method like EM is
%         guaranteed to converge to the global optimum.
%     \end{itemize}
%   }{%
%     \begin{canvas}
%       \point{stuff}{(2cm,2.5cm)};
%       \drawbridge<1->{(stuff)};
%       \node[scale=0.8] at ($(start-bridge) - (-1,2.5cm)$) {\obj{
%       \begin{align*}
%         \log \Pr(\bx) &= \log \sum_{h_1,h_2} 
%         \robustaltm<1>{
%         \mathmb{\Pr(\bx_1 | h_1) \Pr(\bx_2 | h_2)}
%         }{
%         \underbrace{\Pr(\bx_1 | h_1) \Pr(\bx_2 | h_2)}_{\text{known}} 
%         }
%         \mathmb{\Pr(h_1, h_2)}
%       \end{align*}
%       }};
%       \node<2->[scale=0.8,anchor=north] at ($(start-bridge) - (-1cm,3.0cm)$) {
%       \includegraphics[width=\textwidth,height=6cm,keepaspectratio]{figures/piecewise-objective.pdf}
%       };
%     \end{canvas}
%   }
% \end{frame}
% 
% \begin{frame}
%   \frametitle{Composite likelihoods}
%   \splitcolumn{%
%     \begin{itemize}
%       \item<1-> In general, the full likelihood is still non-convex. 
%       \item<2-> Consider {\em composite likelihood} on a subset of observed variables.
%       \item<4-> Can be shown that estimation with composite likelihoods is consistent (\cite{lindsay88composite}).
%       \item<5-> Asymptotically, the composite likelihood estimator is more efficient.
%     \end{itemize}
%   }{%
%     \begin{canvas}
%       \point{stuff}{(2cm,2.5cm)};
%       \drawhmm<1->{(stuff)};
%       \node<1-2>[scale=0.8] at ($(start-hmm) - (-0,2.5cm)$) {\obj{
%       \begin{align*}
%         \log \Pr(\bx_{123}) &= \log \sum_{h_1,h_2,\robustaltm<1>{h_3}{\mathmr{h_3}}} \underbrace{\Pr(\bx_1 \given h_1) \Pr(\bx_2 \given h_2)  \robustaltm<1>{\Pr(\bx_3 \given h_3)}{\mathmr{\Pr(\bx_3 \given h_3)}}}_{\text{known}} \\ 
%         &\quad \hphantom{\log \sum_{h_1,h_2,h_3} } 
%         \robustaltm<1>{\mathmb{\Pr(h_3 \given h_2)}}{\mathmr{\Pr(h_3 \given h_2)}} \mathmb{\Pr(h_1, h_2)}
%       \end{align*}
%       }};
% 
%       \begin{pgfonlayer}{background}
%       \draw<2->[draw=black,fill=green!70,rounded corners,line width=1pt, dotted] 
%                       ($(h1.west) + (180:0.3cm)$) -- 
%                       ($(h1.north) + (90:0.3cm)$) -- 
%                       ($(h2.north) + (90:0.3cm)$) -- 
%                       ($(h2.east) + (0:0.3cm)$) -- 
%                       ($(x2.east) + (0:0.3cm)$) -- 
%                       ($(x2.south) + (-90:0.3cm)$) -- 
%                       ($(x1.south) + (-90:0.3cm)$) -- 
%                       ($(x1.west) + (180:0.3cm)$) -- 
%                       cycle;
%       \end{pgfonlayer}
% 
%       \node<3->[scale=0.8] at ($(start-hmm) - (-1,2.5cm)$) {\obj{
%       \begin{align*}
%         \log \Pr(\bx_{12}) &= \log \sum_{h_1,h_2} \underbrace{\Pr(\bx_1 \given h_1) \Pr(\bx_2 \given h_2)}_{\text{known}} \\ 
%         &\quad \hphantom{\log \sum_{h_1,h_2} } \mathmb{\Pr(h_1, h_2)}
%       \end{align*}
%       }};
%       \begin{pgfonlayer}{background}
%       \node<5->[scale=0.8,anchor=north] at ($(start-bridge) - (-1cm,3.3cm)$) {
%       \includegraphics[width=\textwidth,height=6cm,keepaspectratio]{figures/asymp-k2d5.pdf}
%       };
%       \end{pgfonlayer}
%     \end{canvas}
%   }
% \end{frame}
% 
% \begin{frame}
% \frametitle{Outline} 
% \begin{canvas}
%   \overview{(-2cm,3cm)};
%   \node[right=0.5cm of obs-edge] {1. Solve bottlenecks};
%   \node[right=0.5cm of cond-edge] {\obj{
%   2a. Pseudoinverse \\
%   \alt<1>{\color{DarkGreen}}{}2b. Composite likelihood
%   }};
%   \uncover<2->{
%   \node[right=0.5cm of params-edge] {\obj{
%   \color{DarkGreen}
%   3a. Renormalization \\
%   3b. Convex optimization
%   }};
%   }
% \end{canvas}
% \end{frame}
% 
% \section{Recovering parameters}
% 
% \begin{frame}
%   \frametitle{Recovering parameters in directed models}
%   \splitcolumn{%
%   \begin{itemize}
%     \item Conditional probability tables are the default
%       for a directed model. 
%     \item Can be recovered by normalization:
%       \begin{align*}
%         \Pr(h_2 \given h_1) &= \frac{\Pr(h_1, h_2)}{\sum_{h_2} \Pr(h_1, h_2)}.
%         \end{align*}
%   \end{itemize}
%   }{%
%     \begin{canvas}
%       \point{stuff}{(2cm,0cm)};
%       \drawbridge<1->{(stuff)};
%       \node[scale = 0.5, anchor=south] at (h1h2) {
%         \begin{tabular}{r | l l}
%           \diaghead{aaaaaa}{$h_2$}{$h_1$} &
%           \thead{$0$} & \thead{$1$} \\ \hline 
%           $0$ & \quad & \quad \\ 
%           $1$ & \quad & \quad 
%         \end{tabular}
%       };
%     \end{canvas}
%   }
% \end{frame}
% 
% \begin{frame}
%   \frametitle{Recovering parameters in undirected log-linear models}
%   %\fontsize{8pt}{8.2pt}\selectfont
% 
%   %\splitcolumn{%
%     \begin{itemize}
%       \item<1-> Assume a log-linear parameterization,
%         \begin{align*}
%           p_\theta(\bx, \bh) &= \exp\left( \sum_{\sC \in \sG} \theta^\top \phi(\bx_\sC,\bh_\sC) - A(\theta) \right).
%         \end{align*}
%       \item<2-> The {\em unsupervised} negative log-likelihood is non-convex,
%           \begin{align*}
%             \sL_\text{unsup}(\theta) \eqdef \E_{\bx \sim \sD}[- \log \mathmr{ \sum_{\bh \in \sH} p_\theta(\bx,\bh)} ].
%           \end{align*}
%       \item<3-> However, the {\em supervised} negative log-likelihood is convex,
%           \begin{align*}
%           \sL_\text{sup}(\theta) &\eqdef \E_{(\bx,\bh) \sim \sD_\text{sup}}\left[- \log p_\theta(\bx,\bh) \right] \\
%           &= -\mathmb{\theta^\top} \left(\sum_{\sC \in \sG} \E_{(\bx,\bh) \sim \sD_\text{sup}}[\phi(\bx_\sC,\bh_\sC)]\right) + \mathmb{A(\theta)}.
%           \end{align*}
%     \end{itemize}
%   %}{%
%   %  \begin{canvas}
%   %    \point{stuff}{(2cm,0cm)};
%   %    \drawubridge<1->{(stuff)};
%   %    \node[scale = 1.0, anchor=south] at (h1h2u) {$\theta$};
%   %  \end{canvas}
%   %}
% \end{frame}
% 
% \begin{frame}
%   \frametitle{Recovering parameters in undirected log-linear models}
%   %\fontsize{8pt}{8.2pt}\selectfont
% 
%   %\splitcolumn{%
%     \begin{itemize}
%       \item<1-> Recall, the marginals can typically estimated from
%         supervised data. 
%           \begin{align*}
%           \label{eqn:logLinearSupervised}
%           \sL_\text{sup}(\theta) &= -\mathmb{\theta^\top} \mathmg{\underbrace{\left(\sum_{\sC \in \sG} \E_{(\bx,\bh) \sim \sD_\text{sup}}[\phi(\bx_\sC,\bh_\sC)]\right)}_{\mu_\sC}} + \mathmb{A(\theta)}.
%           \end{align*}
%         \item<2-> However, the marginals can also be {\em consistently}
%           estimated by moments!
%         \begin{align*}
%           \mu_\sC &= \sum_{\bx_\sC, \bh_\sC} \underbrace{\mathmg{\Pr(\bx_\sC \given \bh_\sC)}}_{\textmg{cond. moments}} 
%           \underbrace{\mathmb{\Pr(\bh_\sC)}}_{\textmb{hidden marginals}} \phi(\bx_\sC,\bh_\sC).
%         \end{align*}
%     \end{itemize}
%   %}{%
%   %  \begin{canvas}
%   %    \point{stuff}{(2cm,0cm)};
%   %    \drawubridge<1->{($(stuff) + (0,0cm)$)};
%   %    \node[scale = 1.0, anchor=south] at (h1h2) {$\theta$};
%   %  \end{canvas}
%   %}
% \end{frame}
% 
% \begin{frame}
%   \frametitle{Optimizing pseudolikelihood}
%   %\fontsize{8pt}{8.2pt}\selectfont
% 
%   \splitcolumn{%
%     \begin{itemize}
%         \item<1-> Estimating $\mu_\sC$: independent of treewidth. 
%         \item<2-> Computing $A(\theta)$: dependent on treewidth.
%           \begin{align*}
%             A(\theta) &\eqdef \log \sum_{\bx, \bh} \exp\left(\theta^\top \phi(\bx, \bh) \right).
%           \end{align*}
%         \item<3-> Instead, use pseudolikelihood (\cite{besag75pseudo})
%           to consistently estimate distributions over local
%           neighborhoods. 
%           \begin{align*}
%             A_{\text{pseudo}}(\theta; \sN(a)) &\eqdef \log \sum_{a} \exp\left(\theta^\top \phi(\bx_\sN, \bh_\sN) \right).
%           \end{align*}
%         %\item<4-> Clique marginals not sufficient statistics, but we can still estimate them.
%     \end{itemize}
%   }{%
%     \begin{canvas}
%       \point{stuff}{(2cm,0cm)};
%       \node at ($(stuff) + (1cm,0cm)$) {
%       \includegraphics[width=0.95\textwidth,height=4cm,keepaspectratio]{figures/mrf.pdf}
%       };
%     \end{canvas}
%   }
% \end{frame}
% 
% \begin{frame}
% \frametitle{Outline} 
% \begin{canvas}
%   \overview{(-2cm,3cm)};
%   \node[right=0.5cm of obs-edge] {1. Solve bottlenecks};
%   \node[right=0.5cm of cond-edge] {\obj{
%   2a. Pseudoinverse \\
%   2b. Composite likelihood
%   }};
%   \node[right=0.5cm of params-edge] {\obj{
%   \color{DarkGreen}
%   3a. Renormalization \\
%   3b. Convex optimization
%   }};
% \end{canvas}
% \end{frame}
% 
% \section{Conclusions}
% 
% \begin{frame}
%   \frametitle{Conclusions}
%   %\begin{itemize}
%   %    \item Uniformly bottlenecked models
%   %    \item Scales with the size of each clique, not the tree-width
%   %    \item Solving bottlenecks breaks problem into convex pieces; can be solved more accurately
%   %    \item The marginals make the log-linear recovery problem convex.
%   %\end{itemize}
%   \splitcolumn{%
%     \begin{itemize}
%      % \item {\em Before our work}
%      % \begin{itemize}
%      %   \item Gaussian Mixture Models \tikzmark{gmm}
%      %   \item Hidden Markov Models 
%      %   \item Latent Dirichlet Allocation
%      % \end{itemize}
%      \item \todo{Use outline slide}.
%      \item \todo{Show the venn diagram on progress on generality.}.
%      \item<1-> An algorithm for any (non-degenerate) {\bf bottlenecked discrete graphical models}. \tikzmark{grid}
%     \item<2-> Efficiently learns models with {\bf high-treewidth}.
%     \item<3-> Combine moment estimators with composite
%       likelihood estimators.
%     \item<4-> Extends to {\bf log-linear models}.
%       \begin{itemize}
%         \item Allows for easy regularization, missing data, etc.
%       \end{itemize}
%     \end{itemize}
%   }{%
%   \begin{canvas}
%     \point{mark}{(4cm,0)};
%     %\point{gmm}{({pic cs:gmm} -| mark)};
%     %\point{grid}{({pic cs:grid} -| mark)};
%     %\point{gmm}{($(mark) + (0,3cm)$)};
%     \point{grid}{($(mark) + (0,1cm)$)};
% 
% %    \node[anchor=south west] (mog) at (gmm) {%
% %      \includegraphics[width=0.45\textwidth,height=3cm,keepaspectratio]{figures/mog.pdf}
% %    };
% 
%     %\drawgen{($(gmm) + (0,0.0cm)$)};
%     \drawgrid{(grid)};
%   \end{canvas}
%   }
% 
% 
% %  \cornertext<1->{\cite{AnandkumarGeHsu2012}}
% %
% %  \begin{canvas}
% %    % Tasks.
% %
% %    \node<1->[anchor=west] (diag) at (-3cm, 1cm) {%
% %      \includegraphics[width=0.45\textwidth,height=3cm,keepaspectratio]{figures/mog.pdf}
% %    };
% %    %\drawgen{(-3cm,1cm)}
% %    \node[below=0.6cm of diag.south] {Before};
% %
% %    % Highlight
% %    \draw<2>[scale=0.8,fill=green,opacity=0.4,dashed] (1cm,2.5cm) rectangle (6.5cm,-2.5cm);
% %      \drawgrid{(3cm,1cm)}
% %      \node[below=0.1cm of h4.south] {After};
% %  \end{canvas}
% 
% \end{frame}
% 
% 
% 
% \begin{frame}
%   \frametitle{}
%     Thank you!
% \end{frame}

\end{document}

