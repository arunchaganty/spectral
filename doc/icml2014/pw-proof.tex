\subsection{\lemmaref{mom-pw-variance}}
\label{app:pw-proof}

In \sectionref{piecewise}, we compared the asymptotic variance
  $\Sigma^\ml_S$ of the composite likelihood estimator
  with that of the pseudoinverse estimator, $\Sigma^\mom_S$,
  for a subset of hidden variables $S$.
  Now we will derive these asymptotic variances in detail.
%In this part, we will derive the asymptotic variances of these two
  %estimators and compare their relative efficiencies.

Recall, that in \sectionref{piecewise} we simplified notation by taking
  $m=1$ and flattening the moments $M_\sV$ and hidden marginals $Z_S$
  into vectors $\mu \in \Re^d$ and $z \in \Re^k$ respectively. The
  conditional moments, $O$, is a now matrix $O \in \Re^{d\times k}$ and
  the hidden marginals $z$ and observed marginals $\mu$ are related via
  $\mu = O z$.
%
%Recall also that we reparameterized $z \in \Delta_{k-1}$ and $\mu \in
%  \Delta_{d-1}$ using $\tz \in \Re^{k-1}$ and $\tm \in \Re^{d-1}$ to avoid
%  constraints:
%\begin{align*}
%  \mu &= 
%    \begin{bmatrix}
%      \tm \\
%      1 - \ones^\top\tm
%    \end{bmatrix} 
%  &
%  z &= 
%    \begin{bmatrix}
%      \tz \\
%      1 - \ones^\top\tz
%    \end{bmatrix}.
%\end{align*}
%
%Finally, we showed that $\tz$ and $\tm$ are related as,
%\begin{align*}
%  \tm &= \tO \tz +  O_{\neg d,k},
%\end{align*}
%where $\tO \eqdef O_{\neg d,\neg k} - O_{\neg d, k} \ones^\top$.
%
%We also describe $\mu$ in terms of $\tz$,
%\begin{align*}
%  \mu 
%  &= 
%    \begin{bmatrix}
%      \tm \\
%      1 - \ones^\top\tm
%    \end{bmatrix} \\
%    &=
%    \begin{bmatrix}
%      \tO \\
%      - \ones^\top\tO
%    \end{bmatrix} \tz 
%    + 
%    \begin{bmatrix}
%      O_{\neg d, k} \\
%      1 - \ones^\top O_{\neg d, k}
%    \end{bmatrix}.
%\end{align*}
%
%With this reparameterization, we are ready to derive the asymptotic
%variances of the two estimators.
%
\begin{lemma*}[Asymptotic variances]
  The asymptotic variances of the pseudoinverse estimator $\htz^\mom$
  and composite likelihood estimator $\htz^\ml$ are:
  \begin{align*}
    \Sigmamom 
      &= \tOi (\tD - \tm \tm^\top) \tOit, \\
    \Sigmaml 
      &=
      \left( \tOt (\tD\inv + \td\inv \ones\ones^\top) \tO \right)\inv,
  \end{align*}
  where $\tD \eqdef \diag(\tm)$ and $\td \eqdef 1 - \ones^\top\tm$.
\end{lemma*}
\begin{proof}[Proof for \lemmaref{mom-pw-variance}]
  First, let us look at the
  asymptotic variance of the pseudoinverse estimator
  $\hat{z}^\mom = \tOi (\htm - O_{\neg d,k})$. 
  Note that $\hat\mu = \frac1n\sum_{i=1}^n x_i$, where each $x_i$ is an independent draw from the multinomial distribution $\mu$.
  Hence the variance of
    $\hat\mu$ is $(D - \mu\mu^\top)$ where $D \eqdef \diag(\mu)$.
  Recall that $\htm$ is just the first $d-1$ entries of $\hat\mu$, so
  the variance of
    $\htm$ is $(\tD - \tm \tm^\top)$ where $\tD \eqdef \diag(\tm)$.
  Since $\tz$ is just a linear transformation of $\tm$,
  the asymptotic variance of $\htz^\mom$ is:
  \begin{align*}
      \Sigmamom &= \tOi \Var(\htm) \tOit \\
      &= \tOi (\tD - \tm \tm^\top) \tOit.
      %&= \tOi \tD \tOit - \tOi \tD \ones \ones^\top \tD \tOit,
  \end{align*}
  
  Now, let us look at the variance of the composite likelihood estimator.  Using
  the delta-method \cite{vaart98asymptotic} we have that the asymptotic
  variance of 
  %$\widehat{z}^\ml = \argmin_{\tz \in [0,1]^{k-1}} \hat\E[-\ell(x)]$ is,
  % Definitely not optimizing over the whole set
  $\htz^\ml = \argmax_{\tz} \hat\E[\ell(x ; \tz)]$ is,
  \begin{align*}
    \Sigmaml &= \E[\grad^2 \ell(x ; \tz^*)]^{-1} \Var[\grad \ell(x ; \tz^*)] \E[\grad^2 \ell(x ; \tz^*)]^{-1},
  \end{align*}
  where $\ell(x;\tz)$ is the log-likelihood of the observations $x$
  given parameters $\tz$. We can write $\ell(x;\tz)$ in terms of $\tz$
  and $\tO$ as,
  \begin{align*}
    \ell(x;\tz) 
              &= \log( \mu[x] ) \\
              &= \log \left( 
    e_x^\top \begin{bmatrix}
      \tO \\
      - \ones^\top\tO
    \end{bmatrix} \tz 
    + 
    e_x^\top \begin{bmatrix}
      O_{\neg d, k} \\
      1 - \ones^\top O_{\neg d, k}
    \end{bmatrix}
    \right),
 %             
 %             \BI[x \neq d] \tm[x] + \BI[x = d] (1 - \ones^\top \tm ) )\\
 %             &= \log( \BI[x \neq d] (\tO[x] \tz + O_{\neg d, k}[x]) \\
 %             &\quad + \BI[x = d] (1 - \ones^\top (\tO \tz + O_{\neg d, k})) ).
  \end{align*}
where $e_x$ is an indicator vector on $x$.

Taking the first derivative,
\begin{align}
  \grad \ell(x; \tz)
  &= 
  \frac{
  1}{\mu[x]}
  \begin{bmatrix}
    \tO \\
    -\ones^\top \tO
  \end{bmatrix}^\top
 e_x \nonumber \\
  &= 
  \begin{bmatrix}
    \tO \\
    -\ones^\top \tO
  \end{bmatrix}^\top
  D^{-1} e_x, \label{eqn:lhood-grad}
\end{align}
where $D \eqdef \diag(\mu)$.

It is easily verified that the expectation of the first derivative is indeed $\zeros$:
%Note that the expectation of the first derivative is indeed $\zeros$,
\begin{align*}
  \E[\grad \ell(x; \tz)]
  &= 
  \begin{bmatrix}
    \tO \\
    -\ones^\top \tO
  \end{bmatrix}^\top
  D^{-1} \E[e_x] \\
  &= 
  \begin{bmatrix}
    \tO \\
    -\ones^\top \tO
  \end{bmatrix}^\top
  D^{-1} \mu \\
  &= 
  \begin{bmatrix}
    \tO \\
    -\ones^\top \tO
  \end{bmatrix}^\top
  \ones \\
  &= \tOt\ones - \tOt\ones \\
  &= \zeros.
\end{align*}

Taking the second derivative,
\begin{align}
  \grad^2 \ell(x; \tz)
  &= 
  \frac{
  1}{\mu[x]^2}
  \begin{bmatrix}
    \tO \\
    -\ones^\top \tO
  \end{bmatrix}^\top
 e_x e_x^\top 
  \begin{bmatrix}
    \tO \\
    -\ones^\top \tO
  \end{bmatrix}
  \nonumber
 \\ 
  &= 
  \begin{bmatrix}
    \tO \\
    -\ones^\top \tO
  \end{bmatrix}^\top
    D^{-1} e_x e_x^\top D^{-1} 
  \begin{bmatrix}
    \tO \\
    -\ones^\top \tO
  \end{bmatrix}. \label{eqn:lhood-hess}%
\end{align}

From \equationref{lhood-grad} and \equationref{lhood-hess}, we get
  \begin{align*}
    \E[\grad^2 \ell(x; \tz^*)] 
    &= -
        \begin{bmatrix}
          \tO \\
          -\ones^\top \tO
        \end{bmatrix}^\top
          D^{-1} \E[e_x e_x^\top] D^{-1} 
        \begin{bmatrix}
          \tO \\
          -\ones^\top \tO
        \end{bmatrix} \\
    \Var [\grad \ell(x; \tz^*)] 
    &= \hphantom{-}
        \begin{bmatrix}
          \tO \\
          -\ones^\top \tO
        \end{bmatrix}^\top
          D^{-1} \E[e_x e_x^\top] D^{-1} 
        \begin{bmatrix}
          \tO \\
          -\ones^\top \tO
        \end{bmatrix} \\
    &= \hphantom{-}
        \begin{bmatrix}
          \tO \\
          -\ones^\top \tO
        \end{bmatrix}^\top
          D^{-1} D D^{-1} 
        \begin{bmatrix}
          \tO \\
          -\ones^\top \tO
        \end{bmatrix} \\
    &= \hphantom{-}
        \begin{bmatrix}
          \tO \\
          - \ones^\top \tO
        \end{bmatrix}^\top
        \begin{bmatrix}
          \tD\inv & \zeros \\
          \zeros^\top & \td\inv
        \end{bmatrix}
        \begin{bmatrix}
          \tO \\
          - \ones^\top \tO
        \end{bmatrix} \\
        &= \hphantom{-} 
          \tOt \tD\inv \tO + \td\inv \tOt \ones \ones^\top \tO,
  \end{align*}
where $\tD = \diag(\tm)$ and $\td = 1 - \ones^\top \tm$ are the diagonal
  elements of $D$.
As expected, $\E[\grad^2 \ell(x)] = -\Var [\grad \ell(x)]$ because
$\hat{z}$ is a maximum likelihood estimator. 

Finally, the asymptotic variance of $\Sigmaml$ is,
\begin{align*}
    \Sigmaml &= \E[\grad^2 \ell(x ; \tz^*)]^{-1} \Var[\grad \ell(x ; \tz^*)] \E[\grad^2 \ell(x ; \tz^*)]^{-1} \\
      &= \Var[\grad \ell(x;\tz^*)]^{-1} \\
      &= \left( \tOt \tD\inv \tO + \td\inv \tOt \ones \ones^\top \tO \right)\inv.
      %&= \left( \tOt \tD\inv \tO  \right)\inv - 
      %    \frac{\left( \tOt \tD\inv \tO \right)\inv  \tOt \ones \ones^\top \tO \left( \tOt \tD\inv \tO \right)\inv}
      %    {\td + \ones^\top \tO \left( \tOt \tD\inv \tO \right)\inv \tOt \ones}.
\end{align*}

Given our assumptions, $\ones \succ \mu \succ \zeros$. Consequently,
$\tD$ is invertible and the asymptotic variance is finite.
%and our estimator is consistent as well. 
\end{proof}

