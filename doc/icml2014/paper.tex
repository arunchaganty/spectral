\documentclass{article}

% use Times
\usepackage{times}
% For figures
\usepackage{graphicx} % more modern
\usepackage{subfigure} 
\usepackage{amsmath,amssymb,amsthm} 

\usepackage{datetime}
\usepackage[utf8]{inputenc}

% For citations
\usepackage{natbib}

\usepackage{mathtools}
\usepackage{scabby}
\usepackage{scabby-diag}
\usepackage{standalone}
\usepackage{import}

% For algorithms
\usepackage{algorithm}
\usepackage{algorithmic}

% As of 2011, we use the hyperref package to produce hyperlinks in the
% resulting PDF.  If this breaks your system, please commend out the
% following usepackage line and replace \usepackage{icml2014} with
% \usepackage[nohyperref]{icml2014} above.
\usepackage{hyperref}

% Packages hyperref and algorithmic misbehave sometimes.  We can fix
% this with the following command.
%\newcommand{\theHalgorithm}{\arabic{algorithm}}

% Employ the following version of the ``usepackage'' statement for
% submitting the draft version of the paper for review.  This will set
% the note in the first column to ``Under review.  Do not distribute.''
\usepackage{icml2014} 
% Employ this version of the ``usepackage'' statement after the paper has
% been accepted, when creating the final version.  This will set the
% note in the first column to ``Proceedings of the...''
%\usepackage[accepted]{icml2014}


% The \icmltitle you define below is probably too long as a header.
% Therefore, a short form for the running title is supplied here:
%\icmltitlerunning{Submission and Formatting Instructions for ICML 2014}

\providecommand{\Pa}{\textrm{Pa}}
\providecommand{\TensorFactorize}{\textsc{TensorFactorize}~}
\providecommand{\LearnFactors}{\textsc{LearnFactors}~}


\begin{document} 

%\newcommand\paperTitle{Moment Constraints Make Learning Latent-Variable Models Easier}
\newcommand\paperTitle{Estimating Latent-Variable Graphical Models using Moments and Likelihoods}

\twocolumn[
\icmltitle{\paperTitle}

% It is OKAY to include author information, even for blind
% submissions: the style file will automatically remove it for you
% unless you've provided the [accepted] option to the icml2014
% package.
\icmlauthor{Arun Tejasvi Chaganty}{chaganty@cs.stanford.edu}
\icmlauthor{Percy Liang}{pliang@cs.stanford.edu}
\icmladdress{Stanford University,
Stanford, CA, USA}

% You may provide any keywords that you 
% find helpful for describing your paper; these are used to populate 
% the "keywords" metadata in the PDF but will not be shown in the document
\icmlkeywords{machine learning}

\vskip 0.3in
]

\subimport{}{abstract}
\subimport{}{intro}
\subimport{}{background}
\subimport{}{directed}
\subimport{}{piecewise}
\subimport{}{undirected}
\subimport{}{examples}
\subimport{}{related}
%\subimport{}{experiments}
\subimport{}{discussion}

\bibliography{pliang,ref}
\bibliographystyle{icml2014}

\twocolumn
\appendix
\subimport{}{appendix}

\end{document} 

