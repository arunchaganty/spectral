
%We will now instantiate our algorithm for a couple more examples, illustrated in \figureref{examples}.

%\paragraph{Aggregating observations}
%We describe two practical considerations to use samples more efficiently.
%Firstly, if we have multiple exclusive views for a hidden variable,
%  intuitively, it is better to aggregate over them. 
%For example, consider a hidden variable $h_1$ with multiple exclusive
%  views $x_1$ and $x_2$.
%With the method of moments perspective, to learn the marginal
%  distribution over $h_1$, one must solve the following reconstruction
%  problem, 
%\begin{align*}
%  \hat Z_{h_1} &= \arg\min_{Z_{h_1}} \half \|Z_{h_1}(\mOpp{1}{1}) - M_1 \|^2 + \half \|Z_{h_1}(\mOpp{2}{1}) - M_2 \|^2,
%\end{align*}
%which does not have a closed form solution. 
%In contrast, this naturally fits into the convex optimization framework, where $\hat Z_{h_1}$ will now be,
%\begin{align*}
%  \hat Z_{h_1} &= \arg\min_{Z_{h_1}} \sum_{\vx \in \sD} \log Z_{h_1}( \mOpp{1}{1}[x_1] \odot \mOpp{1}{1}[x_1] ),
%\end{align*}
%where $\cdot$ denotes element-wise multiplication.
%An important note to make is that \TensorFactorize only returns
%  a solution up to permutation; if $\mOpp{1}{1}$ and
%  $\mOpp{1}{2}$ do not belong to the same bottleneck (e.g. $x^a_2$ and
%  $x^b_2$ in \figureref{examples-tree}), then some
%  care must be taken to ensure they have the same labelling.
%
%Secondly, we can exploit parameter sharing by aggregating over
%  disjoint sets of observed variables. 
%For example, in \figureref{examples-hmm}, we can aggregate the statistics for
%  bottlenecks $h_i$ with views $\{x_{i-1}, x_{i}, x_{i+1}\}$ before
%  running \TensorFactorize; this will give us a consistent estimate for
%  $O$ (as well as $T$).

