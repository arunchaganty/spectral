\section{Setup}
\label{sec:setup}

Let $\sG$ be a discrete graphical model with variables $V$, 
  let the subset $X \subseteq V$ be observed and 
  let $H = V \setminus O$ be hidden.
Let $\Pa(v) \subset V$ be the parents of a variable $v$ in a directed
  model, and $\sN(v) \subset V$ be the neighbours of $v$ in an undirected
  model.
We will use $x_1, x_2, \cdots, x_N \in X$ to denote observed variables and
  $h_1, h_2, \cdots, h_M \in H$ to denote hidden variables.
For ease of exposition, we assume that $x_i \in \{1, \cdots, d\}$ for
  all $i$, and that $h_i \in \{1, \cdots, k\}$ for all i.

We assume that the model $\sG$ is parameterized at every clique $\sC$.
If $\sG$ is directed, the parameters correspond to
  the conditional probability tables for each clique.
If $\sG$ is undirected, the parameters correspond to the clique
  potentials.

\paragraph{Problem statement}

% Statement
This paper focuses on the problem of parameter estimation:
We are given $n$ i.i.d.~examples of the observed variables $\sD
  = (\vx^{(1)}, \dots, \vx^{(n)})$ where each $\vx^{(i)} \sim p_{\theta^*}$ for
  some true parameters $\theta^*$.
Our goal is to produce a parameter estimate $\hat\theta$ that
  approximates $\theta^*$.

% Maximum likelihood
The standard estimation procedure is maximum (marginal) likelihood,
  \begin{align*}
    \sL &\eqdef \max_{\theta} \sum_{\vx \in \sD} \log p_\theta(\vx) \\
        &=      \max_{\theta} \sum_{\vx \in \sD} \log \sum_{\vh} p_\theta(\vx,\vh)
  \end{align*}
  which is statistically efficient but computationally intractable
  because we must marginalize over latent variables $z$.
In practice, one uses gradient-based optimization procedures (e.g., EM
  or L-BFGS) on the marginal likelihood, which can get stuck in local
  optima.

\paragraph{Notation}

We use $\otimes$ to denote the tensor product, e.g if $u \in \Re^k,
  v \in \Re^d$, then $u \otimes v \in \Re^{k \times d}$.
For a $\ell$-th order tensor $T \in \Re^{d \times \ldots \times
  d}$ and vectors $x_1, \cdots, x_\ell \in \Re^{d},$, we define 
\begin{align*}
  T(x_1, \cdots, x_\ell) 
      &= \sum_{i_1 = 1}^{d} \cdots \sum_{i_\ell = 1}^{d} 
            T_{i_1 i_2 \ldots i_l} ({x_1})_{i_1} \cdots ({x_\ell})_{i_\ell}.
\end{align*}
Analogously, for a matrices $M_1 \in \Re^{d \times k}, \cdots,
  M_\ell \in \Re^{d \times k}$, we define
\begin{align*}
  T(M_1, \cdots, M_\ell)_{j_1, \ldots, j_l} 
      &= \sum_{i_1 = 1}^{d} \cdots \sum_{i_\ell = 1}^{d} 
            T_{i_1 i_2 \ldots i_l} ({M_1})_{i_1j_1} \cdots ({M_\ell})_{i_\ell j_\ell}.
\end{align*}

To denote the moments of observed variables, we will use the notation
\begin{align*}
  \mO_i &\eqdef \Pr(x_i), &
  \mO_{ij} &\eqdef \Pr(x_i, x_j), &
  \mO_{ijk} &\eqdef \Pr(x_i, x_j, x_k),
\end{align*}
and so on.
We will view $\mO_i, \mO_{ij}, \mO_{ijk}, \cdots$ as tensors of
  order 1, 2, 3, \ldots in $\Re^d, \Re^{d\times d}, \Re^{d \times
  d \times d},\ldots$.
For the moments of hidden variables, we will use 
\begin{align*}
  \mH_i &\eqdef \Pr(h_i), &
  \mH_{ij} &\eqdef \Pr(h_i, h_j), &
  \mH_{ijk} &\eqdef \Pr(h_i, h_j, h_k).
\end{align*}


\paragraph{Estimating three-view mixture models}

\begin{figure}[t]
  \label{fig:three-view}
  \centering
  % vim:ft=tex
\documentclass[tikz,convert={outfile=figure\three-view.pdf}]{standalone}
%\usetikzlibrary{...}% tikz package already loaded by 'tikz' option
\begin{document}

\begin{tikzpicture}

% Hidden nodes
   \node[style=node, scale=0.8] (h) at (0,0) {$h$};
   \node[style=obsnode, scale=0.8] (x1) at (-1.0, -1.0) {$x_1$};
   \node[style=obsnode, scale=0.8] (x2) at (0.0, -1.0) {$x_3$};
   \node[style=obsnode, scale=0.8] (x3) at (1.0, -1.0) {$x_3$};

   \draw[-latex] (h1) -- (x1);
   \draw[-latex] (h1) -- (x2);
   \draw[-latex] (h1) -- (x3);
\end{tikzpicture}

\end{document}

  \caption{A three-view mixture model}
\end{figure}

Recently, \citet{anandkumar12moments} presented a consistent algorithm
  using the method of moments to learn the parameters of a three-view
  mixture model (\figureref{three-view}), an important case of directed
  graphical models, that sidesteps the local optima problem altogether. 
In the model, we have three observed variables $x_1, x_2, x_3$
  that are independent conditioned on a latent variable $h$.  
The parameters of this model are the mixture probabilities $\pi \eqdef
  \Pr(h) \in \Delta_{k-1}$ and the conditional moments $O^{(i)} \eqdef
  \Pr(x_i \given h) \in \Re^{d \times k}$ for $i \in \{1,2,3\}$.

  The key idea is that the observed moments $\mO_{12}, \mO_{23}, \mO_{13}$ and
  $\mO_{123}$ admit a factorization in terms of parameters;
\begin{align*}
  \mO_{ij} &= \sum_h \pi_h {O\oft i}^T_h \otimes {O\oft j}^T_h & \forall i,j \in \{1,2,3\} \\
  \mO_{123} &= \sum_h \pi_h {O\oft 1}^T_h \otimes {O\oft 2}^T_h \otimes  {O\oft 3}^T_h.
\end{align*}

\citet{anandkumar13tensor} presented a robust tensor power method, which
  we call $\TensorFactorize$, which converts consistent estimates of the observed moments 
  into consistent estimates of the parameters, $\pi, O\oft{1},
  O\oft{2}, O\oft{3}$ (up to permutation of the columns), provided
  the following assumption holds:

\begin{assumption}
\label{asm:full-rank}
Assume the conditional moments $O\oft{1}, O\oft{2}, O\oft{3}$ have full column rank
  $k$, and that $\pi \succ 0$.
\end{assumption}

In brief, the algorithm first computes a whitening matrix $W \in
  \Re^{d \times k}$ such that $W^T \mO_{12} W = I_{k \times k}$,
  and uses it to transform $\mO_{123}$ into a symmetric orthogonal tensor.  
Then a robust tensor power method is applied to extract the eigenvectors
  of the whitened $\mO_{123}$; unwhitening yields the columns of $\mO_{3}$.
$\mO_1$ and $\mO_2$ can then be recovered with some algebra.

