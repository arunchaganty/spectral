\section{Setup}
\label{sec:setup}

Let $\sG$ be a discrete graphical model with variables $V$, of which
  $O \subseteq V$ be observed and $H = V \setminus O$ be hidden.
Let $\Pa(x) \subset V$ be the parents of a variable $x$ in a directed
  model, and $\sN(x) \subset V$ be the neighbours of $x$ in an undirected
  model.
We will use $o_1, o_2, \cdots \in O$ to denote observed variables and
  $h_1, h_2, \cdots \in H$ to denote hidden variables.
For ease of exposition, we assume that the domain of all the observed
  variables is $[D]$ and the domain of all the hidden variables is
  $[K]$.

Define $\otimes$ to be the tensor product. In matrix representation, it
  is the Kronecker product.\expand

\paragraph{Problem statement}

% Statement
This paper focuses on the problem of parameter estimation:
We are given $n$ i.i.d.~examples of the observed variables $D = (x^{(1)}, \dots, x^{(n)})$
where each $x^{(i)} \sim p_{\theta^*}$ for some true parameters $\theta^*$.
Our goal is to produce a parameter estimate $\hat\theta$ that approximates $\theta^*$.

% Maximum likelihood
The standard estimation procedure is maximum (marginal) likelihood,
  \begin{align*}
    \sL &\eqdef \max_{\theta \in \Re^d} \E_{x} \log p_\theta(x) \\
        &=      \max_{\theta \in \Re^d} \E_{x} \log \sum_z p_\theta(z),
  \end{align*}
  which is statistically efficient but computationally intractable
  because we must marginalize over latent variables $z$.
In practice, one uses gradient-based optimization procedures (e.g., EM
  or L-BFGS) on the marginal likelihood, which can get stuck in local
  optima.

\paragraph{Estimating three-view mixture models}

\citet{anandkumar12moments} present an algorithm to solve three-view mixture models.

Assume means $M_1, M_2, M_3$. Observe that the moments of the model are $M_{123}$.

Applying a whitening transformation and get a orthogonal tensor decomposition.

Solved using the robust tensor power method \citet{anandkumar13tensor}.

In general, we call hidden variables with this property ``bottlenecks'',
  and define them as follows.
\begin{definition}(Bottleneck)
  A hidden variable $h$ is said to be a bottleneck if there exists at
  least three conditionally independent observed variables $o_1, o_2,
  o_3$. Let $\sB(h)$ be the set of conditionally independent observed
  variables for $h$.
\end{definition}

\begin{assumption}
  We assume that $M_1, M_2, M_3$ are each full rank, and that $\pi \succ
  0$\reword.
\end{assumption}

