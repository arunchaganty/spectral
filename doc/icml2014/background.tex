\section{Setup}
\label{sec:setup}

Let $\sG$ be a discrete graphical model with variables $V$, of which
  $O \subseteq V$ be observed and $H = V \setminus O$ be hidden.
Let $\Pa(x) \subset V$ be the parents of a variable $x$ in a directed
  model, and $\sN(x) \subset V$ be the neighbours of $x$ in an undirected
  model.
We will use $o_1, o_2, \cdots \in O$ to denote observed variables and
  $h_1, h_2, \cdots \in H$ to denote hidden variables.
For ease of exposition, we assume that the domain of all the observed
  variables is $[d]$ and the domain of all the hidden variables is
  $[k]$.

Define $\otimes$ to be the tensor product.
To denote the moments of observed variables, we will use the notation
\begin{align*}
  \mO_i &\eqdef \Pr(x_i), &
  \mO_{ij} &\eqdef \Pr(x_i, x_j), &
  \mO_{ijk} &\eqdef \Pr(x_i, x_j, x_k),
\end{align*}
and so on.
We will view $\mO_i, \mO_{ij}, \mO_{ijk}, \cdots$ as tensors of
  order 1, 2, 3, \ldots in $\Re^D, \Re^{D\times D}, \Re^{D \times
  D \times D}$.
For the moments of hidden variables, we will use analogous notation
\begin{align*}
  \mH_i &\eqdef \Pr(h_i), &
  \mH_{ij} &\eqdef \Pr(h_i, h_j), &
  \mH_{ijk} &\eqdef \Pr(h_i, h_j, h_k).
\end{align*}

\paragraph{Problem statement}

% Statement
This paper focuses on the problem of parameter estimation:
We are given $n$ i.i.d.~examples of the observed variables $\sD = (x^{(1)}, \dots, x^{(n)})$
where each $x^{(i)} \sim p_{\theta^*}$ for some true parameters $\theta^*$.
Our goal is to produce a parameter estimate $\hat\theta$ that approximates $\theta^*$.

% Maximum likelihood
The standard estimation procedure is maximum (marginal) likelihood,
  \begin{align*}
    \sL &\eqdef \max_{\theta \in \Re^d} \E_{x} \log p_\theta(x) \\
        &=      \max_{\theta \in \Re^d} \E_{x} \log \sum_z p_\theta(z),
  \end{align*}
  which is statistically efficient but computationally intractable
  because we must marginalize over latent variables $z$.
In practice, one uses gradient-based optimization procedures (e.g., EM
  or L-BFGS) on the marginal likelihood, which can get stuck in local
  optima.

\paragraph{Estimating three-view mixture models}

\begin{figure}[t]
  \label{fig:three-view}
  \centering
  % vim:ft=tex
\documentclass[tikz,convert={outfile=three-view.pdf}]{standalone}

\usepackage{scabby-diag}
%\usetikzlibrary{...}% tikz package already loaded by 'tikz' option
\begin{document}

\begin{tikzpicture}

% Hidden nodes
   \node[style=node, scale=0.8] (h) at (0,0) {$h_1$};
   \node[style=obsnode, scale=0.8] (x1) at (-1.0, -1.0) {$x_1$};
   \node[style=obsnode, scale=0.8] (x2) at (0.0, -1.0) {$x_2$};
   \node[style=obsnode, scale=0.8] (x3) at (1.0, -1.0) {$x_3$};

   \draw[-latex] (h) -- (x1);
   \draw[-latex] (h) -- (x2);
   \draw[-latex] (h) -- (x3);
\end{tikzpicture}

\end{document}

  \caption{A three-view mixture model}
\end{figure}

Recently, \citet{anandkumar12moments} presented a consistent algorithm
  to learn the parameters of a three-view mixture model
  (\figureref{three-view}) that sidesteps the local optima problem
  altogether. 
In the model, we have three observed variables $x_1, x_2, x_3 \in [d]$
  that are independent conditioned on the latent variable $h \in [k]$. 
  The parameters of this model are the conditional means $O^{(i)} \eqdef
  \E[x_i | h] \in \Re^{D \times K}$ for $i \in \{1,2,3\}$ and the mixture probabilities $\pi
  \in \Delta_{k-1}$.
  The key idea is that the observed moments $\mO_{12}, \mO_{23}, \mO_{13}$ and
  $\mO_{123}$ admit a factorization in terms of parameters;
\begin{align*}
  \mO_{ij} &= \sum_h \pi_h O^{(i)}_h \otimes O^{(j)}_h & \forall i,j \in \{1,2,3\} \\
  \mO_{123} &= \sum_h \pi_h O^{(1)}_h \otimes O^{(2)}_h \otimes O^{(3)}_h.
\end{align*}

\citet{anandkumar13tensor} presented a robust tensor power method, which
  we call $\TensorFactorize$, which converts consistent estimates of the observed moments 
  into consistent estimates of the parameters,
  $\pi, \mO_1, \mO_2, \mO_3$ (up to permutation of the columns), provided
  the following assumption holds;

\begin{assumption}
\label{asm:full-rank}
Assume the conditional moments $\mO_1, \mO_2, \mO_3$ have full column rank
  $k$, and that $\pi \succ 0$.
\end{assumption}

In brief, the algorithm first computes a whitening matrix $W \in
  \Re^{d \times k}$ such that $W^T \mO_{12} W = I_{k \times k}$,
  and uses it to transform $\mO_{123}$ into a symmetric orthogonal tensor.  
Then a robust tensor power method is applied to extract the eigenvectors
  of the whitened $\mO_{123}$; unwhitening yields the columns of $\mO_{3}$.
$\mO_1$ and $\mO_2$ can then be recovered with some algebra.

\todo{Mention {\em method of moments} somewhere?}

