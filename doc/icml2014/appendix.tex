\section{Proofs}
\label{app:proofs}

In the interest of space, we have omitted some proofs from
the main contents of the paper. We present their proofs in detail below.

\subsection{\lemmaref{pw-variance}}
\label{app:pw-variance-proof}

In \sectionref{piecewise}, we compare the asymptotic variance $\Sigma^\ml_\sC$ of the
composite likelihood estimator for a clique $\sC$, $\hat Z^\ml_\sC$, with
that of the pseudolikelihood estimator $\Sigma^\mom_\sC$.

\begin{proof}
  Using the delta-method \cite{vaart98asymptotic}, we have that the
  asymptotic distribution of $Z_\sC$ is,
  \begin{align*}
    \sqrt{n}(\hat Z_{\sC} - Z_{\sC}) &\convind \sN( 0, \grad^2 \sL_\ml^{-1} \Var[\grad \sL_\ml] \grad^2 \sL_\ml^{-1}).
  \end{align*}

Taking the first derivative,
\begin{align}
  \grad_{\mH_\sC} \sL_\ml(\sX_\sV) 
  &= \sum_{x \in \sD} \frac{\mOppAll[\vx]}{\mH_\sC \cdot \mOppAll[\vx]} \nonumber \\ 
  &= \mOppAll[\vx] \diag(\tilde \mO_{\sV})^{-1} \mO_{\sV}, \label{eqn:lhood-grad}
\end{align}
where $\tilde \mO_\sV$ is marginal distribution with parameters $\mH_\sC$, also represented as a vector in $\Re^{d^m}$.

Taking the second derivative.
\begin{align}
  \grad^2_{\mH_\sC} \sL_\ml(\Sx \sV) 
  &= \sum_{x \in \sD} \frac{\mOppAll[\vx] \mOppAllt[\vx]}{(\mH_\sC \cdot \mOppAll[\vx])^2} \nonumber \\
  &= \sum_{x \in \sD}\mOppAll[\vx] \mOppAllt[\vx] \frac{\mO_{\sV}[\vx]}{\tilde \mO_{\sV}^2[\vx]} \nonumber \\
  &= \mOppAll \diag(\mO_{\sV}) \diag(\tilde \mO_{\sV})^{-2} \mOppAllt. \label{eqn:lhood-hess}%
\end{align}

% DONE: don't need this
%It follows that $\grad^2_{\mH_\sC} \sL_\ml(\Sx \sV) \succ 0$ because
%$\tilde \mO_\sV, \tilde \mO_\sV \succ 0$ and $\mOppAll$ is
%full rank and stochastic.

% PL: this should just be a consequence
%Next, we show that it is
%strictly concave, which guarantees that it has a unique maximizer.

  From \equationref{lhood-grad}, we get
  \begin{align*}
    \Var [\grad \sL_\ml(\vec x_\sC)] &= \mOppAll \diag(\tilde M_\sV) \Sigma_\sV \diag(\tilde M_\sV) \mOppAll^T .
  \end{align*}

  Finally, using \equationref{lhood-hess}, we have
  \begin{align*}
    \Sigma_{Z_\sC} 
      &= \grad^2 \sL_\ml(\vec x_\sC)^{-1} \Var [\grad \sL_\ml(\vec x_\sC)] \grad^2 \sL_\ml(\vec x_\sC)^{-1}) \\
      &= \pinvt{\mOppAll} \diag(\tilde M_\sV) \Sigma_\sV \diag(\tilde M_\sV) \pinv{\mOppAll}.
  \end{align*}

  At the true parameters, $\tilde M_\sV = M_\sV$, completing the proof.
%  \todo{argue that asymptotic variance is finite, so the estimator is consistent (this is technically good form,
%but it's fine given space constraints}
\end{proof}

\subsection{Recovering conditional moments}

% Define up front that we will focus on h, x_1, x_2, x_3
In step 1 of \LearnMarginals, we used the bottleneck property of a hidden
  variable $h_i$ to learn conditional moments $\mOpp{v}{i}$ for every
  view $x_v \in \sV_{h_i}$ using \TensorFactorize. 
In order to do so, we require that \assumptionref{full-rank} holds, i.e.
\begin{assumption*}
  Given a bottleneck $h_1$ with views $x_1, x_2, x_3$, the conditional
  moments $\mOpp{1}{1}, \mOpp{2}{1}, \mOpp{3}{1}$ have full column rank
  $k$, and $\pi\oft 1 \succ 0$.
\end{assumption*}

In this section, we will show that the following condition
  (\assumptionref{full-rank-plus}) on the clique marginals suffices;
\begin{assumption*}
For every clique $\sC \in \sG$ (including ones involving observed variables),
  every $\Re^{k \times k}$ slice of the marginals $Z_\sC$ has full column
  rank $k$ constructed by summing out the remaining indices.
\end{assumption*}

%First, let us define some additional notation.
%
%\paragraph{Notation}
%
%Let $\sC = \{h_1, \cdots, h_m\}$ be a hidden clique in the graph, and
%  let $Z_\sC \eqdef \Pr(h_1, \cdots, h_m)$ be the tensor representing the
%  joint distribution. 
%We can obtain a marginal distribution for a sub-set $B \subset \sC$ by
%  marginalizing out the remaining variables. 
%Without loss of generality, let $|B| = m'$ and let the first $m'$
%  indices of $Z_\sC$ correspond to the $h_i \in B$; we can represent the
%  marginalization operation with linear algebra as $Z_B = Z_\sC(
%  \underbrace{I, \cdots, I}_{m' \text{times}}, \underbrace{\ones, \cdots, \ones}_{m - m'
%  \text{times}} )$. 
%
%Next, for a $m$-th order tensor $Z_\sC$ and $B \subset \sC$ with $m' \eqdef |B'|$, define the {\em
%partial diagonal} $\diag_{B}(Z_\sC)$ to be the $m - m' + 1$-th order tensor with
%a diagonal along the elements of $B$; $\diag_{B}(Z_\sC)[i, 
%i_{m'+1, \ldots, m}] = Z_\sC[\underbrace{i, \ldots, i}_{m' \text{times}}, i_{m'+1}, \ldots, m]$.
%
%
%\paragraph{Relation to clique marginals}

\begin{lemma}[Sufficiency of \assumptionref{full-rank-plus}]
  \label{lem:full-rank-suff}
  Given that \assumptionref{full-rank-plus} holds, then for any hidden
  variable $h_1$ and observed variable $x_v$, $\mPi{1} \succ 0$ and
  $\mOpp{v}{i}$ has full column rank.
\end{lemma}

\begin{proof}
Let us explicitly write out $\mPi{1}, \mOpp{1}{1}, \mOpp{2}{1},
\mOpp{3}{1}$ in terms of the clique marginals \{ $Z_\sC \given \sC \in \sG \}$.

\begin{figure}[t]
  \label{fig:assumption-2}
  \centering
  % vim:ft=tex
\documentclass[tikz,convert={outfile=assumption.pdf}]{standalone}
%\usetikzlibrary{...}% tikz package already loaded by 'tikz' option
\usepackage{scabby-diag}
\begin{document}

\begin{tikzpicture}

\xdefinecolor{darkgreen}{RGB}{0,100,0} %{175, 193, 36}

% Hidden nodes
   \node[style=node, scale=0.8] (h1) at (0,0) {$h_1$};
   \node[style=node, scale=0.8, below left= 0.5cm of h1] (h2) {$h_2$};
   \node[style=node, scale=0.8, below right= 0.5cm of h1] (h3) {$h_3$};
   \node[style=node, scale=0.8, below right= 0.5cm of h2] (h4) {$h_4$};

   \draw[-latex] (h1) -- (h2);
   \draw[-latex] (h1) -- (h3);
   \draw[-latex] (h2) -- (h4);
   \draw[-latex] (h3) -- (h4);

% Observed nodes
   \node[style=obsnode, style=inactive, scale=0.7, above left=0.3cm of h1] (x1a) {$x^a_1$};
   \node[style=obsnode, style=inactive, scale=0.7, above right=0.3cm of h1] (x1b) {$x^b_1$};
   \draw[-latex, style=inactive] (h1) -- (x1a);
   \draw[-latex, style=inactive] (h1) -- (x1b);

   \node[style=obsnode, style=inactive, scale=0.7, above left=0.3cm of h2] (x2a) {$x^a_2$};
   \node[style=obsnode, style=inactive, scale=0.7, below left=0.3cm of h2] (x2b) {$x^b_2$};
   \draw[-latex, style=inactive] (h2) -- (x2a);
   \draw[-latex, style=inactive] (h2) -- (x2b);

   \node[style=obsnode, style=inactive, scale=0.7, above right=0.3cm of h3] (x3a) {$x^a_3$};
   \node[style=obsnode, style=inactive, scale=0.7, below right=0.3cm of h3] (x3b) {$x^b_3$};
   \draw[-latex, style=inactive] (h3) -- (x3a);
   \draw[-latex, style=inactive] (h3) -- (x3b);
    
   \node[style=obsnode, scale=0.7, below left=0.3cm of  h4] (x4a) {$x^a_4$};
   \node[style=obsnode, style=inactive, scale=0.7, below right=0.3cm of h4] (x4b) {$x^b_4$};

   \draw[-latex] (h4) -- (x4a);
   \draw[-latex, style=inactive] (h4) -- (x4b);

   \node[below=0.7cm of h4] (eqn) {$O^{(4^a|1)} = O^{(4^a|4)} \times_{4} Y^{(4|2,3)} \times_{2,3} Y^{(2|1,3)} \times_{3} Y^{(3|1)} $};

\begin{pgfonlayer}{background}
\draw[line width=0.5pt, solid, black, fill=blue, opacity=80] 
                ($(h4.north) + (90:0.2cm)$) -- 
                ($(h4.west) + (180:0.2cm)$) -- 
                ($(x4a.west) + (180:0.2cm)$) -- 
                ($(x4a.south) + (-90:0.2cm)$) -- 
                ($(x4a.east) + (0:0.2cm)$) -- 
                ($(h4.east) + (0:0.2cm)$) -- 
                cycle;
\node[blue,anchor=east] at ($(x4a.west) + (180:0.3cm)$) {$O^{(4^a|4)}$};
\end{pgfonlayer}

\begin{pgfonlayer}{background}
\draw[line width=0.5pt, black, solid, fill=darkgreen, opacity=80] 
                %($(h4.south) + (-90:0.2cm)$) -- 
                ($(h4.west) + (180:0.2cm)$) -- 
                ($(h2.west) + (180:0.2cm)$) -- 
                ($(h2.north) + (90:0.2cm)$) -- 
                ($(h2.east) + (0:0.2cm)$) -- 
                ($(h3.west) + (180:0.2cm)$) -- 
                ($(h3.north) + (90:0.2cm)$) -- 
                ($(h3.east) + (0:0.2cm)$) -- 
                ($(h4.east) + (0:0.2cm)$) -- 
                cycle;
\node[darkgreen, anchor=west] at ($(h4.east) + (-30:0.3cm)$) {$Y^{(4|2,3)}$};
\end{pgfonlayer}

\begin{pgfonlayer}{background}
\draw[line width=0.5pt, black, solid, fill = magenta, opacity=80] 
                %($(h2.south) + (-90:0.2cm)$) -- 
                ($(h2.west) + (180:0.2cm)$) -- 
                ($(h1.north) + (90:0.2cm)$) -- 
                ($(h3.east) + (0:0.2cm)$) -- 
                %($(h3.south) + (-90:0.2cm)$) -- 
                ($(h3.west) + (180:0.2cm)$) -- 
                ($(h2.east) + (0:0.2cm)$) -- 
                cycle;
\node[magenta, anchor=east] at ($(h2.west) + (180:0.3cm)$) {$Y^{(2|1,3)}$};
\end{pgfonlayer}

\begin{pgfonlayer}{background}
\draw[line width=0.5pt, black, solid, fill = red, opacity=80,] 
                %($(h3.south) + (-90:0.2cm)$) -- 
                ($(h3.east) + (0:0.2cm)$) -- 
                ($(h1.north) + (90:0.2cm)$) -- 
                ($(h1.center) + (135:0.3cm) $) --
                ($(h3.center)$) --
                cycle;
\node[red, anchor=west] at ($(h1.east) + (30:0.3cm)$) {$Y^{(3|1)}$};
\end{pgfonlayer}

\end{tikzpicture}

\end{document}

  \caption{$\mOpp{v}{1}$ can be estimated by considering marginal distributions along any path from $h_1$ to $x_v$}
\end{figure}

Let $\sC_0 = \{h_1, \cdots, h_m\}$ be any clique containing
  $h_1$. $\mPi{1}$ can be obtained from $Z_\sC$ by marginalizing the
  remaining hidden variables in $\sC$, i.e. $\mPi{1} = Z_{\sC_0}(\cdot,
  \ones, \cdots, \ones)$.
Given that $\mPi{1}$ is strictly greater diagonal of some slice of $Z_{\sC_0}$,
  \assumptionref{full-rank-plus} guarantees that $\mPi{1} \succ 0$.

Expression $\mOpp{v}{1} = \Pr(x_v \given h_1)$ is a bit more
  complicated. 
By the reduction presented in \lemmaref{reduction}, we can assume
  w.l.o.g. that the observed variables only appear as leaf nodes with
  a single parent. 
Let $h_v$ be the parent of $x_v$, and let $h_{i_1}, h_{i_2}, \cdots, h_{i_l}$ be any
  path from $h_1 \eqdef h_{i_1}$ to $h_v \eqdef h_{i_l}$.
Then,
\begin{align*}
  \mOpp{v}{1} &\eqdef \Pr( x_v \given h_1 )  \\
              &= \sum_{h_v} \Pr( x_v \given h_v ) \Pr( h_v \given h_1 ) \\
              &= \sum_{h_{i_l}} \Pr( x_v \given h_{i_l} ) \sum_{h_{i_{l-1}}} \Pr( h_{i_{l}} \given h_{i_{l-1}} ) \Pr( h_{i_{l-1}} \given h_1 ) \\
              &= \sum_{\vh} \Pr( x_v \given h_{i_l} ) \Pr( h_{i_{l}} \given
              h_{i_{l-1}} ) \cdots \Pr( h_{i_2} \given h_{i_1} ).
\end{align*}

Let $C_{i_j}$ be the clique containing $h_{i_j}$ and $h_{i_{j-1}}$. 
The conditional distribution $\Pr( h_{i_j} \given h_{i_{j-1}} )$ can be
  got by first marginalizing all other hidden variables in $\sC$ (achieved by
  multiplying those indices by $\ones$), and then by normalizing by $\Pr(h_{i_{j-1}})$ (achieved by multiplying by the matrix $\inv{\diag(\mPi{i_{j-1}})}$); \todo{Better notation}
\begin{align*}
  Y_{j \given j-1} 
    &\eqdef \Pr( h_{i_j} \given h_{i_{j-1}} ) \\
    &= Z_{\sC_{i_j}}(\cdot, \cdot, \underbrace{\ones, \cdots, \ones}_{h \neq h_{i_j}, h_{i_{j-1}}})
        \inv{\diag(\mPi{i_{j-1}})}.
\end{align*}

Finally, we have the following expression,
\begin{align*}
  \mOpp{v}{1} &= \mOpp{v}{v} Y_{l \given l-1} \cdots Y_{2 \given 1}.
\end{align*}

Note that \assumptionref{full-rank-plus} guarantees that $Y_{j \given
  j -1}$ is full rank, from which it directly follows that $\mOpp{v}{1}$, completing our proof.
\end{proof}

\subsection{Sample Complexity}

In this section, we will explicitly derive the sample complexity of
  $\LearnMarginals$. The end result is an estimate of the error of
  each clique marginal $Z_\sC$.
The analysis proceeds in two stages; (i) bounding the error in
  estimating in each conditional moment $\mOpp{v}{i}$ and (ii) bounding
  the error in estimating $Z_\sC$.

\paragraph{Estimating $\mOpp{v}{i}$}

Consider the estimation of the conditional moments for a bottleneck
  $h_i$ with views $x_{v_1}, x_{v_2}, x_{v_3}$. 
Let the error in estimation of the moments be upper-bounded by
  $\epsilon$, $\|M_{v_i v_j} - \hat M_{v_i v_j} \|_F < \epsilon$ for $i,
  j \in \{1,2,3\}$, and 
the singular values of the moments be lower-bounded by $\alpha$,
  $\sigma_k(M_{v_i v_j}) > \alpha$ for $i, j \in \{1,2,3\}$.
Results from \citet{anandkumar12moments,anandkumar13tensor} show that 
\begin{align*}
  \|\mOpphat{v_1}{i} - \mOppit{v_1}{i}\|^2_F 
    &= O( 
    \frac{k {\mPi{i}}_{\max}^2}
    {\beta^5 } \epsilon ). 
\end{align*}
We will now express this quantity in terms of the parameters of the model.

Let $\sigma_1(Z_\sC)$ and $\sigma_k(Z_\sC)$ respectively be the largest
  and smallest singular value of any slice of $Z_\sC$. 
From the derivation in \lemmaref{full-rank-suff}, we know that $\mPi{i}$
  is obtained by marginalizing some slice of $Z_{\sC_j}$, which gives
  the result that $\sigma_1(Z_\sC) \succ \mPi^{i} \succ \sigma_k(Z_\sC)$.
Furthermore, 
\begin{align*}
  \sigma_k(Y_{j \given j-1}) &\ge \frac{\sigma_k(Z_{\sC_j})}{\mPi{i_{j-1}}_{\max}}
                             \ge \frac{\sigma_k(Z_{\sC_j})}{\sigma_1(Z_{\sC_j})}
                             = \cnd(Z_{\sC_j})^{-1},
\end{align*}
where $\cnd(Z_{\sC_j}) \eqdef
\frac{\sigma_1(Z_{\sC_j})}{\sigma_k(Z_{\sC_j})}$ is defined analogously
to the condition number of a matrix. 
Finally,
\begin{align*}
  \sigma_k(\mOpp{v}{1}) 
    &\ge \sigma_k(\mOpp{v}{v}) \prod_{j=2}^{l} \sigma_k(Y_{j \given j-1}) \\
    &\ge \sigma_k(\mOpp{v}{v}) \prod_{j=2}^{l} \cnd(\sC_{j}).
\end{align*}

Note that $M_{v_a v_b} = \mOpp{v_a}{i} \diag(\mPi{i})
\mOpp{v_b}{i}^\top$.
Without loss of generality, let us assume that $M_{v_a v_b}$ is the
  moment with the smallest singular value.
This gives us the following bound on $\beta$,
\begin{align*}
  \beta &\ge \sigma_k(M_{v_a,v_b}) \\
        &\ge \sigma_k(\mOpp{v_a}{i}) \mPi{i}_{\min} \sigma_k(\mOpp{v_b}{i}) \\
        &\ge \sigma_k(\mOpp{v_a}{v_a}) \sigma_k(\mOpp{v_b}{v_b}) \sigma_k(Z_{\sC_i}) \cnd_{\min}^{d_a + d_b},
\end{align*}
where $\cnd_{\min}$ is the smallest $\cnd(Z_\sC)$ for any $\sC \in \sG$
and $d_a, d_b$ are distance from $h_i$ to $x_{v_a}$ and $x_{v_b}$
respectively.

Finally, we have the result,
\begin{align*}
  \|\mOpphat{v_1}{i} - \mOppit{v_1}{i}\|^2_F 
    &= O( 
    \frac{k \sigma_1^2 }
    {(\sigma_k^3 \cnd_{\min}^{d_a + d_b})^5}
    \epsilon ). 
\end{align*}

\paragraph{Estimating $Z_\sC$}

Similarly, we can apply standard perturbation analysis techniques to get
an error bound on $Z_\sC$;
\begin{align*}
  \|Z_\sC - \hat Z_\sC\|_F 
  &\le \|M_\sV(\mOppit{v_1}{i_1}, \cdots, \mOppit{v_m}{i_m}) \\
      &\quad - \hat M_\sV(\hat{\mOppit{v_1}{i_1}}, \cdots, \hat{\mOppit{v_m}{i_m}})\|^2_F.
\end{align*}
Let $\|\mOppit{v_1}{i_1}\|_F < O$ and $\|\hat{\mOppit{v_1}{i_1}}\|_F < O$; then, we get
\begin{align*}
  \|Z_\sC - \hat Z_\sC\|_F 
  &\le \|M_\sV - \hat M_\sV\|_F O^m \\
  &\quad + \|\hat M_\sV\|_F O^{m-1} \max\{\|\hat {\mOppit{v_1}{i_1}} - \mOppit{v_1}{i_1}\|_F\}.
\end{align*}


% NOTE: This is made redundant, but maybe useful.
% Let $h_v$ be the parent of $x_v$. Then, 
% \begin{align*}
%   \mOpp{v}{1} &\eqdef \Pr( x_v \given h_1 )  \\
%               &= \sum_{h_v} \Pr( x_v \given h_v ) \Pr( h_v \given h_1 ) \\
%               &= \mOpp{v}{v} \mYpp{v}{1},
% \end{align*}
% where $\mYpp{i}{j} \eqdef \Pr( h_i \given h_j )$. 
% We will now give a recursive construction for $\mYpp{v}{1}$.
% 
% Let $\Succ(h_v)$ be the parents of a hidden variable $h_v$ in
%   a topological ordering of the hidden variables rooted at $h_1$. 
% Let $\sC_v$ be any clique containing $h_v$ and a subset of successors
%   $\Succ_{\sC_v}(h_v) \subseteq \Succ(h_v)$.
% Note that $\mYpp{v}{1}$ can be decomposed in terms of its successors;
% \begin{align*}
%   \mYpp{v}{1} &\eqdef \Pr( h_v \given h_1 )  \\
%   &= \sum_{\vh} \Pr( \Succ_{\sC_v}(h_v) \given h_1 ) \Pr( h_v \given \Succ_{\sC_v}(h_v) ).
% \end{align*}
% 
% $\Pr( h_v \given \Succ_{\sC_v}(h_v) )$ can be expressed in terms of the
% $Z_\sC$ by marginalizing out the hidden variables not in
% $\bar\sC_v \eqdef \Succ_{\sC_v}(h_v) \union \{h_v\}$;
% \begin{align*}
%   Z_{\bar\sC} &\eqdef \Pr( h_v \given \Succ_{\sC_v}(h_v) ) \\
%   &= Z_{\sC}(\underbrace{\cdot}_{\in \bar\sC}, \underbrace{\ones}_{\not\in \bar\sC}).
% \end{align*}
% 
% $\Pr( \Succ_{\sC_v}(h_v) \given h_1 )$ can then be similarly expressed
% in terms of its successors. \algorithmref{Y} describes the recursive process.
% 
% \begin{algorithm}
%   \caption{\mYpp{C,1}}
%   \label{algo:Y}
%   \begin{algorithmic}
%     \REQUIRE The root $h_1$, a group of hidden variables $C$ participating in a clique.
%     \ENSURE The hidden moments distribution $\mYpp{C}{1}$.
%     \STATE For successors $\Succ(h_i)$, find 
%       \STATE Return $\mH_\sC \gets \mO_{\Sx{\sV}}( \pinv{\mOpp{v_1}{i_i}}, \cdots, \pinv{\mOpp{v_m}{i_m}} )$.
%   \end{algorithmic}
% \end{algorithm}

