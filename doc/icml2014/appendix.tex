\section{Proofs}
\label{app:proofs}

In the interest of space, we have omitted some proofs from
the main contents of the paper. We present their proofs in detail below.

%\paragraph{Sample complexity for $Z_\sC$}
%
%Consider the estimation of the conditional moments for a bottleneck $h_i$ with views $x_{v_1},
%x_{v_2}, x_{v_3}$. Let the error in estimation of the moments (i.e.
%  $\|M_{v_1 v_2} - \hat M_{v_1 v_2} \|_F$, etc.)
%$\epsilon$. 
%Results from \citet{anandkumar12moments,anandkumar13tensor} show that 
%\begin{align*}
%  \|\mOpphat{v_1}{i} - \mOppit{v_1}{i}\|^2_F 
%    &= O( 
%    \frac{k {\pi\oft{i}}_{\max}^2}
%    {\sigma_k(M_{v_1,v_2})^5 } \epsilon ). 
%\end{align*}
%
%Similarly, we can apply standard perturbation analysis techniques to get
%an error bound on $Z_\sC$;
%\begin{align*}
%  \|Z_\sC - \hat Z_\sC\|_F 
%  &\le \|M_\sV(\mOppit{v_1}{i_1}, \cdots, \mOppit{v_m}{i_m})  - \hat M_\sV(\hat{\mOppit{v_1}{i_1}}, \cdots, \hat{\mOppit{v_m}{i_m}})\|^2_F.
%\end{align*}
%Let $\|\mOppit{v_1}{i_1}\|_F < O$ and $\|\hat{\mOppit{v_1}{i_1}}\|_F < O$; then, we get
%\begin{align*}
%  \|Z_\sC - \hat Z_\sC\|_F 
%  &\le \|M_\sV - \hat M_\sV\|_F O^m + \|\hat M_\sV\|_F O^{m-1} \max\{\|\hat {\mOppit{v_1}{i_1}} - \mOppit{v_1}{i_1}\|_F\}.
%\end{align*}

\subsection{\lemmaref{pw-variance}}
\label{app:pw-variance-proof}

In \sectionref{piecewise}, we compare the asymptotic variance $\Sigma^\ml_\sC$ of the
composite likelihood estimator for a clique $\sC$, $\hat Z^\ml_\sC$, with
that of the pseudolikelihood estimator $\Sigma^\mom_\sC$.

\begin{proof}
  Using the delta-method \cite{vaart98asymptotic}, we have that the
  asymptotic distribution of $Z_\sC$ is,
  \begin{align*}
    \sqrt{n}(\hat Z_{\sC} - Z_{\sC}) &\convind \sN( 0, \grad^2 \sL_\ml^{-1} \Var[\grad \sL_\ml] \grad^2 \sL_\ml^{-1}).
  \end{align*}

Taking the first derivative,
\begin{align}
  \grad_{\mH_\sC} \sL_\ml(\sX_\sV) 
  &= \sum_{x \in \sD} \frac{\mOppAll[\vx]}{\mH_\sC \cdot \mOppAll[\vx]} \nonumber \\ 
  &= \mOppAll[\vx] \diag(\tilde \mO_{\sV})^{-1} \mO_{\sV}, \label{eqn:lhood-grad}
\end{align}
where $\tilde \mO_\sV$ is marginal distribution with parameters $\mH_\sC$, also represented as a vector in $\Re^{d^m}$.

Taking the second derivative.
\begin{align}
  \grad^2_{\mH_\sC} \sL_\ml(\Sx \sV) 
  &= \sum_{x \in \sD} \frac{\mOppAll[\vx] \mOppAllt[\vx]}{(\mH_\sC \cdot \mOppAll[\vx])^2} \nonumber \\
  &= \sum_{x \in \sD}\mOppAll[\vx] \mOppAllt[\vx] \frac{\mO_{\sV}[\vx]}{\tilde \mO_{\sV}^2[\vx]} \nonumber \\
  &= \mOppAll \diag(\mO_{\sV}) \diag(\tilde \mO_{\sV})^{-2} \mOppAllt. \label{eqn:lhood-hess}%
\end{align}

% DONE: don't need this
%It follows that $\grad^2_{\mH_\sC} \sL_\ml(\Sx \sV) \succ 0$ because
%$\tilde \mO_\sV, \tilde \mO_\sV \succ 0$ and $\mOppAll$ is
%full rank and stochastic.

% PL: this should just be a consequence
%Next, we show that it is
%strictly concave, which guarantees that it has a unique maximizer.

  From \equationref{lhood-grad}, we get
  \begin{align*}
    \Var [\grad \sL_\ml(\vec x_\sC)] &= \mOppAll \diag(\tilde M_\sV) \Sigma_\sV \diag(\tilde M_\sV) \mOppAll^T .
  \end{align*}

  Finally, using \equationref{lhood-hess}, we have
  \begin{align*}
    \Sigma_{Z_\sC} 
      &= \grad^2 \sL_\ml(\vec x_\sC)^{-1} \Var [\grad \sL_\ml(\vec x_\sC)] \grad^2 \sL_\ml(\vec x_\sC)^{-1}) \\
      &= \pinvt{\mOppAll} \diag(\tilde M_\sV) \Sigma_\sV \diag(\tilde M_\sV) \pinv{\mOppAll}.
  \end{align*}

  At the true parameters, $\tilde M_\sV = M_\sV$, completing the proof.
%  \todo{argue that asymptotic variance is finite, so the estimator is consistent (this is technically good form,
%but it's fine given space constraints}
\end{proof}
