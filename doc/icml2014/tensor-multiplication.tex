\subsection{Tensor Operations}
\label{app:tensor-multiplication}

%In this section, we present basic results regarding two tensor
%operations we define, tensor multiplication, which is analogous to
%matrix multiplication, and the diagonal tensor.
In this section, we present basic results regarding tensor
multiplication, an operation analogous to matrix multiplication.
First, let us define some notation.

In the following, let $\sP \in \Re^{d_1 \times \ldots \times d_m}$ be
  a $m$-th order tensor and let $\sQ \in \Re^{d'_1 \times \ldots \times
  d'_n}$ be an $n$-th order tensor. %, where wlog, $m \ge n$. 
We use the notation $\Domain(\sP) = [d_1] \times \cdots \times [d_m]$
  to denote the domain of the indices of $\sP$.
Let $M$ be an index matching set, i.e. $M = \{ (a_1; b_1),
  (a_2, b_2), \ldots, (a_\ell, b_\ell) \}$, such that the corresopnding
  dimensions agree, i.e. $d_{a_i} = d'_{b_i}$ for all $i$. Let $D_{\ba}
  = [d_{a_1}] \times \cdots \times [d_{a_\ell}]$ and $D'_{\bb}
  = [d'_{b_1}] \times \cdots \times [d'_{b_\ell}]$ be the domains of the
  corresponding indices.

Finally, for some $\vi \in [d_1] \times \cdots \times [d_\ell]$ and $\vj
\in [d_{\ell + 1}] \times \cdots \times [d_m]$ and  we use the notation
$\sP[\vi \cdot \vj]_\pi \eqdef \sP[\pi(i_1, \cdot, i_\ell, j_{\ell+1},
\cdots j_m)]$ to conjoin indices $\vi$ and $\vj$ up to the permutation
$\pi$. 
%For our purposes, the specific ordering of the tensor's indices
%are not important and are usually referred to by variable naming.

% \begin{definition}[Diagonal Tensor]
%   The diagonalized tensor $\diag(\sP)$ is a $(2m)$-th order tensor
%   constructed by taking the tensor product along $I^{\otimes p}$,
%   \begin{align*}
%     \diag(\sP)[\vi,\vj] &\eqdef \sP[\vi] \delta[\vi,\vj],
%   \end{align*}
%   where $\vi, \vj \in [d_1] \times \cdots \times [d_m]$ and $\delta(\vi,
%   \vj) = \BI[\vi = \vj]$ is the Kronecker delta function.
% \end{definition}
% 
% Note that when $\sP$ is a vector, the diagonal tensor is a diagonal
% matrix, with entries given by the vector.

\begin{definition}[Tensor Multiplication]
  We define the tensor multiplication operation $\sR \eqdef P \times_{M}
  Q$ to be the $(m - \ell + n - \ell)$-th order tensor $\sR$ formed by
  conjoining $\sP$ and $\sQ$ followed by summing out the indices in $M$:
\begin{align*}
  \sR[\vi\cdot\vj] &\eqdef \sum_{\mathclap{\bu, \bv \in D_\ba \times D'_\bb}} \sP[\vi \cdot \bu]_{\pi_P} \sQ[\vj \cdot \bv]_{\pi_Q} ~ \delta_{u_1 v_1} \delta_{u_2 v_2} \ldots \delta_{u_\ell v_\ell},
\end{align*}
  for every $\vi \in \Domain(\sP) \setminus D_a$ and $\vj \in
  \Domain(\sQ) \setminus D'_b$, where $\pi_P$ and $\pi_Q$ are
  permutations of the indices that respect the matching $M$. For
  notational convenience, we will omit $\pi_P$, $\pi_Q$ in the sequel.
\end{definition}

Note that when $m, n = 2$, i.e. $\sP$ and $\sQ$ are matrices, and $M
  = \{(2,1)\}$, the operation is equivalent to matrix multiplication. 
Furthermore, similar to matrix multiplication, the operator defined above
  is associative and distributive. 
It is also commutative for symmetric tensors.

\paragraph{Unfoldings}
For a tensor $\sP$, as defined above, 
  given an index set $I \subset [m]$ we
  define an unfolding of a tensor $\sP$ along $I$,
 $\sP\munf{I}$, to be a matrix of dimensions $(\prod_{i \in I} d_i)
 \times (\prod_{j \not\in I} d_j)$. $\sP\munf{I}$ has elements
 $\sP\munf{I}[\vi;\vj] = \sP[\vi\cdot\vj]_{\pi}$, where $\pi$ is an
 appropriate permutation to align the indices of $\vi$ and $\vj$ with
 the ordering of $I$. 
For example, if $\sP$ were a $4$-th order tensor and $I = \{1,3\}$,
  $\sP\munf{13}[i_1,i_2; j_1,j_2] = \sP[i_1,j_1,i_2,j_2]$.
For notational convenience, we will omit the permutation $\pi$ subsequently.

The main result of this section is, similar to matrix multiplication, if
$\sR = \sP \times_{M} \sQ$, then the singular values of unfoldings
of $\sR$ are bounded by those of $\sP$ and $\sQ$.

\begin{lemma}[Tensor Multiplication and Singular Values]
  \label{lem:tensor-multiplication}
Let $\sP, \sQ$ be $m$-th order and $n$-th order tensors respectively, $M$ be an
  index mapping and $\sR = \sP \times_M \sQ$. 
Let $I \subset [m - k + n - k]$ be any arbitrary subset of the index set
  of $\sR$,
and let $I_P$, $I_Q$ be the partitioning of $I$ into subsets contained
  in $\sP$ and $\sQ$ respectively.
Then,
\begin{align*}
\sigma_{\max}(\sR\munf{I}) &\le \sigma_{\max}(\sP\munf{I_P}) \sigma_{\max}(\sQ\munf{I_P}) \\
\sigma_{\min}(\sR\munf{I}) &\ge \sigma_{\min}(\sP\munf{I_Q}) \sigma_{\min}(\sQ\munf{I_Q}).
\end{align*}
\end{lemma}

Note that in the matrix case, when $m = n = 2$ and $M=\{(2,1)\}$, i.e.
when $R = P Q$, we get the standard result that 
\begin{align}
\sigma_{\max}(R) &\le \sigma_{\max}(P) \sigma_{\max}(Q) \\
\sigma_{\min}(R) &\ge \sigma_{\min}(P) \sigma_{\min}(Q) \label{eqn:matrix-singular}.
\end{align}

\begin{proof}
  Without loss of generality, let $M = \{(1,1), \ldots, (\ell,\ell)\}$,
  and $D_\ell = [d_1] \times \cdots \times [d_\ell]$. 
  Then, 
  \begin{align*}
    \sR[\vi\cdot\vj] &\eqdef \sum_{\vk \in D_\ell} \sP[\vk\cdot\vi_P\cdot\vj_P] \sQ[\vk\cdot\vi_Q\cdot\vj_Q],
  \end{align*}
  where $\vi_P, \vi_Q$, are the components of $\vi$ in $\sP$ and $\sQ$
  respectively and $\vj_P, \vj_Q$ is similarly defined.

Consider an $(I)$-unfolding of $\sR$,
  \begin{align*}
    \sR\munf{I}[\vi;\vj] 
       &= \sum_{\vk \in D_\ell} \sP[\vk\cdot \vi_P \cdot \vj_P] \sQ[\vk\cdot \vi_Q \cdot \vj_Q] \\
       &= \sum_{\vk \in D_\ell} \sP\munf{I_P}[\vi_P; \vj_P \cdot \vk] \sQ\munf{I_Q}[\vj_Q; \vj_Q \cdot \vk].
  \end{align*}
Note that the indices that were summed out belong in the complement of $I_P$ and $I_Q$ respectively.

The key idea is to represent the above expression as matrix multiplication.
To do so, we can temporarily ``inflate'' the unfoldings $\sP\munf{I_P}$
  and $\sQ\munf{I_Q}$ by taking the outer product of $\sP\munf{I_P}$ with
  $|I_Q|$ identity matrices, and vice-versa. Define 
\begin{align*}
  \sP' &\eqdef \sP \otimes I_{d'_1} \otimes \cdots \otimes I_{d'_{|I_Q|}} \\
  \sQ' &\eqdef \sQ \otimes I_{d'_1} \otimes \cdots \otimes I_{d'_{|I_P|}}.
\end{align*}

Now, $\sP'$ is a $(m + 2 |I_Q|)$-th order tensor and similarly $\sQ'$
  is a $(n + 2 |I_P|)$-th order tensor. 
For every identity matrix we tensor $\sP$ and $\sQ$ by we include one of
  its indices to include in the new set $I'_P$ and $I'_Q$ respectively.
  Note that $\sP\munf{I'_P}$ is a $(I_P + I_Q)\times (m - |I_P| + |I_Q| + \ell)$ matrix, and 
$\sP\munf{I'_Q}$ is $(I_P + I_Q) \times (n - |I_Q| + |I_P| + \ell)$
Then,
\begin{align*}
    \sR\munf{I}[\vi;\vj] 
    &= \sum_{\be_P, \be_Q \vk}
    {\sP'}\munf{I'_P}[\vi_P\be'_Q; \vj_P \be_Q \vk]  
    {\sQ'}\munf{I'_Q}[\vi_Q\be'_P; \vj_Q \be_P \vk] \\
    \sR\munf{I} &= {\sP'}\munf{I'_P} {\sQ'}\munf{I'_Q}.
\end{align*}

Using lemma \lemmaref{tensor-prod} and the bound on matrix singular
values, \equationref{matrix-singular}, we get,
\begin{align*}
\sigma_{\max}(\sR\munf{I}) 
    &\le \sigma_{\max}({\sP'}\munf{I'_P}) \sigma_{\max}({\sQ'}\munf{I'_Q}) \\
      &= \sigma_{\max}(\sP\munf{I_P}) \sigma_{\max}(\sQ\munf{I_Q}) \\
    \sigma_{\min}(\sR\munf{I})  
    &\ge \sigma_{\min}({\sP'}\munf{I'_P}) \sigma_{\min}({\sQ'}\munf{I'_Q}) \\
      &= \sigma_{\min}(\sP\munf{I_P}) \sigma_{\min}(\sQ\munf{I_Q}).
\end{align*}
\end{proof}

For completeness, we also provide a proof for the following standard
lemma regarding the tensor product of matrices.

\begin{lemma}[Singular values of the tensor products of matrices]
  \label{lem:tensor-prod}
  Let $A = \Un1 \Sigma_1 \Vnt1$ and $B = \Un2 \Sigma_2 \Vnt2$ be two
  matrices and their singular value decompositions, 
  with $\Sigma_1 = \diag(\sigma_{11}, \ldots, \sigma_{1m})$ and $\Sigma_2
  = \diag(\sigma_{21}, \ldots, \sigma_{2n})$. 
  Then, for $C = A \otimes B$, the matrix unfolding $C\munf{13}$ is full
  rank, with singular values $\sigma_{11} \sigma_{21}, \ldots, \sigma_{1m}
  \sigma_{2n}$.
\end{lemma}
\begin{proof}
  An alternative representation of $A$ and $B$ (and consequently $C$) is as follows
  \begin{align*}
    A &= \sum_{i=1}^m \sigma_{1i}~ \un{1}_i \otimes \vn{1}_i \\
    B &= \sum_{j=1}^n \sigma_{2j}~ \un{2}_j \otimes \vn{2}_j \\
    C^{13} &= \sum_{i=1}^m \sum_{j=1}^n \sigma_{1i} \sigma_{2j}~ (\un{1}_i \otimes \un{2}_j)(\vn{1}_i \otimes \vn{2}_j)^\top.
  \end{align*}

  For the unfolding $\{1,3\}$ which pairs $\un{1}$ and $\un{2}$, we have
basis elements $\un{1} \otimes \un{2}$ that are orthogonal to each
other, and thus the above is a valid singular value decomposition,
with singular values $\{ \sigma_{1i} \sigma_{2j} \}_{i=1,j=1}^{i=m,j=m}$. 
\end{proof}

