\subsection{Tensor Multiplication}
\label{app:tensor-multiplication}

In this section, we present basic results regarding tensor
multiplication. First, let us review definitions.

Let $\sP \in \Re^{d_1 \times \ldots \times d_m}$ be a $m$-th order tensor and let $\sQ \in \Re^{d'_1 \times \ldots \times d'_n}$ be an $n$-th order tensor, where wlog, $m \ge n$. 
Let $M$ be an index matching set $M = \{ (a_1; b_1),
  (a_2, b_2), \ldots, (a_\ell, b_\ell) \}$, such that the corresopnding
  dimensions agree, i.e. $d_{a_i} = d'_{b_i}$ for all $i$. Let $D_{\ba}
  = [d_{a_1}] \times \cdots \times [d_{a_\ell}]$ and $D'_{\bb}
  = [d'_{a_1}] \times \cdots \times [d'_{a_\ell}]$ be the domains of the
  corresponding indices.
We define the tensor multiplication operation $\sR \eqdef P \times_{M}
  Q$ to be the $(m - \ell + n - \ell)$-th order tensor formed by
  conjoining $\sP$ and $\sR$ followed by summing out the indices in $M$:
\begin{align*}
  \sR[\vi,\vj] &\eqdef \sum_{\bu, \bv \in D_\ba \times D'_\bb} \sP[\vi,\bu] \sQ[\bv ,\vj] ~ \delta_{u_1 v_1} \delta_{u_2 v_2} \ldots \delta_{u_\ell v_\ell}.
\end{align*}

Note that when $m, n = 2$, i.e. $\sP$ and $\sQ$ are matrices, and $M
  = {(2,1)}$, the operation is equivalent to matrix multiplication. 
Furthermore, similar to matrix multiplication, the operator defined above
  is associative and distributive. 
It is also commutative for symmetric tensors.

Given an index set $I \subset [m]$ we
  define an unfolding of a tensor $\sP$ along $I$,
 $\sP\munf{I}$, to be a matrix of dimension $(\prod_{i \in I} d_i)
 \times (\prod_{j \not\in I} d_j)$. $\sP\munf{I}$ has elements
 $\sP\munf{I}[\vi;\vj] = \sP[\pi(\vi\vj)]$, where $\pi$ is an
 appropriate permutation to align the indices of $\vi$ and $\vj$ with
 the ordering of $I$. 
For example, if $\sP$ were a $4$-th order tensor and $I = \{1,3\}$,
  $\sP\munf{13}[i_1,i_2; j_1,j_2] = \sP[i_1,j_1,i_2,j_2]$.
For notational convenience, we will omit the permutation $\pi$ subsequently.


The main result of this section is, similar to matrix multiplication, if
$\sR = \sP \times_{M} \sQ$, then the singular values of mode-unfolding
of $\sR$ are bounded by those of $\sP$ and $\sQ$.

\begin{lemma}[Tensor Multiplication and Singular Values]
  \label{lem:tensor-multiplication}
Let $\sP, \sQ$ be $n$-th order and $m$-th order tensors, $M$ be an
  admissible index mapping and $\sR = \sP \times_M \sQ$. 
Let $I \subset [m - k + n - k]$ be an arbitrary subset of the index set
  of $\sR$,
and let $I_1$, $I_2$ be the respective subsets of $I$ contained in $\sP$
  and $\sQ$.
Then,
\begin{align*}
\sigma_{\max}(\sR\munf{I}) &\le \sigma_{\max}(\sP\munf{I_1}) \sigma_{\max}(\sQ\munf{I_2}) \\
\sigma_{\min}(\sR\munf{I}) &\ge \sigma_{\min}(\sP\munf{I_1}) \sigma_{\min}(\sQ\munf{I_2}).
\end{align*}
\end{lemma}

Note that in the matrix case, when $m = n = 2$, when $R = P Q^\top$, we get the standard result that 
\begin{align}
\sigma_{\max}(R) &\le \sigma_{\max}(P) \sigma_{\max}(Q) \\
\sigma_{\min}(R) &\ge \sigma_{\min}(P) \sigma_{\min}(Q) \label{eqn:matrix-singular}.
\end{align}

\begin{proof}
  Without loss of generality, let $M = \{(1,1), \ldots, (\ell,\ell)\}$,
  and $D = [d_1] \times \cdots \times [d_\ell]$. 
  Then,
  \begin{align*}
    \sR[\vi\vj] &\eqdef \sum_{\vk \in D_\ell} \sP[\vk \vi] \sQ[\vk \vj].
  \end{align*}

Consider an $(I)$-unfolding of $\sR$,
  \begin{align*}
    \sR\munf{I}[\vi,\vj] 
       &= \sum_{\vk} \sP[\vk \vi] \sQ[\vk \vj] \\
       &= \sum_{\vk} \sP\munf{I_1}[\vx_1; \vy_1, \vk] \sQ\munf{I_2}[\vx_2; \vy_2, \vk],
  \end{align*}
  where $\vx_1 \in \Range(I_1)$, $\vy_1 \in \Range([m] \setminus I_1)$, $\vx_2 \in \Range(I_2)$ and $\vy_2  \in \Range([n] \setminus I_2)$.

The key idea is to represent the above as a matrix multiplication.
To do so, we can temporarily ``inflate'' the unfoldings $\sP\munf{I_1}$
  and $\sQ\munf{I_2}$ by taking the outer product of $\sP\munf{I_1}$ with
  $|I_2|$ identity matrices, and vice-versa. Define 
\begin{align*}
  \sP' &\eqdef \sP \otimes I_{d'_1} \otimes \cdots \otimes I_{d'_{|I_2|}} \\
  \sQ' &\eqdef \sQ \otimes I_{d'_1} \otimes \cdots \otimes I_{d'_{|I_1|}}.
\end{align*}

Now, $\sP'$ is a $(m + 2 * |I_2|)$-th order tensor and similarly $\sQ'$
  is a $(n + 2 * |I_1|)$-th order tensor. We choose new index sets
  $I'_1$ and $I'_2$ to include exactly one of these identity indices. 
Then,
\begin{align*}
    \sR\munf{I}[\vi,\vj] 
    &= \sum_{\vy_1, \vy_2 \vk}
    {\sP'}\munf{I'_1}[\vx_1\vy'_2; \vy_1 \vy_2 \vk]  
    {\sQ'}\munf{I'_2}[\vx_2\vy'_1; \vy_1 \vy_2 \vk] \\
    \sR\munf{I} &= {\sP'}\munf{I'_1} {\sQ'}\munf{I'_2}.
\end{align*}

Using lemma \lemmaref{tensor-prod} and the bound on matrix singular
values, \equationref{matrix-singular}, we get,
\begin{align*}
\sigma_{\max}(\sR\munf{I}) 
    &\le \sigma_{\max}({\sP'}\munf{I'_1}) \sigma_{\max}({\sQ'}\munf{I'_2}) \\
      &= \sigma_{\max}(\sP\munf{I_1}) \sigma_{\max}(\sQ\munf{I_2}) \\
    \sigma_{\min}(\sR\munf{I})  
    &\ge \sigma_{\min}({\sP'}\munf{I'_1}) \sigma_{\min}({\sQ'}\munf{I'_2}) \\
      &= \sigma_{\min}(\sP\munf{I_1}) \sigma_{\min}(\sQ\munf{I_2}).
\end{align*}
\end{proof}

\begin{lemma}[Singular values of the tensor products of matrices]
  \label{lem:tensor-prod}
  Let $A = \Un1 \Sigma_1 \Vnt1$ and $B = \Un2 \Sigma_2 \Vnt2$ be two
  matrices and their singular value decompositions, 
  with $\Sigma_1 = \diag(\sigma_{11}, \ldots, \sigma_{1m})$ and $\Sigma_2
  = \diag(\sigma_{21}, \ldots, \sigma_{2n})$. 
  Then, for $C = A \otimes B$, the matrix unfolding $C\munf{13}$ is full
  rank, with singular values $\sigma_{11} \sigma_{21}, \ldots, \sigma_{1m}
  \sigma_{2n}$.
\end{lemma}
\begin{proof}
  An alternative representation of $A$ and $B$ (and consequently $C$) is as follows
  \begin{align*}
    A &= \sum_{i=1}^m \sigma_{1i}~ \un{1}_i \otimes \vn{1}_i \\
    B &= \sum_{j=1}^n \sigma_{2j}~ \un{2}_j \otimes \vn{2}_j \\
    C^{13} &= \sum_{i=1}^m \sum_{j=1}^n \sigma_{1i} \sigma_{2j}~ (\un{1}_i \otimes \un{2}_j)(\vn{1}_i \otimes \vn{2}_j)^\top.
  \end{align*}

  For the unfolding $\{1,3\}$ which pairs $\un{1}$ and $\un{2}$, we have
basis elements $\un{1} \otimes \un{2}$ that are orthogonal to each
other, and thus the above is a valid singular value decomposition,
with singular values $\{ \sigma_{1i} \sigma_{2j} \}_{i=1,j=1}^{i=m,j=m}$. 
\end{proof}

